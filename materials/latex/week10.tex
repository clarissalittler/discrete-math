\section{Week 10: Analysis of Algorithm Efficiency}
\subsection*{Reading}
Epp \S 11.1--11.5.

\subsection*{Learning objectives}
\begin{itemize}
  \item Compare growth rates using limits and dominance.
  \item Use big-$O$, big-$\Omega$, and big-$\Theta$ definitions.
  \item Analyze simple loops and summations.
  \item Solve basic recurrences using expansion or master theorem.
\end{itemize}

\subsection*{Key definitions and facts}
\begin{definition}[Big-$O$]
$f(n)=O(g(n))$ if there exist constants $c,n_0$ such that
$0\le f(n) \le c g(n)$ for all $n\ge n_0$.
\end{definition}

\begin{definition}[Big-$\Theta$]
$f(n)=\Theta(g(n))$ if $f(n)=O(g(n))$ and $f(n)=\Omega(g(n))$.
\end{definition}

\subsection*{Worked example}
\begin{example}
Analyze the runtime of a loop that runs $n$ times and inside runs a loop $n$ times.
\emph{Sketch.} The total number of iterations is $n^2$, so the runtime is $\Theta(n^2)$.
\end{example}

\subsection*{Practice}
\begin{enumerate}
  \item Order the functions $n\log n$, $n^{1.5}$, $2^n$, $n^3$ by growth rate.
  \item Show that $3n^2+5n+7$ is $\Theta(n^2)$.
  \item Solve the recurrence $T(n)=2T(n/2)+n$ with $T(1)=1$.
  \item Analyze the runtime of binary search.
\end{enumerate}
