\section{Week 8: Trees and Graph Algorithms}
\subsection*{Reading}
Epp \S 10.4--10.6.

\subsection*{Learning objectives}
\begin{itemize}
  \item Identify trees, rooted trees, and m-ary trees.
  \item Use basic properties such as $|E|=|V|-1$ for trees.
  \item Find spanning trees in connected graphs.
  \item Apply shortest-path algorithms conceptually.
\end{itemize}

\subsection*{Key definitions and facts}
\begin{definition}[Tree]
A tree is a connected graph with no simple circuits.
Equivalently, a tree on $n$ vertices has exactly $n-1$ edges.
\end{definition}

\begin{definition}[Spanning tree]
A spanning tree of a graph $G$ is a subgraph that is a tree containing all vertices of $G$.
\end{definition}

\subsection*{Worked example}
\begin{example}
Prove that a tree on $n$ vertices has $n-1$ edges.
\emph{Sketch.} Use induction on $n$ by removing a leaf and its incident edge.
\end{example}

\subsection*{Practice}
\begin{enumerate}
  \item How many leaves can a full $m$-ary tree of height $h$ have?
  \item Find a spanning tree of the complete graph $K_5$.
  \item Explain why removing any edge from a tree disconnects it.
  \item Run Dijkstra's algorithm on a weighted graph with 5 vertices of your choice.
\end{enumerate}
