\section{Week 4: Counting and Probability I}
\subsection*{Reading}
Epp \S 9.1--9.4.

\subsection*{Learning objectives}
\begin{itemize}
  \item Build sample spaces and events for basic probability models.
  \item Apply the multiplication rule to count outcomes.
  \item Apply the addition rule and inclusion--exclusion for two sets.
  \item Use the pigeonhole principle to force collisions.
\end{itemize}

\subsection*{Key definitions and facts}
\begin{definition}[Sample space and event]
A sample space $S$ is the set of all outcomes. An event is a subset of $S$.
For equally likely outcomes, $P(E)=|E|/|S|$.
\end{definition}

\begin{theorem}[Addition rule]
For events $A,B$, $|A\cup B|=|A|+|B|-|A\cap B|$.
\end{theorem}

\begin{theorem}[Pigeonhole principle]
If $n+1$ objects are placed into $n$ boxes, then some box contains at least two objects.
\end{theorem}

\subsection*{Worked example}
\begin{example}
How many 3-letter strings over $\{A,B,C,D\}$ have no repeated letters?
\emph{Sketch.} Multiplication rule: $4\cdot 3\cdot 2=24$.
\end{example}

\subsection*{Practice}
\begin{enumerate}
  \item A fair die is rolled twice. What is the probability the sum is $7$?
  \item How many 5-bit binary strings contain at least one $1$?
  \item Use inclusion--exclusion to count integers in $\{1,\dots,100\}$ divisible by $2$ or $3$.
  \item Use the pigeonhole principle to show that among 13 people, two share a birth month.
\end{enumerate}
