\section{Chapter 0: Proof Refresher}

\textit{Or: ``How to convince someone you're right, rigorously''}

This chapter reviews the proof techniques and logical foundations from the first quarter. Think of it as a reference card you can flip back to throughout the course. If anything here feels unfamiliar rather than ``oh right, that,'' you may want to spend extra time with the readings.

\subsection*{Propositional Logic Review}

Let's start with the building blocks. A \textbf{proposition} is a statement that is either true or false---no maybes, no ``it depends,'' no quantum superpositions. ``It's raining'' is a proposition. ``Is it raining?'' is not (it's a question). ``This statement is false'' is not (it's a paradox, and we don't allow those).

\begin{definition}[Logical connectives]
Let $P$ and $Q$ be propositions.

\begin{center}
\begin{tabular}{clc}
\textbf{Symbol} & \textbf{Name} & \textbf{True when...} \\
\hline
$\neg P$ & Negation (NOT) & $P$ is false \\
$P \land Q$ & Conjunction (AND) & Both $P$ and $Q$ are true \\
$P \lor Q$ & Disjunction (OR) & At least one of $P$, $Q$ is true \\
$P \to Q$ & Implication (IF-THEN) & $P$ is false, or $Q$ is true \\
$P \leftrightarrow Q$ & Biconditional (IFF) & $P$ and $Q$ have the same truth value \\
\end{tabular}
\end{center}
\end{definition}

The one that trips everyone up is implication. ``$P \to Q$'' is true whenever $P$ is false \emph{or} $Q$ is true. This means ``if pigs fly, then I'm the queen of England'' is a \emph{true} statement (because pigs don't fly). This feels weird, but it's the definition that makes mathematics work. We'll just have to live with it.

\begin{theorem}[Key equivalences]
The following are logically equivalent (have the same truth table):
\begin{align*}
P \to Q &\equiv \neg P \lor Q \equiv \neg Q \to \neg P \quad \text{(contrapositive)} \\
\neg(P \land Q) &\equiv \neg P \lor \neg Q \quad \text{(De Morgan)} \\
\neg(P \lor Q) &\equiv \neg P \land \neg Q \quad \text{(De Morgan)} \\
P \to Q &\not\equiv Q \to P \quad \text{(converse is NOT equivalent!)}
\end{align*}
\end{theorem}

That last line deserves emphasis. ``If it's raining, the ground is wet'' does \emph{not} mean ``if the ground is wet, it's raining.'' Someone could have turned on the sprinklers. The converse of a true statement can be false. This mistake shows up constantly in student proofs.

\subsection*{Predicate Logic Review}

Propositional logic lets us talk about specific statements. But what about ``all even numbers are divisible by 2'' or ``there exists a prime greater than 100''? For that, we need \textbf{quantifiers}.

\begin{definition}[Quantifiers]
Let $P(x)$ be a \textbf{predicate}---a statement that depends on a variable $x$ and becomes true or false once you plug in a specific value.

\textbf{Universal quantifier:} $\forall x \in D.\, P(x)$ means ``for all $x$ in domain $D$, $P(x)$ holds.'' To prove this, you need to show $P(x)$ for \emph{every} possible $x$. No exceptions.

\textbf{Existential quantifier:} $\exists x \in D.\, P(x)$ means ``there exists some $x$ in $D$ such that $P(x)$ holds.'' To prove this, you need to find \emph{one} example.
\end{definition}

\begin{theorem}[Negating quantifiers]
Here's how negation interacts with quantifiers:
\begin{align*}
\neg(\forall x.\, P(x)) &\equiv \exists x.\, \neg P(x) \\
\neg(\exists x.\, P(x)) &\equiv \forall x.\, \neg P(x)
\end{align*}

In English: the negation of ``all cats are black'' isn't ``no cats are black''---it's ``there exists a cat that isn't black.'' You only need one counterexample to disprove a universal claim.
\end{theorem}

\begin{theorem}[Quantifier order matters]
\[
\forall x.\, \exists y.\, P(x,y) \quad \not\equiv \quad \exists y.\, \forall x.\, P(x,y)
\]
This is subtle but crucial. Let $P(x,y)$ mean ``$y > x$'' over $\mathbb{R}$:
\begin{itemize}
  \item $\forall x.\, \exists y.\, (y > x)$: ``For every number, there's a larger one.'' \textbf{True.} (Given any $x$, take $y = x + 1$.)
  \item $\exists y.\, \forall x.\, (y > x)$: ``There's a number larger than all others.'' \textbf{False.} (No largest real number exists.)
\end{itemize}

The difference: in the first, $y$ can depend on $x$ (different $x$, different $y$). In the second, you must pick a single $y$ that works for \emph{all} $x$ simultaneously. That's a much stronger requirement.
\end{theorem}

\subsection*{Proof Techniques}

Now for the main event: how to actually prove things. There are several standard approaches, and part of mathematical maturity is recognizing which technique fits which situation.

\needspace{12\baselineskip}
\begin{strategy}[Direct proof]
To prove $P \to Q$: Assume $P$ is true, then show $Q$ follows.

This is the most straightforward approach. You start with what you're given and push forward until you reach what you want.

\textbf{Template:}
\begin{quote}
\emph{Assume $P$. [Reasoning...] Therefore $Q$.}
\end{quote}

\textbf{Example.} Prove: If $n$ is even, then $n^2$ is even.

\emph{Proof.} Assume $n$ is even. Then $n = 2k$ for some integer $k$. (This is what ``even'' means---we're unpacking the definition.) So $n^2 = (2k)^2 = 4k^2 = 2(2k^2)$. Since $2k^2$ is an integer, $n^2$ has the form $2 \cdot (\text{integer})$, which is the definition of even. \qed
\end{strategy}

\begin{strategy}[Proof by contrapositive]
To prove $P \to Q$: Prove $\neg Q \to \neg P$ instead (they're logically equivalent).

Why would you do this? Sometimes $\neg Q$ gives you more to work with than $P$ does. If you're stuck on a direct proof, try the contrapositive.

\textbf{Template:}
\begin{quote}
\emph{We prove the contrapositive. Assume $\neg Q$. [Reasoning...] Therefore $\neg P$.}
\end{quote}

\textbf{Example.} Prove: If $n^2$ is odd, then $n$ is odd.

\emph{Proof.} We prove the contrapositive: if $n$ is even, then $n^2$ is even.

Assume $n$ is even. Then $n = 2k$ for some integer $k$. So $n^2 = 4k^2 = 2(2k^2)$, which is even. \qed

(Notice this is almost the same as the previous example. The contrapositive often turns a ``weird'' statement into a more natural one.)
\end{strategy}

\begin{strategy}[Proof by contradiction]
To prove $P$: Assume $\neg P$ and derive a contradiction.

This is the ``suppose not'' approach. You assume the opposite of what you want to prove, then show this leads somewhere impossible. The logic is: if assuming $\neg P$ leads to nonsense, then $P$ must be true.

\textbf{Template:}
\begin{quote}
\emph{Suppose for contradiction that $\neg P$. [Reasoning...] This contradicts [known fact]. Therefore $P$.}
\end{quote}

\textbf{Example.} Prove: $\sqrt{2}$ is irrational.

\emph{Proof.} Suppose for contradiction that $\sqrt{2}$ is rational. Then $\sqrt{2} = a/b$ where $a, b$ are integers with no common factors (we've reduced to lowest terms).

Squaring both sides: $2 = a^2/b^2$, so $a^2 = 2b^2$. This means $a^2$ is even, so $a$ is even (we proved this above). Write $a = 2c$ for some integer $c$.

Substituting: $(2c)^2 = 2b^2$, so $4c^2 = 2b^2$, so $b^2 = 2c^2$. This means $b^2$ is even, so $b$ is even.

But now both $a$ and $b$ are even---they share a factor of 2. This contradicts our assumption that $a/b$ was in lowest terms. \qed

(This is one of the oldest proofs in mathematics, attributed to the ancient Greeks. Legend has it that Hippasus was drowned for revealing it.)
\end{strategy}

\begin{strategy}[Proof by cases]
To prove $P$: Partition into exhaustive cases and prove $P$ in each one.

Sometimes a statement is true for different reasons in different situations. Rather than finding one unified argument, you can just handle each situation separately.

\textbf{Template:}
\begin{quote}
\emph{Case 1: [Condition]. [Proof of $P$ in this case.]}\\
\emph{Case 2: [Condition]. [Proof of $P$ in this case.]}\\
\emph{These cases are exhaustive, so $P$ holds.}
\end{quote}

\textbf{Example.} Prove: For any integer $n$, $n^2 + n$ is even.

\emph{Proof.} Every integer is either even or odd. We handle each case.

\emph{Case 1: $n$ is even.} Then $n^2$ is even (proved earlier). So $n^2 + n$ is even + even = even. \checkmark

\emph{Case 2: $n$ is odd.} Then $n^2$ is odd (similar argument). So $n^2 + n$ is odd + odd = even. \checkmark

Since every integer is either even or odd, we've covered all cases. \qed
\end{strategy}

\subsection*{Mathematical Induction}

Induction is how we prove statements about all natural numbers (or anything else we can line up in a sequence). The idea is beautifully simple: prove the first case, then prove that each case implies the next. Like dominoes.

\begin{strategy}[Weak induction]
To prove $\forall n \geq n_0.\, P(n)$:
\begin{enumerate}
  \item \textbf{Base case:} Prove $P(n_0)$.
  \item \textbf{Inductive step:} Prove that $P(k) \to P(k+1)$ for an arbitrary $k \geq n_0$.
\end{enumerate}

\textbf{Why it works:} The base case establishes $P(n_0)$. The inductive step gives us $P(n_0) \to P(n_0+1)$, so $P(n_0+1)$ is true. The inductive step again gives $P(n_0+1) \to P(n_0+2)$, so $P(n_0+2)$ is true. And so on, forever. We've built an infinite chain of implications, starting from a known truth.
\end{strategy}

\begin{example}
Prove: $\displaystyle\sum_{i=1}^{n} i = \frac{n(n+1)}{2}$ for all $n \geq 1$.

\emph{Proof.} By induction on $n$.

\textbf{Base case} ($n = 1$): The left side is $\sum_{i=1}^{1} i = 1$. The right side is $\frac{1 \cdot 2}{2} = 1$. They match. \checkmark

\textbf{Inductive step:} Assume the formula holds for some $k \geq 1$:
\[
\sum_{i=1}^{k} i = \frac{k(k+1)}{2} \quad \text{(this is our \textbf{inductive hypothesis})}
\]

We need to prove it holds for $k+1$:
\begin{align*}
\sum_{i=1}^{k+1} i &= \left(\sum_{i=1}^{k} i\right) + (k+1) && \text{(peel off the last term)} \\
&= \frac{k(k+1)}{2} + (k+1) && \text{(apply inductive hypothesis!)} \\
&= \frac{k(k+1) + 2(k+1)}{2} && \text{(common denominator)} \\
&= \frac{(k+1)(k+2)}{2} && \text{(factor)}
\end{align*}

This is exactly the formula with $n = k+1$. \checkmark

By induction, the formula holds for all $n \geq 1$. \qed
\end{example}

The key move in the inductive step is the second line: we \emph{used} the inductive hypothesis. If your inductive step doesn't reference the assumption $P(k)$, something's wrong---either your proof has a bug, or you didn't need induction in the first place.

\begin{strategy}[Strong induction]
To prove $\forall n \geq n_0.\, P(n)$:
\begin{enumerate}
  \item \textbf{Base case(s):} Prove $P(n_0)$ (and possibly several more).
  \item \textbf{Inductive step:} Assume $P(n_0), P(n_0+1), \ldots, P(k)$ all hold. Prove $P(k+1)$.
\end{enumerate}

The difference from weak induction: you get to assume $P$ holds for \emph{all} values up to $k$, not just $k$ itself. Use strong induction when proving $P(k+1)$ requires jumping back further than just one step.
\end{strategy}

\begin{example}
Prove: Every integer $n \geq 2$ can be written as a product of primes.

\emph{Proof.} By strong induction on $n$.

\textbf{Base case} ($n = 2$): 2 is prime, so it's a ``product of primes'' (a product with one factor). \checkmark

\textbf{Inductive step:} Assume every integer from 2 to $k$ can be written as a product of primes. We need to show $k + 1$ can too.

\emph{Case 1: $k+1$ is prime.} Then $k+1$ is trivially a product of primes. \checkmark

\emph{Case 2: $k+1$ is composite.} Then $k+1 = ab$ where $2 \leq a, b < k+1$. (Both factors are smaller than $k+1$ but at least 2.)

Since $a \leq k$ and $b \leq k$, the inductive hypothesis applies to both. So $a$ is a product of primes and $b$ is a product of primes. Therefore $k+1 = ab$ is also a product of primes. \checkmark

By strong induction, every $n \geq 2$ is a product of primes. \qed
\end{example}

(This is half of the Fundamental Theorem of Arithmetic. The other half---that the factorization is unique---requires more work.)

\begin{strategy}[Structural induction]
To prove a property $P$ holds for all elements of a recursively-defined set $S$:
\begin{enumerate}
  \item \textbf{Base case(s):} Prove $P$ for each base element of $S$.
  \item \textbf{Inductive step(s):} For each recursive rule, assume $P$ holds for the inputs and prove $P$ for the output.
\end{enumerate}

This is induction generalized beyond numbers. If you've defined a data structure recursively (trees, lists, formulas, etc.), structural induction is the natural way to prove things about it.
\end{strategy}

\begin{example}
Define binary trees recursively:
\begin{itemize}
  \item \textbf{Base:} A single node $\bullet$ is a binary tree.
  \item \textbf{Recursive:} If $T_1$ and $T_2$ are binary trees, then the tree with a root node connected to $T_1$ (left) and $T_2$ (right) is a binary tree.
\end{itemize}

Prove: Every binary tree has one more leaf than internal node.

\emph{Proof.} By structural induction. Let $P(T)$ be the statement ``$T$ has one more leaf than internal node.''

\textbf{Base case:} A single node $\bullet$ has 1 leaf (itself) and 0 internal nodes. Since $1 = 0 + 1$, we have one more leaf than internal node. \checkmark

\textbf{Inductive step:} Suppose $T_1$ has $\ell_1$ leaves and $i_1$ internal nodes with $\ell_1 = i_1 + 1$. Similarly, $T_2$ has $\ell_2 = i_2 + 1$.

Form tree $T$ by connecting a new root to $T_1$ and $T_2$. In $T$:
\begin{itemize}
  \item The leaves are exactly the leaves of $T_1$ and $T_2$ (the new root isn't a leaf).
  \item The internal nodes are the internal nodes of $T_1$ and $T_2$, plus the new root.
\end{itemize}

So:
\begin{align*}
\text{leaves}(T) &= \ell_1 + \ell_2 \\
\text{internal}(T) &= i_1 + i_2 + 1
\end{align*}

Check: $\ell_1 + \ell_2 = (i_1 + 1) + (i_2 + 1) = (i_1 + i_2 + 1) + 1 = \text{internal}(T) + 1$. \checkmark

By structural induction, every binary tree has one more leaf than internal node. \qed
\end{example}

\subsection*{Common Inference Rules}

These are the ``legal moves'' in formal reasoning. You'll use them implicitly in most proofs.

\begin{center}
\begin{tabular}{lll}
\textbf{Name} & \textbf{Rule} & \textbf{In English} \\
\hline
Modus ponens & $P, \; P \to Q \;\vdash\; Q$ & ``If $P$ and `$P$ implies $Q$', then $Q$'' \\
Modus tollens & $\neg Q, \; P \to Q \;\vdash\; \neg P$ & ``If not $Q$ and `$P$ implies $Q$', then not $P$'' \\
Hypothetical syllogism & $P \to Q, \; Q \to R \;\vdash\; P \to R$ & Chain implications together \\
Disjunctive syllogism & $P \lor Q, \; \neg P \;\vdash\; Q$ & ``It's $P$ or $Q$, and not $P$, so $Q$'' \\
Conjunction & $P, \; Q \;\vdash\; P \land Q$ & Combine two facts \\
Simplification & $P \land Q \;\vdash\; P$ & Extract one part of an AND \\
Addition & $P \;\vdash\; P \lor Q$ & Weaken to a disjunction \\
Resolution & $P \lor Q, \; \neg P \lor R \;\vdash\; Q \lor R$ & The workhorse of automated theorem provers \\
\end{tabular}
\end{center}

Modus ponens and modus tollens are the two you'll use most. Modus ponens is ``forward reasoning'' (from premise to conclusion). Modus tollens is ``backward reasoning'' (from failure of conclusion to failure of premise).

\subsection*{Existential and Universal Proofs}

\begin{strategy}[Proving $\exists x.\, P(x)$]
Find a specific witness $c$ and show $P(c)$ holds.

This is often the easiest type of proof: just produce an example.

\textbf{Example.} Prove: There exists an integer $n$ such that $n^2 = n$.

\emph{Proof.} Take $n = 1$. Then $1^2 = 1$. \qed

(Or take $n = 0$. There are often multiple witnesses.)
\end{strategy}

\begin{strategy}[Proving $\forall x.\, P(x)$]
Let $x$ be an \emph{arbitrary} element of the domain and prove $P(x)$ without assuming anything special about $x$.

The word ``arbitrary'' is key. You can't say ``let $x = 5$'' because then you've only proved $P(5)$, not $P(x)$ for all $x$.

\textbf{Example.} Prove: For all integers $n$, if $n$ is odd, then $n^2$ is odd.

\emph{Proof.} Let $n$ be an arbitrary odd integer. Then $n = 2k + 1$ for some integer $k$.

So $n^2 = (2k+1)^2 = 4k^2 + 4k + 1 = 2(2k^2 + 2k) + 1$.

This has the form $2m + 1$ where $m = 2k^2 + 2k$ is an integer, so $n^2$ is odd. \qed
\end{strategy}

\begin{strategy}[Disproving $\forall x.\, P(x)$]
Find a counterexample: a specific $c$ where $P(c)$ is false.

Remember: the negation of ``for all $x$, $P(x)$'' is ``there exists an $x$ where $P(x)$ fails.'' One counterexample is enough.

\textbf{Example.} Disprove: For all primes $p$, $2^p - 1$ is prime.

\emph{Counterexample.} Let $p = 11$, which is prime. Then $2^{11} - 1 = 2047 = 23 \times 89$. Since 2047 is composite, the claim is false. \qed

(Numbers of the form $2^p - 1$ are called Mersenne numbers. When they're prime, they're Mersenne primes. Finding large Mersenne primes is an active area of number theory.)
\end{strategy}

\subsection*{Common Pitfalls}

Here's where students most often go wrong. Read these carefully---you'll probably make at least one of these mistakes this quarter, and recognizing it quickly will save you grief.

\begin{commonmistake}
\textbf{Assuming what you're trying to prove.} In a direct proof of $P \to Q$, you assume $P$, not $Q$. If you find yourself writing ``Assume $Q$...'' in a direct proof, you've gone circular. The whole point is to \emph{derive} $Q$ from other things.
\end{commonmistake}

\begin{commonmistake}
\textbf{Confusing implication with its converse.} $P \to Q$ is NOT the same as $Q \to P$. Proving ``if $n$ is even, then $n^2$ is even'' does \emph{not} prove ``if $n^2$ is even, then $n$ is even.'' (The latter happens to be true, but it requires a separate proof.)
\end{commonmistake}

\begin{commonmistake}
\textbf{Induction: not using the hypothesis.} In the inductive step, you \emph{must} use the assumption that $P(k)$ holds. If your proof of $P(k+1)$ doesn't mention $P(k)$ anywhere, either:
\begin{itemize}
  \item Your proof has a gap (most likely), or
  \item You didn't actually need induction (rare, but possible)
\end{itemize}
\end{commonmistake}

\begin{commonmistake}
\textbf{Induction: wrong base case.} If your claim is ``for all $n \geq 5$, $P(n)$,'' your base case must be $n = 5$, not $n = 0$ or $n = 1$. The base case is where your chain of dominoes starts.
\end{commonmistake}

\begin{commonmistake}
\textbf{Proof by example.} Checking that $P(1), P(2), P(3)$ are true does \emph{not} prove $\forall n.\, P(n)$. You need either a general argument or proper induction. ``It worked for the first few cases'' is not a proof.
\end{commonmistake}

\begin{commonmistake}
\textbf{Existential overgeneralization.} From ``there exists an $x$ with property $P$,'' you cannot conclude that every $x$ has property $P$. Just because some integer is even doesn't mean all integers are even.
\end{commonmistake}

\subsection*{Practice}

\begin{enumerate}
  \item Prove by induction: $\displaystyle\sum_{i=1}^{n} i^2 = \frac{n(n+1)(2n+1)}{6}$.

  \item Prove by induction: $n! > 2^n$ for all $n \geq 4$. (What's the base case?)

  \item Prove by strong induction: Every amount of postage $\geq 12$ cents can be made using 4-cent and 5-cent stamps. (Hint: you'll need multiple base cases.)

  \item Prove by contrapositive: If $n^2$ is divisible by 3, then $n$ is divisible by 3.

  \item Prove by contradiction: There are infinitely many primes. (This is Euclid's proof, one of the most beautiful in mathematics.)

  \item Prove or disprove: For all integers $a, b, c$, if $a \mid bc$, then $a \mid b$ or $a \mid c$. (The notation $a \mid b$ means ``$a$ divides $b$.'')

  \item Prove by structural induction: For any arithmetic expression built from integers using $+$ and $\times$, the result is an integer. (First, define the set of arithmetic expressions recursively.)

  \item Negate the following statement, then determine which version (original or negation) is true:

  ``For every $\epsilon > 0$, there exists $\delta > 0$ such that for all $x$, if $|x| < \delta$ then $|f(x)| < \epsilon$.''

  (This is related to the definition of a limit at 0.)

  \item Find the error in this ``proof'':

  \emph{Claim: All horses are the same color.}

  \emph{``Proof'' by induction on the number of horses:}

  Base case: One horse is trivially the same color as itself. \checkmark

  Inductive step: Assume any group of $k$ horses are the same color. Given $k+1$ horses, remove one; the remaining $k$ are the same color (by the inductive hypothesis). Put that horse back and remove a \emph{different} one; those $k$ horses are also the same color. Since the two groups overlap, all $k+1$ horses are the same color. \qed

  (Where exactly does this argument fail?)

  \item Prove: For all sets $A$ and $B$, if $A \subseteq B$, then $\mathcal{P}(A) \subseteq \mathcal{P}(B)$.
\end{enumerate}
