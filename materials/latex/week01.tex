\section{Week 1: Set Theory}
\subsection*{Reading}
Epp \S 6.1--6.4.

\subsection*{Learning objectives}
\begin{itemize}
  \item Translate between roster and set-builder notation.
  \item Prove set identities with the element method.
  \item Apply standard set laws (commutative, associative, distributive).
  \item Use power sets and partitions correctly.
\end{itemize}

\subsection*{Key definitions and facts}
\begin{definition}[Subset and equality]
For sets $A,B$, $A \subseteq B$ means every element of $A$ is in $B$.
$A=B$ means $A \subseteq B$ and $B \subseteq A$.
\end{definition}

\begin{definition}[Power set and partition]
The power set $\Pow(A)$ is the set of all subsets of $A$.
A partition of $A$ is a collection of nonempty, pairwise disjoint subsets whose union is $A$.
\end{definition}

\begin{theorem}[De Morgan's laws]
$(A \cup B)^c = A^c \cap B^c$ and $(A \cap B)^c = A^c \cup B^c$.
\end{theorem}

\subsection*{Worked example}
\begin{example}
Prove $A \cap (B \cup C) = (A \cap B) \cup (A \cap C)$.
\emph{Sketch.} For any $x$,
$x \in A \cap (B \cup C)$ iff $x \in A$ and $(x \in B$ or $x \in C)$,
iff $(x \in A \cap B)$ or $(x \in A \cap C)$.
\end{example}

\subsection*{Practice}
\begin{enumerate}
  \item Convert $\{x \in \Z : x^2 < 10\}$ into roster notation.
  \item Prove $A \setminus B = A \cap B^c$ using the element method.
  \item List $\Pow(\{a,b,c\})$ and verify its size.
  \item Give a counterexample showing $A \cap (B \setminus C) = (A \cap B) \setminus C$ can fail.
\end{enumerate}
