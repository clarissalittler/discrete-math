\section{Week 6: Expected Value and Introduction to Graphs}
\subsection*{Reading}
Epp \S 9.8; 10.1; 10.2.

\subsection*{Learning objectives}
\begin{itemize}
  \item Define and compute expected value for discrete random variables.
  \item Apply linearity of expectation to simplify calculations.
  \item Use indicator random variables for counting.
  \item Define graphs, vertices, edges, and basic terminology.
  \item Apply the handshake theorem to relate degrees and edges.
  \item Distinguish simple graphs, multigraphs, and digraphs.
\end{itemize}

\subsection*{Part I: Expected Value}

\begin{definition}[Probability axioms (Kolmogorov)]
A \textbf{probability measure} on a sample space $S$ is a function $P$ assigning to each event $A \subseteq S$ a number $P(A)$ satisfying:
\begin{enumerate}
  \item \textbf{Non-negativity:} $P(A) \geq 0$ for all events $A$.
  \item \textbf{Normalization:} $P(S) = 1$.
  \item \textbf{Additivity:} If $A_1, A_2, \ldots$ are pairwise disjoint events, then
  \[
  P\left(\bigcup_{i=1}^{\infty} A_i\right) = \sum_{i=1}^{\infty} P(A_i)
  \]
\end{enumerate}
\end{definition}

\begin{proposition}[Consequences of the axioms]
From the three axioms, we can derive:
\begin{itemize}
  \item $P(\emptyset) = 0$
  \item $P(A^c) = 1 - P(A)$
  \item If $A \subseteq B$, then $P(A) \leq P(B)$
  \item $P(A \cup B) = P(A) + P(B) - P(A \cap B)$
  \item $0 \leq P(A) \leq 1$ for all events $A$
\end{itemize}
\end{proposition}

\begin{definition}[Random variable]
A \textbf{random variable} $X$ on a sample space $S$ is a function $X: S \to \R$ that assigns a real number to each outcome. For discrete random variables, the possible values form a finite or countably infinite set.
\end{definition}

\begin{definition}[Expected value]
The \textbf{expected value} (or \textbf{expectation} or \textbf{mean}) of a discrete random variable $X$ with possible values $x_1, x_2, \ldots$ and probabilities $p_i = P(X = x_i)$ is:
\[
E[X] = \sum_i x_i \cdot P(X = x_i) = \sum_i x_i \cdot p_i
\]
provided the sum converges absolutely.
\end{definition}

\begin{theorem}[Linearity of expectation]
For any random variables $X$ and $Y$ (even if dependent) and constants $a, b \in \R$:
\[
E[aX + bY] = aE[X] + bE[Y]
\]
More generally, for any $X_1, \ldots, X_n$:
\[
E\left[\sum_{i=1}^n X_i\right] = \sum_{i=1}^n E[X_i]
\]
\end{theorem}

\begin{keyresult}
Linearity of expectation is extremely powerful because it works \emph{regardless of whether the random variables are independent}. This makes many expected value calculations surprisingly simple.
\end{keyresult}

\begin{definition}[Indicator random variable]
An \textbf{indicator random variable} $I_A$ for event $A$ is:
\[
I_A = \begin{cases} 1 & \text{if } A \text{ occurs} \\ 0 & \text{otherwise} \end{cases}
\]
Note that $E[I_A] = P(A)$.
\end{definition}

\begin{theorem}[Counting with indicators]
If $X$ counts the number of events $A_1, \ldots, A_n$ that occur, then:
\[
X = I_{A_1} + I_{A_2} + \cdots + I_{A_n}
\]
and by linearity:
\[
E[X] = P(A_1) + P(A_2) + \cdots + P(A_n)
\]
\end{theorem}

\begin{definition}[Common distributions]
\begin{itemize}
  \item \textbf{Bernoulli($p$):} $X = 1$ with probability $p$, $X = 0$ with probability $1-p$. $E[X] = p$.

  \item \textbf{Binomial($n, p$):} Number of successes in $n$ independent trials, each with success probability $p$. $E[X] = np$.

  \item \textbf{Geometric($p$):} Number of trials until first success. $E[X] = 1/p$.

  \item \textbf{Uniform on $\{1, \ldots, n\}$:} Each value equally likely. $E[X] = (n+1)/2$.
\end{itemize}
\end{definition}

\begin{definition}[Variance and standard deviation]
The \textbf{variance} of $X$ is:
\[
\text{Var}(X) = E[(X - E[X])^2] = E[X^2] - (E[X])^2
\]
The \textbf{standard deviation} is $\sigma = \sqrt{\text{Var}(X)}$.
\end{definition}

\subsection*{Part II: Introduction to Graphs}

\begin{definition}[Graph]
A \textbf{graph} $G = (V, E)$ consists of:
\begin{itemize}
  \item $V$: a finite nonempty set of \textbf{vertices} (or nodes)
  \item $E$: a set of \textbf{edges}, each connecting two vertices
\end{itemize}
\end{definition}

\begin{definition}[Types of graphs]
\begin{itemize}
  \item \textbf{Simple graph:} No loops (edges from a vertex to itself) and no multiple edges between the same pair of vertices.
  \item \textbf{Multigraph:} Allows multiple edges between the same pair of vertices.
  \item \textbf{Pseudograph:} Allows loops and multiple edges.
  \item \textbf{Directed graph (digraph):} Edges have direction, going from one vertex to another.
\end{itemize}
\end{definition}

\begin{definition}[Basic terminology]
Let $G = (V, E)$ be a graph.
\begin{itemize}
  \item Two vertices are \textbf{adjacent} if an edge connects them.
  \item An edge is \textbf{incident} to its endpoints.
  \item The \textbf{degree} $\deg(v)$ of vertex $v$ is the number of edges incident to $v$ (loops count twice).
  \item A vertex with degree 0 is \textbf{isolated}.
  \item A vertex with degree 1 is a \textbf{leaf} (or pendant vertex).
  \item The \textbf{neighborhood} $N(v)$ is the set of vertices adjacent to $v$.
\end{itemize}
\end{definition}

\begin{theorem}[Handshake theorem]
In any graph $G = (V, E)$:
\[
\sum_{v \in V} \deg(v) = 2|E|
\]
\emph{Proof idea:} Each edge contributes exactly 2 to the sum of degrees (1 to each endpoint). \qed
\end{theorem}

\begin{corollary}
In any graph, the number of vertices with odd degree is even.
\end{corollary}

\begin{definition}[Special graphs]
\begin{itemize}
  \item \textbf{Complete graph $K_n$:} Simple graph on $n$ vertices with all possible edges. Has $\binom{n}{2} = \frac{n(n-1)}{2}$ edges.

  \item \textbf{Cycle $C_n$:} Graph on $n$ vertices forming a single cycle. Has $n$ edges.

  \item \textbf{Path $P_n$:} Graph on $n$ vertices forming a single path. Has $n-1$ edges.

  \item \textbf{Complete bipartite graph $K_{m,n}$:} Vertices partitioned into sets of sizes $m$ and $n$; every vertex in one set is adjacent to every vertex in the other. Has $mn$ edges.

  \item \textbf{$n$-cube $Q_n$:} Vertices are $n$-bit strings; edges connect strings differing in exactly one bit. Has $2^n$ vertices and $n \cdot 2^{n-1}$ edges.
\end{itemize}
\end{definition}

\begin{definition}[Degree sequence]
The \textbf{degree sequence} of a graph is the list of vertex degrees in non-increasing order. For example, $K_4$ has degree sequence $(3, 3, 3, 3)$.
\end{definition}

\begin{theorem}[Degree sequence realizability]
A sequence of non-negative integers $(d_1, d_2, \ldots, d_n)$ with $d_1 \geq d_2 \geq \cdots \geq d_n$ is the degree sequence of a simple graph if and only if:
\begin{enumerate}
  \item The sum $\sum d_i$ is even.
  \item The sequence satisfies the Erdős--Gallai conditions (or can be checked using the Havel--Hakimi algorithm).
\end{enumerate}
\end{theorem}

\begin{definition}[Subgraph]
$H = (V', E')$ is a \textbf{subgraph} of $G = (V, E)$ if $V' \subseteq V$ and $E' \subseteq E$.

$H$ is an \textbf{induced subgraph} if $E'$ contains all edges of $G$ whose endpoints are both in $V'$.
\end{definition}

\begin{definition}[Graph complement]
The \textbf{complement} $\overline{G}$ of a simple graph $G = (V, E)$ has the same vertices as $G$, and two vertices are adjacent in $\overline{G}$ iff they are not adjacent in $G$.
\end{definition}

\subsection*{Worked examples}

\begin{example}
A fair die is rolled. Let $X$ be the outcome. Compute $E[X]$.

\emph{Solution.} Each outcome $1, 2, 3, 4, 5, 6$ has probability $\frac{1}{6}$.
\[
E[X] = 1 \cdot \frac{1}{6} + 2 \cdot \frac{1}{6} + 3 \cdot \frac{1}{6} + 4 \cdot \frac{1}{6} + 5 \cdot \frac{1}{6} + 6 \cdot \frac{1}{6} = \frac{21}{6} = 3.5
\]
\end{example}

\begin{example}
A coin is flipped 10 times. What is the expected number of heads?

\emph{Solution.} Let $X_i = 1$ if flip $i$ is heads, 0 otherwise. Then $X = X_1 + \cdots + X_{10}$ counts heads.

By linearity: $E[X] = E[X_1] + \cdots + E[X_{10}] = 10 \cdot \frac{1}{2} = 5$.
\end{example}

\begin{example}
In a random permutation of $n$ elements, what is the expected number of fixed points (elements in their original position)?

\emph{Solution.} Let $X_i = 1$ if element $i$ is in position $i$. Then $X = \sum_{i=1}^n X_i$ counts fixed points.

$P(\text{element } i \text{ is in position } i) = \frac{1}{n}$ (any of the $n!$ permutations, the element has $\frac{(n-1)!}{n!} = \frac{1}{n}$ chance of being fixed).

By linearity: $E[X] = n \cdot \frac{1}{n} = 1$.

Remarkably, the expected number of fixed points is exactly 1, regardless of $n$!
\end{example}

\begin{example}
What is the expected number of times we must roll a die to get a 6?

\emph{Solution.} This is a geometric random variable with success probability $p = \frac{1}{6}$.

$E[X] = \frac{1}{p} = 6$.
\end{example}

\begin{example}
Verify the handshake theorem for $K_4$.

\emph{Solution.} $K_4$ has 4 vertices, each with degree 3 (connected to all others).
\begin{itemize}
  \item Sum of degrees: $3 + 3 + 3 + 3 = 12$
  \item Number of edges: $\binom{4}{2} = 6$
  \item Check: $2 \times 6 = 12$ \checkmark
\end{itemize}
\end{example}

\begin{example}
Is there a simple graph with degree sequence $(3, 3, 2, 2, 2)$?

\emph{Solution.} Sum of degrees: $3 + 3 + 2 + 2 + 2 = 12$, which is even. \checkmark

Using Havel--Hakimi: Sort: $(3, 3, 2, 2, 2)$. Remove 3 and subtract 1 from next 3 degrees: $(2, 1, 1, 2)$. Sort: $(2, 2, 1, 1)$. Remove 2: $(1, 0, 1)$. Sort: $(1, 1, 0)$. Remove 1: $(0, 0)$. This is realizable (empty graph).

Yes, such a graph exists.
\end{example}

\begin{example}
How many edges does the $n$-cube $Q_n$ have?

\emph{Solution.} $Q_n$ has $2^n$ vertices, each an $n$-bit string. Each vertex has degree $n$ (can flip any of $n$ bits).

Sum of degrees: $n \cdot 2^n$.

By handshake theorem: $|E| = \frac{n \cdot 2^n}{2} = n \cdot 2^{n-1}$.
\end{example}

\begin{example}
Prove: The sum of degrees in a tree on $n$ vertices is $2(n-1)$.

\emph{Solution.} A tree on $n$ vertices has exactly $n-1$ edges (this is a standard fact---see Week 8). By the handshake theorem:
\[
\sum_{v \in V} \deg(v) = 2|E| = 2(n-1)
\]
\end{example}

\begin{example}
Show that every simple graph on $n \geq 2$ vertices has at least two vertices of the same degree.

\emph{Solution.} Degrees in a simple graph range from 0 to $n-1$. That's $n$ possible values. But if some vertex has degree 0 (isolated), no vertex can have degree $n-1$ (connected to all). So at most $n-1$ distinct degrees are possible among $n$ vertices. By pigeonhole, two must share a degree.
\end{example}

\begin{commonmistake}
\textbf{Forgetting linearity works for dependent variables.} The formula $E[X + Y] = E[X] + E[Y]$ does NOT require $X$ and $Y$ to be independent. Many students add independence as an assumption when it's unnecessary.
\end{commonmistake}

\begin{commonmistake}
\textbf{Confusing $E[X \cdot Y]$ with $E[X] \cdot E[Y]$.} These are equal only when $X$ and $Y$ are independent. In general, $E[XY] = E[X]E[Y] + \text{Cov}(X,Y)$.
\end{commonmistake}

\subsection*{Practice}
\begin{enumerate}
  \item A coin is flipped 10 times. What is the expected number of heads?

  \item Show that the sum of degrees in a tree on $n$ vertices is $2(n-1)$.

  \item Find $E[X]$ for a geometric random variable with success probability $p$.

  \item Decide whether a graph with degree sequence $(3, 3, 2, 2, 2)$ is possible.

  \item In a random permutation of $\{1, 2, \ldots, n\}$, what is the expected number of elements greater than all previous elements?

  \item How many edges does $K_{4,5}$ have? What are the degrees of the vertices?

  \item Prove that the complement of $K_n$ is an empty graph (no edges).

  \item A bag contains 5 red and 3 blue marbles. Two are drawn without replacement. What is the expected number of red marbles drawn?

  \item Show that the number of edges in a simple graph on $n$ vertices is at most $\binom{n}{2}$.

  \item Using the handshake theorem, prove: If $G$ is a graph where every vertex has degree at least $k$, then $|E| \geq \frac{k|V|}{2}$.

  \item Prove that every graph has an even number of vertices with odd degree.

  \item In a room of 100 people, everyone shakes hands with exactly 3 other people. Is this possible?
\end{enumerate}
