\section{Week 6: Expected Value and Intro Graphs}
\subsection*{Reading}
Epp \S 9.8; 1.4; 4.9.

\subsection*{Learning objectives}
\begin{itemize}
  \item Apply the axioms of probability to compute expectations.
  \item Use linearity of expectation to simplify calculations.
  \item Describe graphs, vertices, edges, and degree.
  \item Use the handshake theorem to relate degrees and edges.
\end{itemize}

\subsection*{Key definitions and facts}
\begin{definition}[Expected value]
For a discrete random variable $X$ with values $x_i$ and probabilities $p_i$,
$E[X]=\sum_i x_i p_i$.
\end{definition}

\begin{theorem}[Linearity of expectation]
For random variables $X,Y$, $E[X+Y]=E[X]+E[Y]$.
\end{theorem}

\begin{theorem}[Handshake theorem]
In any finite graph, the sum of all vertex degrees equals $2|E|$.
\end{theorem}

\subsection*{Worked example}
\begin{example}
A fair die is rolled. Compute $E[X]$.
\emph{Sketch.} $E[X]=(1+2+3+4+5+6)/6=3.5$.
\end{example}

\subsection*{Practice}
\begin{enumerate}
  \item A coin is flipped 10 times. What is the expected number of heads?
  \item Show that the sum of degrees in a tree on $n$ vertices is $2(n-1)$.
  \item Find $E[X]$ for a geometric random variable with success probability $p$ (give the formula).
  \item Decide whether a graph with degree sequence $3,3,2,2,2$ is possible.
\end{enumerate}
