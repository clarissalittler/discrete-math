\section{Week 7: Graph Theory I --- Paths and Connectivity}
\subsection*{Reading}
Epp \S 10.1--10.3.

\subsection*{Learning objectives}
\begin{itemize}
  \item Distinguish walks, trails, paths, and circuits.
  \item Apply Euler's criteria for trails and circuits.
  \item Determine graph connectivity and connected components.
  \item Represent graphs with adjacency matrices and adjacency lists.
  \item Determine whether two graphs are isomorphic.
\end{itemize}

\subsection*{Key definitions and facts}

\begin{definition}[Walk]
A \textbf{walk} in a graph $G$ from vertex $v_0$ to vertex $v_n$ is a sequence:
\[
v_0, e_1, v_1, e_2, v_2, \ldots, e_n, v_n
\]
where each $e_i$ is an edge connecting $v_{i-1}$ and $v_i$. The \textbf{length} of the walk is $n$ (the number of edges).
\end{definition}

\begin{definition}[Types of walks]
\begin{itemize}
  \item A \textbf{trail} is a walk with no repeated edges.
  \item A \textbf{path} is a walk with no repeated vertices (hence no repeated edges).
  \item A \textbf{closed walk} is a walk where $v_0 = v_n$.
  \item A \textbf{circuit} (or closed trail) is a closed walk with no repeated edges.
  \item A \textbf{cycle} (or simple circuit) is a circuit with no repeated vertices except $v_0 = v_n$.
\end{itemize}
\end{definition}

\begin{proposition}[Path existence]
If there is a walk from $u$ to $v$ in a graph, then there is a path from $u$ to $v$.
\end{proposition}

\begin{definition}[Connectivity]
\begin{itemize}
  \item A graph is \textbf{connected} if there is a path between every pair of vertices.
  \item A \textbf{connected component} of a graph is a maximal connected subgraph.
  \item A \textbf{cut vertex} (or articulation point) is a vertex whose removal disconnects the graph.
  \item A \textbf{bridge} is an edge whose removal disconnects the graph.
\end{itemize}
\end{definition}

\begin{definition}[Euler trail and circuit]
An \textbf{Euler trail} is a trail that uses every edge of the graph exactly once.
An \textbf{Euler circuit} is a circuit that uses every edge exactly once (starts and ends at the same vertex).
\end{definition}

\begin{theorem}[Euler's theorem]
Let $G$ be a connected graph.
\begin{enumerate}
  \item $G$ has an \textbf{Euler circuit} if and only if every vertex has even degree.
  \item $G$ has an \textbf{Euler trail} (but no Euler circuit) if and only if exactly two vertices have odd degree. In this case, the trail must start and end at the odd-degree vertices.
\end{enumerate}
\end{theorem}

\begin{keyresult}
To check if a connected graph has an Euler circuit or trail:
\begin{enumerate}
  \item Count vertices of odd degree.
  \item 0 odd-degree vertices $\Rightarrow$ Euler circuit exists.
  \item 2 odd-degree vertices $\Rightarrow$ Euler trail exists (but no circuit).
  \item $>2$ odd-degree vertices $\Rightarrow$ no Euler trail.
\end{enumerate}
\end{keyresult}

\begin{definition}[Hamiltonian path and cycle]
A \textbf{Hamiltonian path} visits every vertex exactly once.
A \textbf{Hamiltonian cycle} is a cycle that visits every vertex exactly once (except returning to start).
\end{definition}

\begin{remark}
Unlike Euler paths/circuits, there is no simple characterization for when Hamiltonian paths/cycles exist. Determining existence is NP-complete.
\end{remark}

\subsection*{Graph representations}

\begin{definition}[Adjacency matrix]
The \textbf{adjacency matrix} $A$ of a graph $G$ with $n$ vertices is an $n \times n$ matrix where:
\[
A_{ij} = \text{number of edges between vertex } i \text{ and vertex } j
\]
For a simple graph, $A_{ij} \in \{0, 1\}$. The matrix is symmetric for undirected graphs.
\end{definition}

\begin{proposition}[Properties of adjacency matrices]
For the adjacency matrix $A$ of a simple graph:
\begin{itemize}
  \item The sum of row $i$ (or column $i$) equals $\deg(v_i)$.
  \item The sum of all entries equals $2|E|$.
  \item The diagonal is all zeros (no loops).
  \item $(A^k)_{ij}$ counts the number of walks of length $k$ from $v_i$ to $v_j$.
\end{itemize}
\end{proposition}

\begin{definition}[Adjacency list]
An \textbf{adjacency list} representation stores, for each vertex, a list of its neighbors. This is more space-efficient for sparse graphs.
\end{definition}

\subsection*{Graph isomorphism}

\begin{definition}[Graph isomorphism]
Two graphs $G_1 = (V_1, E_1)$ and $G_2 = (V_2, E_2)$ are \textbf{isomorphic}, written $G_1 \cong G_2$, if there exists a bijection $f: V_1 \to V_2$ such that:
\[
\{u, v\} \in E_1 \iff \{f(u), f(v)\} \in E_2
\]
The function $f$ is called an \textbf{isomorphism}.
\end{definition}

\begin{theorem}[Isomorphism invariants]
If $G_1 \cong G_2$, then:
\begin{enumerate}
  \item $|V_1| = |V_2|$
  \item $|E_1| = |E_2|$
  \item They have the same degree sequence
  \item They have the same number of cycles of each length
  \item They have the same number of connected components
  \item Corresponding subgraphs are isomorphic
\end{enumerate}
These are \emph{necessary} but not \emph{sufficient} conditions for isomorphism.
\end{theorem}

\begin{strategy}
To show two graphs are NOT isomorphic, find an invariant they don't share. To show they ARE isomorphic, construct an explicit bijection and verify edge preservation.
\end{strategy}

\begin{definition}[Automorphism]
An \textbf{automorphism} of a graph $G$ is an isomorphism from $G$ to itself. The set of all automorphisms forms a group under composition.
\end{definition}

\subsection*{Distance and diameter}

\begin{definition}[Distance]
The \textbf{distance} $d(u, v)$ between vertices $u$ and $v$ is the length of the shortest path between them. If no path exists, $d(u, v) = \infty$.
\end{definition}

\begin{definition}[Eccentricity, radius, diameter]
\begin{itemize}
  \item The \textbf{eccentricity} of a vertex $v$ is the maximum distance from $v$ to any other vertex: $\max_{u} d(v, u)$.
  \item The \textbf{diameter} of a connected graph is the maximum eccentricity.
  \item The \textbf{radius} is the minimum eccentricity.
  \item A \textbf{center} is a vertex with minimum eccentricity.
\end{itemize}
\end{definition}

\subsection*{Graph coloring}

\begin{definition}[Vertex coloring]
A \textbf{(proper) vertex coloring} of a graph $G$ is an assignment of colors to vertices such that no two adjacent vertices share the same color. A \textbf{$k$-coloring} uses at most $k$ colors.
\end{definition}

\begin{definition}[Chromatic number]
The \textbf{chromatic number} $\chi(G)$ is the minimum number of colors needed to properly color $G$.
\end{definition}

\begin{theorem}[Chromatic number bounds]
For any graph $G$:
\begin{enumerate}
  \item $\chi(G) \geq \omega(G)$, where $\omega(G)$ is the size of the largest clique (complete subgraph).
  \item $\chi(G) \leq \Delta(G) + 1$, where $\Delta(G)$ is the maximum degree.
  \item If $G$ is connected and not a complete graph or odd cycle, then $\chi(G) \leq \Delta(G)$ (Brooks' theorem).
\end{enumerate}
\end{theorem}

\begin{theorem}[Chromatic numbers of special graphs]
\begin{itemize}
  \item $\chi(K_n) = n$ (complete graph needs $n$ colors)
  \item $\chi(C_n) = 2$ if $n$ is even; $\chi(C_n) = 3$ if $n$ is odd
  \item $\chi(K_{m,n}) = 2$ (bipartite graphs are 2-colorable)
  \item A tree with at least one edge has $\chi(T) = 2$
\end{itemize}
\end{theorem}

\begin{definition}[Bipartite graph]
A graph is \textbf{bipartite} if its vertices can be partitioned into two sets such that every edge connects a vertex in one set to a vertex in the other. Equivalently, $G$ is bipartite iff $\chi(G) \leq 2$.
\end{definition}

\begin{theorem}[Bipartite characterization]
A graph is bipartite if and only if it contains no odd-length cycle.
\end{theorem}

\begin{example}
Is the Petersen graph 3-colorable?

\emph{Solution.} The Petersen graph contains triangles (3-cycles), so $\chi \geq 3$. In fact, $\chi(\text{Petersen}) = 3$. You can verify by constructing a 3-coloring: color the outer 5-cycle with alternating colors (using 3 since it's odd), then color the inner 5-cycle consistently.
\end{example}

\subsection*{Planar graphs}

\begin{definition}[Planar graph]
A graph is \textbf{planar} if it can be drawn in the plane with no edges crossing (except at vertices). Such a drawing is called a \textbf{planar embedding}.
\end{definition}

\begin{definition}[Faces]
In a planar embedding, the plane is divided into \textbf{faces} (regions), including one unbounded \textbf{outer face}. The boundary of each face consists of edges and vertices.
\end{definition}

\begin{theorem}[Euler's formula for planar graphs]
For a connected planar graph with $V$ vertices, $E$ edges, and $F$ faces:
\[
V - E + F = 2
\]
\end{theorem}

\begin{example}
Verify Euler's formula for the tetrahedron graph $K_4$.

\emph{Solution.} $K_4$ has $V = 4$ vertices and $E = \binom{4}{2} = 6$ edges. Drawing it as a triangle with a point in the center gives $F = 4$ faces (3 inner triangles + 1 outer face).

Check: $4 - 6 + 4 = 2$. \checkmark
\end{example}

\begin{theorem}[Edge bound for planar graphs]
For a connected planar graph with $V \geq 3$ vertices:
\[
E \leq 3V - 6
\]
If the graph has no triangles (is triangle-free), then $E \leq 2V - 4$.
\end{theorem}

\begin{corollary}
$K_5$ and $K_{3,3}$ are not planar.
\end{corollary}

\begin{proof}
For $K_5$: $V = 5$, $E = 10$. But $3V - 6 = 9 < 10$. Violates the bound.

For $K_{3,3}$: $V = 6$, $E = 9$. Since $K_{3,3}$ is bipartite, it has no triangles, so we need $E \leq 2V - 4 = 8 < 9$. Violates the bound.
\end{proof}

\begin{theorem}[Kuratowski's theorem]
A graph is planar if and only if it contains no subgraph that is a subdivision of $K_5$ or $K_{3,3}$.

(A \textbf{subdivision} is obtained by inserting vertices of degree 2 into edges.)
\end{theorem}

\begin{theorem}[Four Color Theorem]
Every planar graph can be colored with at most 4 colors: $\chi(G) \leq 4$ for planar $G$.
\end{theorem}

\begin{remark}
The Four Color Theorem was proved in 1976 using computer assistance to check thousands of cases. Simpler proofs exist but none are ``hand-checkable.''
\end{remark}

\begin{example}
Show that the cube graph $Q_3$ is planar.

\emph{Solution.} $Q_3$ has $V = 8$ vertices and $E = 12$ edges. Check: $3V - 6 = 18 \geq 12$. \checkmark (This doesn't prove planarity, but it's consistent.)

To prove planarity, we draw $Q_3$ without crossings: draw the outer square as the front face, the inner square as the back face, and connect corresponding vertices.
\end{example}

\begin{strategy}
To show a graph is NOT planar:
\begin{enumerate}
  \item Show it violates $E \leq 3V - 6$, or
  \item Find a $K_5$ or $K_{3,3}$ subdivision.
\end{enumerate}
To show a graph IS planar:
\begin{enumerate}
  \item Draw it without crossings, or
  \item Prove $V$ and $E$ satisfy the bounds (necessary but not sufficient).
\end{enumerate}
\end{strategy}

\subsection*{Worked examples}

\begin{example}
Does a connected graph with degrees $(2, 2, 2, 4, 4)$ have an Euler circuit?

\emph{Solution.} All degrees are even (2, 2, 2, 4, 4), so yes, an Euler circuit exists by Euler's theorem.
\end{example}

\begin{example}
Does the graph $K_4$ (complete graph on 4 vertices) have an Euler circuit?

\emph{Solution.} In $K_4$, each vertex has degree 3 (odd). All 4 vertices have odd degree. Since we need 0 or 2 vertices of odd degree for an Euler trail/circuit, $K_4$ has neither.
\end{example}

\begin{example}
Does the graph $K_5$ have an Euler circuit?

\emph{Solution.} In $K_5$, each vertex has degree 4 (even). All vertices have even degree, so $K_5$ has an Euler circuit.
\end{example}

\begin{example}
Find the adjacency matrix for the cycle $C_4$ on vertices $\{1, 2, 3, 4\}$.

\emph{Solution.} The edges are $\{1,2\}, \{2,3\}, \{3,4\}, \{4,1\}$.
\[
A = \begin{pmatrix}
0 & 1 & 0 & 1 \\
1 & 0 & 1 & 0 \\
0 & 1 & 0 & 1 \\
1 & 0 & 1 & 0
\end{pmatrix}
\]
\end{example}

\begin{example}
Show that the sum of entries in an adjacency matrix of a simple graph equals $2|E|$.

\emph{Solution.} Each edge $\{u, v\}$ contributes 1 to entry $(u, v)$ and 1 to entry $(v, u)$, for a total of 2 per edge. Thus the sum equals $2|E|$.
\end{example}

\begin{example}
Determine whether these two graphs are isomorphic:

$G_1$: vertices $\{a, b, c, d\}$, edges $\{a,b\}, \{b,c\}, \{c,d\}, \{d,a\}$

$G_2$: vertices $\{1, 2, 3, 4\}$, edges $\{1,2\}, \{2,3\}, \{3,4\}, \{4,1\}$

\emph{Solution.} Both are cycles of length 4.
\begin{itemize}
  \item Same number of vertices: 4 \checkmark
  \item Same number of edges: 4 \checkmark
  \item Same degree sequence: $(2, 2, 2, 2)$ \checkmark
\end{itemize}

The bijection $f: a \mapsto 1, b \mapsto 2, c \mapsto 3, d \mapsto 4$ preserves edges:
$\{a,b\} \mapsto \{1,2\}$, $\{b,c\} \mapsto \{2,3\}$, $\{c,d\} \mapsto \{3,4\}$, $\{d,a\} \mapsto \{4,1\}$. All edges match, so $G_1 \cong G_2$.
\end{example}

\begin{example}
Prove that $C_5$ and $K_5$ are not isomorphic.

\emph{Solution.} $C_5$ has 5 edges (a cycle). $K_5$ has $\binom{5}{2} = 10$ edges. Since they have different numbers of edges, they are not isomorphic.
\end{example}

\begin{example}
Are two graphs with the same degree sequence necessarily isomorphic?

\emph{Solution.} No! Consider:
\begin{itemize}
  \item $G_1$: a 6-cycle $C_6$. Degree sequence: $(2, 2, 2, 2, 2, 2)$.
  \item $G_2$: two disjoint triangles $K_3 \sqcup K_3$. Degree sequence: $(2, 2, 2, 2, 2, 2)$.
\end{itemize}
Same degree sequence, but $G_1$ is connected and $G_2$ is not. Not isomorphic.
\end{example}

\begin{example}
Find the diameter of the complete graph $K_n$.

\emph{Solution.} Every pair of vertices is connected by an edge, so $d(u, v) = 1$ for all $u \neq v$. The diameter is 1.
\end{example}

\begin{example}
Find an Euler trail in a graph with vertices $\{A, B, C, D\}$ and edges
\[
\{A,B\}, \{A,C\}, \{B,C\}, \{B,D\}, \{C,D\}.
\]

\emph{Solution.} First, check degrees: $\deg(A) = 2$, $\deg(B) = 3$, $\deg(C) = 3$, $\deg(D) = 2$.

Odd-degree vertices: $B$ and $C$ (exactly 2). So an Euler trail exists, starting and ending at $B$ and $C$.

One Euler trail starting at $B$: $B \to A \to C \to B \to D \to C$.

Verify: Uses edges $\{B,A\}, \{A,C\}, \{C,B\}, \{B,D\}, \{D,C\}$ --- all 5 edges, each exactly once. \checkmark
\end{example}

\begin{example}
Compute $A^2$ for the path graph $P_3$ on vertices $\{1, 2, 3\}$ with edges $\{1,2\}$ and $\{2,3\}$. Interpret the result.

\emph{Solution.}
\[
A = \begin{pmatrix} 0 & 1 & 0 \\ 1 & 0 & 1 \\ 0 & 1 & 0 \end{pmatrix}
\]
\[
A^2 = \begin{pmatrix} 0 & 1 & 0 \\ 1 & 0 & 1 \\ 0 & 1 & 0 \end{pmatrix}
\begin{pmatrix} 0 & 1 & 0 \\ 1 & 0 & 1 \\ 0 & 1 & 0 \end{pmatrix}
= \begin{pmatrix} 1 & 0 & 1 \\ 0 & 2 & 0 \\ 1 & 0 & 1 \end{pmatrix}
\]

Interpretation: $(A^2)_{ij}$ is the number of walks of length 2 from $i$ to $j$.
\begin{itemize}
  \item $(A^2)_{11} = 1$: one walk $1 \to 2 \to 1$.
  \item $(A^2)_{13} = 1$: one walk $1 \to 2 \to 3$.
  \item $(A^2)_{22} = 2$: two walks $2 \to 1 \to 2$ and $2 \to 3 \to 2$.
\end{itemize}
\end{example}

\begin{commonmistake}
\textbf{Confusing Euler and Hamiltonian.}
\begin{itemize}
  \item Euler: visits every \emph{edge} exactly once.
  \item Hamiltonian: visits every \emph{vertex} exactly once.
\end{itemize}
Euler has a simple characterization (degree conditions). Hamiltonian does not.
\end{commonmistake}

\begin{commonmistake}
\textbf{Thinking matching invariants proves isomorphism.} Equal vertex count, edge count, and degree sequence are \emph{necessary} but not \emph{sufficient} for isomorphism. You must construct a bijection or find a distinguishing property.
\end{commonmistake}

\begin{goingdeeper}[Going Deeper: Graphs Generate Categories]
The categorical thread continues: graphs give rise to categories, and this perspective illuminates why adjacency matrices count paths.

\subsubsection*{The Free Category on a Graph}

Given a directed graph $G$, we can build a category $\mathbf{Path}(G)$:
\begin{itemize}
  \item \textbf{Objects:} Vertices of $G$
  \item \textbf{Morphisms from $u$ to $v$:} Directed paths from $u$ to $v$
  \item \textbf{Composition:} Concatenation of paths
  \item \textbf{Identity at $v$:} The empty path (length 0) at $v$
\end{itemize}

This is called the \emph{free category} on $G$---it's the category with ``just enough structure'' to capture the graph.

\textbf{Example.} For the graph $1 \to 2 \to 3$:
\begin{itemize}
  \item Morphisms $1 \to 3$: just the path $1 \to 2 \to 3$ (one morphism)
  \item Morphisms $2 \to 2$: just the empty path $\id_2$ (one morphism)
  \item Morphisms $3 \to 1$: none (no path backwards)
\end{itemize}

\textbf{Example.} For a cycle $1 \to 2 \to 3 \to 1$:
\begin{itemize}
  \item Morphisms $1 \to 1$: empty path, $1 \to 2 \to 3 \to 1$, twice around, thrice around, ...
  \item This is infinite! The cycle generates infinitely many paths.
\end{itemize}

\subsubsection*{Adjacency Matrices Count Morphisms}

Here's the key insight: $(A^k)_{ij}$ counts the number of \textbf{morphisms of length $k$} from $i$ to $j$ in $\mathbf{Path}(G)$.

Why? Let's see for $k = 2$:
\[
(A^2)_{ij} = \sum_m A_{im} \cdot A_{mj}
\]
Each term $A_{im} \cdot A_{mj}$ counts paths that go $i \to m \to j$ (one for each intermediate vertex $m$ with edges from $i$ and to $j$).

This is exactly the \textbf{composition} of morphisms in $\mathbf{Path}(G)$!

\subsubsection*{Composition as Matrix Multiplication}

The correspondence is:
\begin{center}
\begin{tabular}{cc}
\textbf{Category} & \textbf{Matrix} \\
\hline
Composition of paths & Matrix multiplication \\
Length-$k$ paths & $A^k$ \\
Identity (length-0 path) & $I$ (identity matrix)
\end{tabular}
\end{center}

\subsubsection*{Quotient Categories: Imposing Relations}

What if we want to declare two paths equal? For example, in a commutative square:
\[
\begin{tikzcd}
1 \arrow[r, "a"] \arrow[d, "c"'] & 2 \arrow[d, "b"] \\
3 \arrow[r, "d"'] & 4
\end{tikzcd}
\]
In $\mathbf{Path}(G)$, the paths $b \circ a$ and $d \circ c$ are different morphisms. But if we impose the relation $b \circ a = d \circ c$, we get a \emph{quotient category} where these paths are identified.

This is how commutative diagrams work: they specify which paths should be considered equal.

\subsubsection*{Exercises: Graphs and Paths}

\begin{enumerate}
  \item For the graph $1 \to 2 \to 3$, list all morphisms in $\mathbf{Path}(G)$ from each vertex to each vertex.

  \item For the graph with edges $1 \to 2$, $2 \to 3$, $3 \to 1$, how many morphisms of length 3 are there from 1 to 1? Verify using $A^3$.

  \item For a graph with edges $a \to b$, $b \to c$, $a \to c$, are there two different morphisms from $a$ to $c$? In the \emph{quotient} category where we impose $c \circ b \circ a^{-1} = \id$... wait, we can't do that without inverses. Just count: how many distinct paths from $a$ to $c$?

  \item Write the adjacency matrix for the 3-cycle. Compute $A^2$ and verify that $(A^2)_{11}$ equals the number of length-2 paths from 1 to 1.

  \item A graph has edges $1 \to 2$, $2 \to 1$ (a 2-cycle). How many morphisms of length 4 are there from 1 to 1? Compute using $A^4$.

  \item \textbf{Challenge:} If $G$ is a graph with no directed cycles, prove that $\mathbf{Path}(G)$ has finitely many morphisms between any two vertices. (Hint: What bounds the length of paths?)

  \item Draw a directed graph whose path category has exactly 3 morphisms from vertex $a$ to vertex $b$.

  \item In the path category, explain why composition is associative and why the empty path is the identity.
\end{enumerate}
\end{goingdeeper}

\subsection*{Practice}
\begin{enumerate}
  \item Give the adjacency matrix for the 4-cycle $C_4$.

  \item Determine whether the two graphs below are isomorphic (construct your own example).

  \item Find an Euler trail in a graph with exactly two odd-degree vertices.

  \item Show that the sum of entries in an adjacency matrix equals $2|E|$.

  \item Prove: If $G$ is a simple graph and $\overline{G}$ is its complement, then $G \cong \overline{G}$ implies $|V| \equiv 0$ or $1 \pmod 4$.

  \item Compute $A^2$ for $K_3$ and interpret the entries.

  \item Prove that every connected graph on $n$ vertices has at least $n-1$ edges.

  \item Find the diameter of the $n$-cube $Q_n$.

  \item Does $K_{3,3}$ (complete bipartite graph) have an Euler circuit? An Euler trail?

  \item Prove that a graph is bipartite if and only if it contains no odd-length cycles.

  \item How many automorphisms does the cycle $C_n$ have?

  \item Prove: If $G$ is connected and has exactly 2 vertices of odd degree, any Euler trail must start and end at those vertices.

  \item Find $\chi(C_7)$ and $\chi(C_8)$.

  \item Find the chromatic number of the wheel graph $W_5$ (a 5-cycle with a central vertex connected to all).

  \item Use Euler's formula to find the number of faces in a connected planar graph with 10 vertices and 15 edges.

  \item Prove that every planar graph has a vertex of degree at most 5.

  \item Is the Petersen graph planar? Prove your answer.

  \item A planar graph has 12 faces, and each face is bounded by exactly 3 edges. How many edges and vertices does it have?

  \item Give a 3-coloring of the graph $K_4$ minus one edge.

  \item Prove: If $G$ is planar with no cycles of length $\leq 4$, then $E \leq \frac{5}{3}(V - 2)$.
\end{enumerate}
