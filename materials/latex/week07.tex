\section{Week 7: Graph Theory I --- Paths and Connectivity}

\textit{Or: ``How do you get from here to there, and is there more than one way?''}

\subsection*{Reading}
Epp \S 10.1--10.3.\\
\textbf{Category theory companion:} Weeks 6--7 (\texttt{category\_theory\_companion.pdf}).

\subsection*{Why Paths and Connectivity?}

Last week we introduced graphs as a way to model ``things with connections.'' But having a graph is just the beginning. The interesting questions are about \emph{navigation}: Can you get from vertex $A$ to vertex $B$? How many ways? What's the shortest route?

These questions show up everywhere. Can a message travel through a network from sender to receiver? Can you drive from Portland to Boston? Can a chess knight visit every square? Can you trace a figure without lifting your pen?

This week introduces the vocabulary for talking about movement through graphs: walks, paths, circuits, and the beautiful theorem of Euler that tells us exactly when we can traverse every edge exactly once.

\subsection*{Learning objectives}
\begin{itemize}
  \item Distinguish walks, trails, paths, and circuits.
  \item Apply Euler's criteria for trails and circuits.
  \item Determine graph connectivity and connected components.
  \item Represent graphs with adjacency matrices and adjacency lists.
  \item Determine whether two graphs are isomorphic.
\end{itemize}

\subsection*{Key definitions and facts}

\subsubsection*{Ways of Walking Through a Graph}

\begin{definition}[Walk]
A \textbf{walk} from vertex $v_0$ to vertex $v_n$ is a sequence:
\[
v_0, e_1, v_1, e_2, v_2, \ldots, e_n, v_n
\]
where each $e_i$ connects $v_{i-1}$ to $v_i$. The \textbf{length} is $n$ (the number of edges traversed).
\end{definition}

A walk is the most general notion: you're just wandering through the graph, following edges. You can revisit vertices, cross the same edge multiple times, go in circles---anything goes as long as each step follows an actual edge.

But often we want more structure:

\begin{definition}[Types of walks]
\begin{itemize}
  \item A \textbf{trail} is a walk with no repeated edges. You can revisit vertices, but you can't cross the same bridge twice.
  \item A \textbf{path} is a walk with no repeated vertices. This is the ``clean'' version---no backtracking, no loops, just a direct route.
  \item A \textbf{closed walk} returns to its starting point ($v_0 = v_n$).
  \item A \textbf{circuit} is a closed trail---returns to start without repeating any edge.
  \item A \textbf{cycle} is a closed path---returns to start without repeating any vertex (except the start/end).
\end{itemize}
\end{definition}

The hierarchy goes: walk $\supseteq$ trail $\supseteq$ path, and closed walk $\supseteq$ circuit $\supseteq$ cycle. Each additional restriction makes the object ``nicer'' but harder to find.

\begin{proposition}[Path existence]
If there's a walk from $u$ to $v$, there's a path from $u$ to $v$.
\end{proposition}

Why? If your walk repeats a vertex, you went in a circle---just cut out the circle and you have a shorter walk. Keep cutting until no repeats remain. This ``shortcutting'' argument shows that walks and paths detect the same reachability information.

\subsubsection*{Connectivity}

\begin{definition}[Connectivity]
\begin{itemize}
  \item A graph is \textbf{connected} if there's a path between every pair of vertices. You can get anywhere from anywhere.
  \item A \textbf{connected component} is a maximal connected piece---add any more vertices and you'd break connectivity.
  \item A \textbf{cut vertex} (or articulation point) is a vertex whose removal disconnects the graph. It's a bottleneck.
  \item A \textbf{bridge} is an edge whose removal disconnects the graph.
\end{itemize}
\end{definition}

Every graph is a disjoint union of its connected components. A connected graph has exactly one component. The extreme case: a graph with no edges has $n$ components (each vertex is its own island).

\subsubsection*{Euler Trails and Circuits}

Here's a famous problem from 1736: the city of Königsberg had seven bridges connecting four landmasses. Can you walk through the city crossing each bridge exactly once?

Euler showed the answer is no, and in doing so invented graph theory.

\begin{definition}[Euler trail and circuit]
An \textbf{Euler trail} is a trail that uses every edge exactly once.
An \textbf{Euler circuit} is an Euler trail that starts and ends at the same vertex.
\end{definition}

The question is: when do these exist? The answer is surprisingly clean.

\begin{theorem}[Euler's theorem]
Let $G$ be a connected graph.
\begin{enumerate}
  \item $G$ has an Euler circuit if and only if every vertex has even degree.
  \item $G$ has an Euler trail (but no circuit) if and only if exactly two vertices have odd degree. The trail must start at one odd-degree vertex and end at the other.
\end{enumerate}
\end{theorem}

\emph{Why does this work?} Think about what happens when you walk through a vertex: you enter on one edge and leave on another. That uses up edges in pairs. If a vertex has odd degree, you can't pair up all its edges---one will be left over. That means an odd-degree vertex must be where you start or end your journey (using that unpaired edge as the first or last step).

An Euler circuit visits every edge and returns home, so every vertex gets entered and exited the same number of times---all degrees must be even. An Euler trail can have exactly two odd-degree vertices (start and end).

\begin{keyresult}
The Euler circuit/trail test:
\begin{enumerate}
  \item Count vertices of odd degree.
  \item 0 odd-degree vertices $\Rightarrow$ Euler circuit exists.
  \item 2 odd-degree vertices $\Rightarrow$ Euler trail exists (no circuit).
  \item $>2$ odd-degree vertices $\Rightarrow$ no Euler trail at all.
\end{enumerate}
\end{keyresult}

For Königsberg: each landmass had odd degree (3, 3, 3, 5). Four odd-degree vertices, so no Euler trail. The citizens couldn't do it.

\begin{definition}[Hamiltonian path and cycle]
A \textbf{Hamiltonian path} visits every \emph{vertex} exactly once.
A \textbf{Hamiltonian cycle} visits every vertex exactly once, returning to start.
\end{definition}

Here's an interesting contrast: Euler is about edges, Hamilton is about vertices. You might think they'd have similar characterizations, but they don't.

\begin{remark}
Unlike Euler paths/circuits, there's no simple test for Hamiltonian paths/cycles. Determining whether one exists is NP-complete---one of the classic hard problems in computer science.
\end{remark}

\subsection*{Graph representations}

How do we actually store a graph in a computer?

\begin{definition}[Adjacency matrix]
The \textbf{adjacency matrix} $A$ of a graph with $n$ vertices is an $n \times n$ matrix where:
\[
A_{ij} = \text{number of edges between vertex } i \text{ and vertex } j
\]
For simple graphs, $A_{ij} \in \{0, 1\}$. The matrix is symmetric for undirected graphs.
\end{definition}

\begin{proposition}[Properties of adjacency matrices]
\begin{itemize}
  \item The sum of row $i$ equals $\deg(v_i)$.
  \item The sum of all entries equals $2|E|$ (handshake theorem in matrix form).
  \item The diagonal is all zeros for simple graphs (no loops).
  \item $(A^k)_{ij}$ counts the number of walks of length $k$ from $v_i$ to $v_j$.
\end{itemize}
\end{proposition}

That last property is remarkable. Matrix multiplication corresponds to ``extending walks by one step.'' The $(i,j)$ entry of $A^2$ counts length-2 walks from $i$ to $j$ because it sums over all intermediate vertices $m$: how many ways to go $i \to m \to j$?

\begin{definition}[Adjacency list]
An \textbf{adjacency list} stores, for each vertex, a list of its neighbors. More space-efficient for sparse graphs (when $|E| \ll |V|^2$).
\end{definition}

Adjacency matrix: $O(|V|^2)$ space, $O(1)$ edge lookup.
Adjacency list: $O(|V| + |E|)$ space, $O(\deg(v))$ edge lookup.
Choose based on your graph's density.

\subsection*{Graph isomorphism}

When are two graphs ``the same''? They might have different vertex names, different drawings, but the same underlying structure.

\begin{definition}[Graph isomorphism]
Two graphs $G_1 = (V_1, E_1)$ and $G_2 = (V_2, E_2)$ are \textbf{isomorphic}, written $G_1 \cong G_2$, if there exists a bijection $f: V_1 \to V_2$ such that:
\[
\{u, v\} \in E_1 \iff \{f(u), f(v)\} \in E_2
\]
The function $f$ is an \textbf{isomorphism}.
\end{definition}

An isomorphism is a relabeling of vertices that preserves adjacency. If two graphs are isomorphic, they have the same structure---they're the same graph wearing different name tags.

\begin{keyresult}[Categorical terminology]
In the category $\cat{Graph}$, graph isomorphisms are exactly the bijective graph homomorphisms whose inverse is also a homomorphism. This matches the general categorical definition: an isomorphism is a morphism with a two-sided inverse.

A graph homomorphism is a monomorphism if and only if it's injective on both vertices and edges. It's an epimorphism if every vertex and edge in the codomain is ``hit.'' Graph isomorphisms are both---they're bijective morphisms with bijective inverses.
\end{keyresult}

\begin{theorem}[Isomorphism invariants]
If $G_1 \cong G_2$, then they share:
\begin{enumerate}
  \item Vertex count: $|V_1| = |V_2|$
  \item Edge count: $|E_1| = |E_2|$
  \item Degree sequence (sorted list of degrees)
  \item Number of cycles of each length
  \item Number of connected components
  \item Corresponding subgraph structure
\end{enumerate}
These are \emph{necessary} but not \emph{sufficient}. Two graphs can match on all these counts and still not be isomorphic.
\end{theorem}

\begin{strategy}
To show two graphs are NOT isomorphic: find an invariant they differ on.
To show they ARE isomorphic: construct an explicit bijection and verify edge preservation.
\end{strategy}

\begin{definition}[Automorphism]
An \textbf{automorphism} is an isomorphism from a graph to itself---a ``symmetry'' of the graph. The set of automorphisms forms a group under composition.
\end{definition}

A highly symmetric graph (like the complete graph or the hypercube) has many automorphisms. A ``lopsided'' graph with no symmetry has only the trivial automorphism (the identity).

\subsection*{Distance and diameter}

\begin{definition}[Distance]
The \textbf{distance} $d(u, v)$ between vertices $u$ and $v$ is the length of a shortest path between them. If no path exists, $d(u, v) = \infty$.
\end{definition}

Distance is a metric: $d(u,u) = 0$, $d(u,v) = d(v,u)$, and $d(u,w) \leq d(u,v) + d(v,w)$.

\begin{definition}[Eccentricity, radius, diameter]
\begin{itemize}
  \item The \textbf{eccentricity} of vertex $v$ is the maximum distance from $v$ to any other vertex: how far away is the farthest point?
  \item The \textbf{diameter} of a connected graph is the maximum eccentricity---the farthest two points can be.
  \item The \textbf{radius} is the minimum eccentricity---there's a ``center'' vertex that's as close to everything as possible.
  \item A \textbf{center} is a vertex with eccentricity equal to the radius.
\end{itemize}
\end{definition}

\subsection*{Graph coloring}

Here's a classic problem: given a map, color the countries so no two adjacent countries share a color. How many colors do you need?

\begin{definition}[Vertex coloring]
A \textbf{(proper) vertex coloring} assigns colors to vertices such that no two adjacent vertices share a color. A \textbf{$k$-coloring} uses at most $k$ colors.
\end{definition}

\begin{definition}[Chromatic number]
The \textbf{chromatic number} $\chi(G)$ is the minimum number of colors needed to properly color $G$.
\end{definition}

\begin{theorem}[Chromatic number bounds]
\begin{enumerate}
  \item $\chi(G) \geq \omega(G)$, where $\omega(G)$ is the clique number (largest complete subgraph). If there's a $K_5$ hiding in your graph, you need at least 5 colors.
  \item $\chi(G) \leq \Delta(G) + 1$, where $\Delta(G)$ is the maximum degree. Greedy coloring never needs more than this.
  \item \textbf{Brooks' theorem:} If $G$ is connected and isn't a complete graph or odd cycle, then $\chi(G) \leq \Delta(G)$.
\end{enumerate}
\end{theorem}

\begin{theorem}[Chromatic numbers of special graphs]
\begin{itemize}
  \item $\chi(K_n) = n$ --- complete graph needs $n$ colors (everyone's adjacent)
  \item $\chi(C_n) = 2$ if $n$ is even, $\chi(C_n) = 3$ if $n$ is odd
  \item $\chi(K_{m,n}) = 2$ --- bipartite graphs are 2-colorable (color the parts!)
  \item Trees (with at least one edge) have $\chi(T) = 2$
\end{itemize}
\end{theorem}

\begin{definition}[Bipartite graph]
A graph is \textbf{bipartite} if its vertices can be split into two sets where every edge crosses between sets. Equivalently, $\chi(G) \leq 2$.
\end{definition}

\begin{theorem}[Bipartite characterization]
A graph is bipartite if and only if it contains no odd-length cycle.
\end{theorem}

This is a beautiful characterization: odd cycles are the only obstruction to bipartiteness.

\subsection*{Planar graphs}

\begin{definition}[Planar graph]
A graph is \textbf{planar} if it can be drawn in the plane with no edge crossings (except at vertices).
\end{definition}

\begin{definition}[Faces]
In a planar drawing, the plane is divided into \textbf{faces} (regions), including one unbounded outer face.
\end{definition}

\begin{theorem}[Euler's formula for planar graphs]
For a connected planar graph with $V$ vertices, $E$ edges, and $F$ faces:
\[
V - E + F = 2
\]
\end{theorem}

This is a topological invariant---it doesn't depend on how you draw the graph, only on its structure.

\begin{theorem}[Edge bound for planar graphs]
For a connected planar graph with $V \geq 3$:
\[
E \leq 3V - 6
\]
If triangle-free (no 3-cycles), then $E \leq 2V - 4$.
\end{theorem}

\begin{corollary}
$K_5$ and $K_{3,3}$ are not planar.
\end{corollary}

\begin{proof}
$K_5$: $V = 5$, $E = 10$. But $3V - 6 = 9 < 10$. Too many edges.

$K_{3,3}$: $V = 6$, $E = 9$. It's bipartite, so triangle-free. Need $E \leq 2V - 4 = 8 < 9$. Too many edges.
\end{proof}

\begin{theorem}[Kuratowski's theorem]
A graph is planar if and only if it contains no subdivision of $K_5$ or $K_{3,3}$.
\end{theorem}

A \textbf{subdivision} inserts degree-2 vertices into edges---``stretching'' them without changing the essential structure.

\begin{theorem}[Four Color Theorem]
Every planar graph can be colored with at most 4 colors.
\end{theorem}

This was conjectured in 1852 and finally proved in 1976---with substantial computer assistance. It remains one of the most famous theorems to require computational verification.

\subsection*{Worked examples}

\begin{example}[Euler circuit check]
Does a connected graph with degrees $(2, 2, 2, 4, 4)$ have an Euler circuit?

\emph{Solution.} All degrees are even, so yes. By Euler's theorem, an Euler circuit exists.
\end{example}

\begin{example}[No Euler trail]
Does $K_4$ have an Euler circuit?

\emph{Solution.} In $K_4$, every vertex has degree 3 (odd). All four vertices have odd degree. Since we need 0 or 2 odd-degree vertices for an Euler trail, $K_4$ has neither an Euler circuit nor an Euler trail.
\end{example}

\begin{example}[Euler circuit exists]
Does $K_5$ have an Euler circuit?

\emph{Solution.} In $K_5$, every vertex has degree 4 (even). All vertices have even degree, so an Euler circuit exists.
\end{example}

\begin{example}[Adjacency matrix]
Find the adjacency matrix for the cycle $C_4$ on vertices $\{1, 2, 3, 4\}$.

\emph{Solution.} Edges are $\{1,2\}, \{2,3\}, \{3,4\}, \{4,1\}$.
\[
A = \begin{pmatrix}
0 & 1 & 0 & 1 \\
1 & 0 & 1 & 0 \\
0 & 1 & 0 & 1 \\
1 & 0 & 1 & 0
\end{pmatrix}
\]
Notice: symmetric, zero diagonal, each row sums to 2 (the degree).
\end{example}

\begin{example}[Matrix entries sum]
Show that the sum of entries in an adjacency matrix of a simple graph equals $2|E|$.

\emph{Solution.} Each edge $\{u, v\}$ contributes 1 to entry $(u, v)$ and 1 to entry $(v, u)$. Total contribution per edge: 2. Sum over all edges: $2|E|$.
\end{example}

\begin{example}[Testing isomorphism]
Are these graphs isomorphic?

$G_1$: vertices $\{a, b, c, d\}$, edges $\{a,b\}, \{b,c\}, \{c,d\}, \{d,a\}$

$G_2$: vertices $\{1, 2, 3, 4\}$, edges $\{1,2\}, \{2,3\}, \{3,4\}, \{4,1\}$

\emph{Solution.} Both are 4-cycles.
\begin{itemize}
  \item Same vertex count: 4 \checkmark
  \item Same edge count: 4 \checkmark
  \item Same degree sequence: $(2, 2, 2, 2)$ \checkmark
\end{itemize}

The bijection $f: a \mapsto 1, b \mapsto 2, c \mapsto 3, d \mapsto 4$ preserves edges. So $G_1 \cong G_2$.
\end{example}

\begin{example}[Non-isomorphic by edge count]
Prove $C_5$ and $K_5$ are not isomorphic.

\emph{Solution.} $C_5$ has 5 edges. $K_5$ has $\binom{5}{2} = 10$ edges. Different edge counts, so not isomorphic.
\end{example}

\begin{example}[Same degree sequence, not isomorphic]
Are two graphs with the same degree sequence necessarily isomorphic?

\emph{Solution.} No! Consider $C_6$ (a 6-cycle) and $K_3 \sqcup K_3$ (two disjoint triangles). Both have degree sequence $(2, 2, 2, 2, 2, 2)$. But $C_6$ is connected and $K_3 \sqcup K_3$ is not. Not isomorphic.
\end{example}

\begin{example}[Diameter of complete graph]
Find the diameter of $K_n$.

\emph{Solution.} Every pair of vertices is adjacent, so $d(u, v) = 1$ for all $u \neq v$. Diameter = 1.
\end{example}

\begin{example}[Finding an Euler trail]
Find an Euler trail in the graph with vertices $\{A, B, C, D\}$ and edges $\{A,B\}, \{A,C\}, \{B,C\}, \{B,D\}, \{C,D\}$.

\emph{Solution.} First check degrees: $\deg(A) = 2$, $\deg(B) = 3$, $\deg(C) = 3$, $\deg(D) = 2$.

Odd-degree vertices: $B$ and $C$ (exactly 2). An Euler trail exists, starting at $B$ and ending at $C$ (or vice versa).

One Euler trail from $B$: $B \to A \to C \to B \to D \to C$.

Check: uses edges $\{B,A\}, \{A,C\}, \{C,B\}, \{B,D\}, \{D,C\}$---all 5 edges, each exactly once. \checkmark
\end{example}

\begin{example}[Powers of adjacency matrix]
Compute $A^2$ for the path $P_3$ on vertices $\{1, 2, 3\}$ with edges $\{1,2\}$ and $\{2,3\}$. Interpret the result.

\emph{Solution.}
\[
A = \begin{pmatrix} 0 & 1 & 0 \\ 1 & 0 & 1 \\ 0 & 1 & 0 \end{pmatrix}
\]
\[
A^2 = \begin{pmatrix} 1 & 0 & 1 \\ 0 & 2 & 0 \\ 1 & 0 & 1 \end{pmatrix}
\]

$(A^2)_{ij}$ counts length-2 walks from $i$ to $j$:
\begin{itemize}
  \item $(A^2)_{11} = 1$: one walk $1 \to 2 \to 1$
  \item $(A^2)_{13} = 1$: one walk $1 \to 2 \to 3$
  \item $(A^2)_{22} = 2$: two walks $2 \to 1 \to 2$ and $2 \to 3 \to 2$
\end{itemize}
\end{example}

\begin{commonmistake}
\textbf{Confusing Euler and Hamiltonian.}
\begin{itemize}
  \item Euler: visits every \emph{edge} exactly once.
  \item Hamiltonian: visits every \emph{vertex} exactly once.
\end{itemize}
Euler has a simple characterization (degree parity). Hamiltonian is NP-complete to decide.
\end{commonmistake}

\begin{commonmistake}
\textbf{Thinking matching invariants proves isomorphism.} Equal vertex count, edge count, and degree sequence are necessary but not sufficient. You must either construct an explicit bijection or find a property they differ on.
\end{commonmistake}

\begin{goingdeeper}[Going Deeper: Graphs Generate Categories]
Graphs give rise to categories in a natural way. This perspective explains why adjacency matrices count paths.
For more detail, see the Category Theory Companion, Weeks 6--7.

\subsubsection*{The Free Category on a Graph}

Given a directed graph $G$, we can build a category $\mathbf{Path}(G)$:
\begin{itemize}
  \item \textbf{Objects:} Vertices of $G$
  \item \textbf{Morphisms from $u$ to $v$:} Directed paths from $u$ to $v$
  \item \textbf{Composition:} Concatenation of paths
  \item \textbf{Identity at $v$:} The empty path (length 0) staying at $v$
\end{itemize}

This is the \emph{free category} on $G$---the category with ``just enough structure'' to capture the graph.

\textbf{Example.} For the graph $1 \to 2 \to 3$:
\begin{itemize}
  \item Morphisms $1 \to 3$: just the path $1 \to 2 \to 3$ (one morphism)
  \item Morphisms $2 \to 2$: just the empty path $\id_2$
  \item Morphisms $3 \to 1$: none (no directed path backward)
\end{itemize}

\textbf{Example.} For a directed cycle $1 \to 2 \to 3 \to 1$:
\begin{itemize}
  \item Morphisms $1 \to 1$: the empty path, once around, twice around, thrice around, and so on
  \item Infinitely many morphisms! The cycle generates unboundedly many paths.
\end{itemize}

\subsubsection*{Adjacency Matrices Count Morphisms}

Here's the categorical insight: $(A^k)_{ij}$ counts the morphisms of length $k$ from $i$ to $j$ in $\mathbf{Path}(G)$.

For $k = 2$:
\[
(A^2)_{ij} = \sum_m A_{im} \cdot A_{mj}
\]
Each term counts paths that go $i \to m \to j$ for intermediate vertex $m$. This is exactly composition of morphisms in $\mathbf{Path}(G)$.

\subsubsection*{Composition as Matrix Multiplication}

The correspondence:
\begin{center}
\begin{tabular}{cc}
\textbf{Category} & \textbf{Matrix} \\
\hline
Composition of paths & Matrix multiplication \\
Length-$k$ paths & $A^k$ \\
Identity (length-0 path) & $I$ (identity matrix)
\end{tabular}
\end{center}

\subsubsection*{Connected Components as a Functor}
There's a functor $\pi_0: \cat{Graph} \to \Set$ that sends a graph to its set of connected components. A graph homomorphism $f: G \to H$ induces a function $\pi_0(f)$ by sending each component of $G$ to the component of $H$ containing its image.

This functor captures the fact that connectivity is preserved by structure-preserving maps.

\subsubsection*{Graph Homomorphisms and Colorings}

A graph $k$-coloring is secretly a graph homomorphism! If $K_k$ is the complete graph on $k$ vertices (``colors''), then a $k$-coloring of $G$ is a homomorphism $G \to K_k$. Adjacent vertices in $G$ must map to adjacent vertices in $K_k$---which are all pairs, so they must map to \emph{different} colors.

\subsubsection*{Exercises: Graphs and Paths}

\begin{enumerate}
  \item For the graph $1 \to 2 \to 3$, list all morphisms in $\mathbf{Path}(G)$ from each vertex to each vertex.

  \item For the directed 3-cycle $1 \to 2 \to 3 \to 1$, how many morphisms of length 3 are there from 1 to 1? Verify using $A^3$.

  \item For a graph with edges $a \to b$, $b \to c$, $a \to c$: how many distinct paths are there from $a$ to $c$?

  \item Write the adjacency matrix for the 3-cycle. Compute $A^2$ and verify that $(A^2)_{11}$ equals the number of length-2 paths from 1 to 1.

  \item A graph has edges $1 \to 2$, $2 \to 1$ (a 2-cycle). How many morphisms of length 4 are there from 1 to 1? Compute using $A^4$.

  \item \textbf{Challenge:} If $G$ is a directed acyclic graph (DAG), prove that $\mathbf{Path}(G)$ has finitely many morphisms between any two vertices.

  \item Explain why a $k$-coloring of graph $G$ corresponds to a graph homomorphism $G \to K_k$.

  \item In the path category, explain why composition is associative and why the empty path is the identity.
\end{enumerate}
\end{goingdeeper}

\subsection*{Practice}
\begin{enumerate}
  \item Give the adjacency matrix for the 4-cycle $C_4$.

  \item Construct two non-isomorphic graphs with the same degree sequence and prove they're not isomorphic.

  \item Find an Euler trail in a graph with exactly two odd-degree vertices.

  \item Show that the sum of entries in an adjacency matrix equals $2|E|$.

  \item Prove: If $G$ is a simple graph and $\overline{G}$ is its complement, then $G \cong \overline{G}$ implies $|V| \equiv 0$ or $1 \pmod 4$.

  \item Compute $A^2$ for $K_3$ and interpret the entries.

  \item Prove that every connected graph on $n$ vertices has at least $n-1$ edges.

  \item Find the diameter of the $n$-cube $Q_n$.

  \item Does $K_{3,3}$ have an Euler circuit? An Euler trail? Justify.

  \item Prove that a graph is bipartite if and only if it contains no odd-length cycles.

  \item How many automorphisms does the cycle $C_n$ have?

  \item Prove: If $G$ is connected with exactly 2 odd-degree vertices, any Euler trail must start and end at those vertices.

  \item Find $\chi(C_7)$ and $\chi(C_8)$.

  \item Find the chromatic number of the wheel graph $W_5$ (a 5-cycle with a central vertex connected to all).

  \item Use Euler's formula to find the number of faces in a connected planar graph with 10 vertices and 15 edges.

  \item Prove that every planar graph has a vertex of degree at most 5.

  \item Is the Petersen graph planar? Prove your answer.

  \item A planar graph has 12 faces, and each face is bounded by exactly 3 edges. How many edges and vertices does it have?

  \item Give a 3-coloring of $K_4$ minus one edge.

  \item Prove: If $G$ is planar with no cycles of length $\leq 4$, then $E \leq \frac{5}{3}(V - 2)$.
\end{enumerate}
