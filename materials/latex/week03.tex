\section{Week 3: Relations and Modular Arithmetic}
\subsection*{Reading}
Epp \S 8.1--8.4.

\subsection*{Learning objectives}
\begin{itemize}
  \item Describe a relation using sets, matrices, or digraphs.
  \item Test whether a relation is reflexive, symmetric, or transitive.
  \item Form equivalence classes and connect them to partitions.
  \item Compute congruences and modular inverses.
\end{itemize}

\subsection*{Key definitions and facts}
\begin{definition}[Equivalence relation]
A relation $R$ on a set $A$ is an equivalence relation if it is reflexive, symmetric, and transitive.
Equivalence classes $[a]$ partition $A$.
\end{definition}

\begin{definition}[Congruence]
For $n\ge 2$, $a\equiv b \pmod{n}$ iff $n$ divides $a-b$.
This is an equivalence relation on $\Z$.
\end{definition}

\subsection*{Worked example}
\begin{example}
Show that $\equiv$ mod $n$ is an equivalence relation on $\Z$.
\emph{Sketch.} Reflexive: $n\mid (a-a)=0$.
Symmetric: if $n\mid (a-b)$ then $n\mid (b-a)$.
Transitive: if $n\mid (a-b)$ and $n\mid (b-c)$ then $n\mid (a-c)$.
\end{example}

\subsection*{Practice}
\begin{enumerate}
  \item For $n=5$, list the equivalence classes of $\Z$ modulo $n$.
  \item Find the inverse of $3$ modulo $7$.
  \item Decide whether the relation $xRy$ iff $x-y$ is even is an equivalence relation on $\Z$.
  \item Prove that every equivalence relation on $A$ defines a partition of $A$.
\end{enumerate}
