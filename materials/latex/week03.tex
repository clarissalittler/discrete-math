\section{Week 3: Relations and Modular Arithmetic}
\subsection*{Reading}
Epp \S 8.1--8.4.

\subsection*{Learning objectives}
\begin{itemize}
  \item Describe a relation using sets, matrices, or digraphs.
  \item Test whether a relation is reflexive, symmetric, or transitive.
  \item Form equivalence classes and connect them to partitions.
  \item Compute congruences and modular inverses.
  \item Apply the extended Euclidean algorithm.
\end{itemize}

\subsection*{Key definitions and facts}

\begin{definition}[Relation]
A \textbf{relation} from set $A$ to set $B$ is a subset $R \subseteq A \times B$. We write $aRb$ or $(a,b) \in R$ to indicate that $a$ is related to $b$. A relation on $A$ is a relation from $A$ to $A$.
\end{definition}

\begin{definition}[Properties of relations]
Let $R$ be a relation on set $A$.
\begin{itemize}
  \item $R$ is \textbf{reflexive} if $aRa$ for all $a \in A$.
  \item $R$ is \textbf{symmetric} if $aRb$ implies $bRa$ for all $a,b \in A$.
  \item $R$ is \textbf{antisymmetric} if $aRb$ and $bRa$ imply $a = b$ for all $a,b \in A$.
  \item $R$ is \textbf{transitive} if $aRb$ and $bRc$ imply $aRc$ for all $a,b,c \in A$.
  \item $R$ is \textbf{irreflexive} if $\neg(aRa)$ for all $a \in A$.
  \item $R$ is \textbf{total} if $aRb$ or $bRa$ for all $a,b \in A$.
\end{itemize}
\end{definition}

\begin{definition}[Equivalence relation]
A relation $R$ on a set $A$ is an \textbf{equivalence relation} if it is reflexive, symmetric, and transitive. For an equivalence relation $\sim$, the \textbf{equivalence class} of $a$ is:
\[
[a] = \{x \in A : x \sim a\}
\]
\end{definition}

\begin{theorem}[Equivalence classes partition]
If $\sim$ is an equivalence relation on $A$, then:
\begin{enumerate}
  \item Every element belongs to exactly one equivalence class.
  \item Two equivalence classes are either identical or disjoint.
  \item The equivalence classes partition $A$: $A = \bigsqcup_{a \in A} [a]$.
\end{enumerate}
\end{theorem}

\begin{definition}[Partial order]
A relation $R$ on $A$ is a \textbf{partial order} if it is reflexive, antisymmetric, and transitive. The pair $(A, R)$ is called a \textbf{partially ordered set} (poset).
\end{definition}

\begin{definition}[Total order]
A partial order $\leq$ on $A$ is a \textbf{total order} if for all $a, b \in A$, either $a \leq b$ or $b \leq a$.
\end{definition}

\begin{definition}[Congruence modulo $n$]
For integers $a, b$ and positive integer $n$, we say $a$ is \textbf{congruent} to $b$ modulo $n$, written $a \equiv b \pmod{n}$, if $n$ divides $a - b$. Equivalently:
\[
a \equiv b \pmod{n} \iff a \bmod n = b \bmod n \iff \exists k \in \Z: a = b + kn
\]
\end{definition}

\begin{theorem}[Congruence is an equivalence relation]
For any positive integer $n$, congruence modulo $n$ is an equivalence relation on $\Z$. The equivalence classes are the \textbf{residue classes}:
\[
[0], [1], [2], \ldots, [n-1]
\]
The set of residue classes is denoted $\Z_n$ or $\Z/n\Z$.
\end{theorem}

\begin{theorem}[Modular arithmetic]
For all $a, b, c, d \in \Z$ and positive integer $n$:
\begin{enumerate}
  \item If $a \equiv b \pmod{n}$ and $c \equiv d \pmod{n}$, then $a + c \equiv b + d \pmod{n}$.
  \item If $a \equiv b \pmod{n}$ and $c \equiv d \pmod{n}$, then $ac \equiv bd \pmod{n}$.
  \item If $a \equiv b \pmod{n}$, then $a^k \equiv b^k \pmod{n}$ for all $k \geq 0$.
\end{enumerate}
\end{theorem}

\begin{definition}[Modular inverse]
For $a \in \Z_n$, the \textbf{modular inverse} of $a$ modulo $n$ is an integer $b$ such that $ab \equiv 1 \pmod{n}$. We write $b = a^{-1} \pmod{n}$.
\end{definition}

\begin{theorem}[Existence of modular inverse]
The integer $a$ has a modular inverse modulo $n$ if and only if $\gcd(a, n) = 1$.
\end{theorem}

\begin{theorem}[Division algorithm]
For any integers $a$ and $b$ with $b > 0$, there exist unique integers $q$ (quotient) and $r$ (remainder) such that:
\[
a = bq + r \quad \text{and} \quad 0 \leq r < b
\]
\end{theorem}

\begin{theorem}[Euclidean algorithm]
For positive integers $a$ and $b$, $\gcd(a, b)$ can be computed by repeatedly applying: $\gcd(a, b) = \gcd(b, a \bmod b)$, until one argument becomes 0.
\end{theorem}

\begin{theorem}[Extended Euclidean algorithm (Bézout's identity)]
For any positive integers $a$ and $b$, there exist integers $x$ and $y$ such that:
\[
ax + by = \gcd(a, b)
\]
\end{theorem}

\begin{strategy}
To find the modular inverse of $a$ modulo $n$ when $\gcd(a,n) = 1$:
\begin{enumerate}
  \item Use the extended Euclidean algorithm to find $x, y$ with $ax + ny = 1$.
  \item Then $a^{-1} \equiv x \pmod{n}$.
\end{enumerate}
\end{strategy}

\subsection*{Representations of relations}

\begin{definition}[Matrix representation]
A relation $R$ on a finite set $A = \{a_1, \ldots, a_n\}$ can be represented by an $n \times n$ \textbf{relation matrix} $M$ where:
\[
M_{ij} = \begin{cases} 1 & \text{if } a_i R a_j \\ 0 & \text{otherwise} \end{cases}
\]
\end{definition}

\begin{proposition}[Matrix properties]
For a relation matrix $M$:
\begin{itemize}
  \item $R$ is reflexive iff all diagonal entries are 1.
  \item $R$ is symmetric iff $M = M^T$ (matrix is symmetric).
  \item $R$ is antisymmetric iff $M_{ij} = 1$ and $M_{ji} = 1$ imply $i = j$.
  \item $R$ is transitive iff $M^2$ (Boolean matrix product) has no 1 where $M$ has 0.
\end{itemize}
\end{proposition}

\begin{definition}[Digraph representation]
A relation $R$ on $A$ can be represented as a directed graph (digraph) with vertices $A$ and a directed edge from $a$ to $b$ whenever $aRb$.
\end{definition}

\subsection*{Closures}

\begin{definition}[Closure]
The \textbf{reflexive closure} of $R$ is the smallest reflexive relation containing $R$: $R \cup \{(a,a) : a \in A\}$.

The \textbf{symmetric closure} of $R$ is the smallest symmetric relation containing $R$: $R \cup R^{-1}$ where $R^{-1} = \{(b,a) : (a,b) \in R\}$.

The \textbf{transitive closure} of $R$, denoted $R^+$, is the smallest transitive relation containing $R$.
\end{definition}

\begin{theorem}[Computing transitive closure]
$R^+ = R \cup R^2 \cup R^3 \cup \cdots$ where $R^n = R \circ R^{n-1}$ (relation composition). For finite sets, $R^+ = R \cup R^2 \cup \cdots \cup R^n$ where $|A| = n$.
\end{theorem}

\subsection*{Worked examples}

\begin{example}
Show that congruence modulo $n$ is an equivalence relation on $\Z$.

\emph{Proof.}
\begin{itemize}
  \item \textbf{Reflexive:} For any $a \in \Z$, we have $a - a = 0 = n \cdot 0$, so $n \mid (a - a)$. Thus $a \equiv a \pmod{n}$.

  \item \textbf{Symmetric:} Suppose $a \equiv b \pmod{n}$. Then $n \mid (a - b)$, so $a - b = nk$ for some integer $k$. Then $b - a = -nk = n(-k)$, so $n \mid (b - a)$. Thus $b \equiv a \pmod{n}$.

  \item \textbf{Transitive:} Suppose $a \equiv b \pmod{n}$ and $b \equiv c \pmod{n}$. Then $a - b = nj$ and $b - c = nk$ for integers $j, k$. Adding: $a - c = (a - b) + (b - c) = nj + nk = n(j + k)$. So $n \mid (a - c)$, and $a \equiv c \pmod{n}$. \qed
\end{itemize}
\end{example}

\begin{example}
Find the inverse of $3$ modulo $7$.

\emph{Solution.} We need $x$ such that $3x \equiv 1 \pmod{7}$.

\textbf{Method 1 (Trial):} Check $3 \cdot 1 = 3$, $3 \cdot 2 = 6$, $3 \cdot 3 = 9 \equiv 2$, $3 \cdot 4 = 12 \equiv 5$, $3 \cdot 5 = 15 \equiv 1 \pmod{7}$. So $3^{-1} \equiv 5 \pmod{7}$.

\textbf{Method 2 (Extended Euclidean):}
\begin{align*}
7 &= 2 \cdot 3 + 1 \\
1 &= 7 - 2 \cdot 3 = 1 \cdot 7 + (-2) \cdot 3
\end{align*}
So $(-2) \cdot 3 + 1 \cdot 7 = 1$, meaning $3^{-1} \equiv -2 \equiv 5 \pmod{7}$.
\end{example}

\begin{example}
Let $A = \{1, 2, 3, 4\}$ and $R = \{(1,1), (1,2), (2,1), (2,2), (3,3), (3,4), (4,3), (4,4)\}$. Show this is an equivalence relation and find the equivalence classes.

\emph{Solution.}
\begin{itemize}
  \item \textbf{Reflexive:} $(1,1), (2,2), (3,3), (4,4) \in R$. \checkmark
  \item \textbf{Symmetric:} For each $(a,b) \in R$ with $a \neq b$: $(1,2) \in R$ and $(2,1) \in R$; $(3,4) \in R$ and $(4,3) \in R$. \checkmark
  \item \textbf{Transitive:} Check all pairs: $(1,2), (2,1) \Rightarrow (1,1) \in R$; $(3,4), (4,3) \Rightarrow (3,3) \in R$. All other transitive requirements are satisfied. \checkmark
\end{itemize}

Equivalence classes: $[1] = [2] = \{1, 2\}$ and $[3] = [4] = \{3, 4\}$.

The partition is $\{\{1,2\}, \{3,4\}\}$.
\end{example}

\begin{example}
Use the Euclidean algorithm to find $\gcd(252, 105)$.

\emph{Solution.}
\begin{align*}
252 &= 2 \cdot 105 + 42 \\
105 &= 2 \cdot 42 + 21 \\
42 &= 2 \cdot 21 + 0
\end{align*}
So $\gcd(252, 105) = 21$.
\end{example}

\begin{example}
Use the extended Euclidean algorithm to express $\gcd(252, 105)$ as a linear combination.

\emph{Solution.} Working backwards:
\begin{align*}
21 &= 105 - 2 \cdot 42 \\
   &= 105 - 2 \cdot (252 - 2 \cdot 105) \\
   &= 105 - 2 \cdot 252 + 4 \cdot 105 \\
   &= 5 \cdot 105 + (-2) \cdot 252
\end{align*}
So $21 = 5 \cdot 105 + (-2) \cdot 252$.
\end{example}

\begin{example}[Toy RSA example]
This example illustrates RSA arithmetic (not secure for real use). Let $p = 5$ and $q = 11$, so $n = pq = 55$ and $\phi(n) = (p-1)(q-1) = 4 \cdot 10 = 40$.

Choose public exponent $e = 3$ (which is coprime to 40). Find the private exponent $d$ such that $ed \equiv 1 \pmod{40}$.

\emph{Solution.} We need $3d \equiv 1 \pmod{40}$. Using trial or the extended Euclidean algorithm:
\[
40 = 13 \cdot 3 + 1 \implies 1 = 40 - 13 \cdot 3
\]
So $d \equiv -13 \equiv 27 \pmod{40}$. Check: $3 \cdot 27 = 81 = 2 \cdot 40 + 1 \equiv 1 \pmod{40}$. \checkmark

\textbf{Encryption:} To encrypt message $m = 12$, compute $c \equiv m^e \pmod{n}$:
\[
c \equiv 12^3 = 1728 \equiv 1728 - 31 \cdot 55 = 1728 - 1705 = 23 \pmod{55}
\]

\textbf{Decryption:} To decrypt, compute $c^d \pmod{n}$. We have $23^{27} \pmod{55}$.

Using repeated squaring modulo 55:
$23^2 = 529 \equiv 529 - 9 \cdot 55 = 34$, $23^4 \equiv 34^2 = 1156 \equiv 1156 - 21 \cdot 55 = 1$, so $23^{27} = 23^{24} \cdot 23^3 = (23^4)^6 \cdot 23^3 \equiv 1^6 \cdot 23^3 = 12167 \equiv 12 \pmod{55}$.

We recover the original message $m = 12$!
\end{example}

\begin{example}
Prove that if $R$ is symmetric and transitive, and every element is related to some element, then $R$ is reflexive.

\emph{Proof.} Let $a \in A$. By assumption, there exists $b$ such that $aRb$. By symmetry, $bRa$. By transitivity (using $aRb$ and $bRa$), we get $aRa$. Since $a$ was arbitrary, $R$ is reflexive. \qed
\end{example}

\subsection*{Advanced number theory}

The following theorems are powerful tools for modular arithmetic, especially in cryptography and algorithm design.

\begin{definition}[Euler's totient function]
For a positive integer $n$, \textbf{Euler's totient function} $\phi(n)$ counts the integers from 1 to $n$ that are coprime to $n$:
\[
\phi(n) = |\{k : 1 \leq k \leq n \text{ and } \gcd(k, n) = 1\}|
\]
\end{definition}

\begin{theorem}[Computing $\phi(n)$]
\begin{itemize}
  \item If $p$ is prime: $\phi(p) = p - 1$
  \item If $p$ is prime: $\phi(p^k) = p^k - p^{k-1} = p^{k-1}(p - 1)$
  \item If $\gcd(m, n) = 1$: $\phi(mn) = \phi(m)\phi(n)$ \quad (multiplicative)
\end{itemize}
In general, if $n = p_1^{a_1} p_2^{a_2} \cdots p_k^{a_k}$, then:
\[
\phi(n) = n \prod_{p \mid n} \left(1 - \frac{1}{p}\right) = n \cdot \frac{p_1 - 1}{p_1} \cdot \frac{p_2 - 1}{p_2} \cdots \frac{p_k - 1}{p_k}
\]
\end{theorem}

\begin{example}
Compute $\phi(12)$.

\emph{Solution.} $12 = 2^2 \cdot 3$. So:
\[
\phi(12) = 12 \cdot \left(1 - \frac{1}{2}\right) \cdot \left(1 - \frac{1}{3}\right) = 12 \cdot \frac{1}{2} \cdot \frac{2}{3} = 4
\]
We can verify: the integers from 1 to 12 coprime to 12 are $\{1, 5, 7, 11\}$. Indeed, $|\{1, 5, 7, 11\}| = 4$.
\end{example}

\begin{theorem}[Fermat's Little Theorem]
If $p$ is prime and $\gcd(a, p) = 1$, then:
\[
a^{p-1} \equiv 1 \pmod{p}
\]
Equivalently, for any integer $a$: $a^p \equiv a \pmod{p}$.
\end{theorem}

\begin{example}
Compute $2^{100} \bmod 7$.

\emph{Solution.} By Fermat's Little Theorem, $2^6 \equiv 1 \pmod{7}$ (since 7 is prime and $\gcd(2,7) = 1$).

Write $100 = 6 \cdot 16 + 4$. Then:
\[
2^{100} = 2^{6 \cdot 16 + 4} = (2^6)^{16} \cdot 2^4 \equiv 1^{16} \cdot 16 \equiv 16 \equiv 2 \pmod{7}
\]
\end{example}

\begin{theorem}[Euler's Theorem]
If $\gcd(a, n) = 1$, then:
\[
a^{\phi(n)} \equiv 1 \pmod{n}
\]
This generalizes Fermat's Little Theorem (when $n = p$ is prime, $\phi(p) = p-1$).
\end{theorem}

\begin{example}
Compute $7^{222} \bmod 12$.

\emph{Solution.} Since $\gcd(7, 12) = 1$ and $\phi(12) = 4$, Euler's theorem gives $7^4 \equiv 1 \pmod{12}$.

Write $222 = 4 \cdot 55 + 2$. Then:
\[
7^{222} = (7^4)^{55} \cdot 7^2 \equiv 1^{55} \cdot 49 \equiv 49 \equiv 1 \pmod{12}
\]
\end{example}

\begin{keyresult}
Euler's Theorem explains \emph{why RSA works}. If $n = pq$ with primes $p, q$, and $ed \equiv 1 \pmod{\phi(n)}$, then for any message $m$ coprime to $n$:
\[
(m^e)^d = m^{ed} = m^{1 + k\phi(n)} = m \cdot (m^{\phi(n)})^k \equiv m \cdot 1^k = m \pmod{n}
\]
Decryption recovers the original message!
\end{keyresult}

\begin{theorem}[Chinese Remainder Theorem (CRT)]
Let $n_1, n_2, \ldots, n_k$ be pairwise coprime positive integers (i.e., $\gcd(n_i, n_j) = 1$ for $i \neq j$). Then for any integers $a_1, a_2, \ldots, a_k$, the system of congruences:
\begin{align*}
x &\equiv a_1 \pmod{n_1} \\
x &\equiv a_2 \pmod{n_2} \\
&\vdots \\
x &\equiv a_k \pmod{n_k}
\end{align*}
has a unique solution modulo $N = n_1 n_2 \cdots n_k$.
\end{theorem}

\begin{strategy}
To solve a CRT system with two moduli $n_1$ and $n_2$:
\begin{enumerate}
  \item From $x \equiv a_1 \pmod{n_1}$, write $x = a_1 + n_1 t$ for some integer $t$.
  \item Substitute into $x \equiv a_2 \pmod{n_2}$: solve $a_1 + n_1 t \equiv a_2 \pmod{n_2}$ for $t$.
  \item Compute $x = a_1 + n_1 t$.
\end{enumerate}
\end{strategy}

\begin{example}
Solve the system:
\begin{align*}
x &\equiv 2 \pmod{3} \\
x &\equiv 3 \pmod{5}
\end{align*}

\emph{Solution.} From the first equation: $x = 2 + 3t$ for some integer $t$.

Substitute into the second: $2 + 3t \equiv 3 \pmod{5}$, so $3t \equiv 1 \pmod{5}$.

To find $3^{-1} \pmod{5}$: $3 \cdot 2 = 6 \equiv 1 \pmod{5}$, so $t \equiv 2 \pmod{5}$.

Thus $t = 2 + 5s$ for some $s$, and:
\[
x = 2 + 3(2 + 5s) = 8 + 15s
\]
So $x \equiv 8 \pmod{15}$.

\textbf{Check:} $8 = 2 + 2 \cdot 3$, so $8 \equiv 2 \pmod 3$. \checkmark \quad $8 = 3 + 1 \cdot 5$, so $8 \equiv 3 \pmod 5$. \checkmark
\end{example}

\begin{example}
Find the last two digits of $3^{100}$.

\emph{Solution.} We need $3^{100} \bmod 100$. Since $100 = 4 \cdot 25$ and $\gcd(4, 25) = 1$, we use CRT.

\textbf{Modulo 4:} $3^2 = 9 \equiv 1 \pmod{4}$, so $3^{100} = (3^2)^{50} \equiv 1 \pmod{4}$.

\textbf{Modulo 25:} $\phi(25) = 25(1 - 1/5) = 20$. By Euler, $3^{20} \equiv 1 \pmod{25}$.

Since $100 = 20 \cdot 5$: $3^{100} = (3^{20})^5 \equiv 1 \pmod{25}$.

\textbf{Combine using CRT:} We need $x$ with $x \equiv 1 \pmod{4}$ and $x \equiv 1 \pmod{25}$.

Both congruences give $x \equiv 1$, so $x \equiv 1 \pmod{100}$.

The last two digits of $3^{100}$ are $\boxed{01}$.
\end{example}

\begin{commonmistake}
\textbf{Forgetting the coprimality requirement.} Fermat's Little Theorem requires $\gcd(a, p) = 1$. For example, $3^5 \not\equiv 1 \pmod{5}$ because $\gcd(3, 5) = 1$... wait, that's wrong! Let's check: $3^4 = 81 \equiv 1 \pmod 5$. \checkmark

But $5^4 \not\equiv 1 \pmod 5$ because $\gcd(5, 5) = 5 \neq 1$. In fact, $5^4 \equiv 0 \pmod 5$.
\end{commonmistake}

\begin{example}
Write the relation matrix for ``divides'' on $\{1, 2, 3, 4\}$.

\emph{Solution.} $a \mid b$ means $a$ divides $b$.
\[
M = \begin{pmatrix}
1 & 1 & 1 & 1 \\
0 & 1 & 0 & 1 \\
0 & 0 & 1 & 0 \\
0 & 0 & 0 & 1
\end{pmatrix}
\]
Row $i$, column $j$ is 1 iff $i \mid j$. This is reflexive (diagonal is 1), antisymmetric (if $i \mid j$ and $j \mid i$ with $i,j \geq 1$, then $i = j$), and transitive. So divisibility is a partial order.
\end{example}

\begin{goingdeeper}[Going Deeper: Preorders---The Simplest Categories]
This week we discover that the arrow-and-diagram language from Weeks 1--2 applies beautifully to familiar structures: preorders. This gives us concrete, easy-to-visualize categories to practice with.

\subsubsection*{Preorders You Already Know}

A \emph{preorder} is a set with a reflexive and transitive relation. You've seen many:
\begin{itemize}
  \item $(\N, \leq)$: natural numbers with ``less than or equal to''
  \item $(\Pow(X), \subseteq)$: subsets of $X$ ordered by inclusion
  \item $(Div_n, \mid)$: divisors of $n$ ordered by divisibility
  \item $(\Z, \leq)$: integers with the usual ordering
\end{itemize}

\subsubsection*{Preorders as Categories}

Here's the key insight: \textbf{a preorder IS a category}. Given $(P, \leq)$:
\begin{itemize}
  \item \textbf{Objects:} Elements of $P$
  \item \textbf{Arrows:} There is exactly one arrow $a \to b$ if $a \leq b$, and no arrow otherwise
  \item \textbf{Identity:} $a \leq a$ (reflexivity) gives the identity arrow $a \to a$
  \item \textbf{Composition:} $a \leq b$ and $b \leq c$ imply $a \leq c$ (transitivity)
\end{itemize}

This is called a \emph{thin category}: between any two objects, there's \emph{at most one} arrow. The existence of an arrow $a \to b$ simply records that $a \leq b$.

\textbf{Example: Divisibility.} Consider $(\{1, 2, 3, 6\}, \mid)$:
\[
\begin{tikzcd}
 & 6 & \\
2 \arrow[ur] & & 3 \arrow[ul] \\
 & 1 \arrow[ul] \arrow[ur] &
\end{tikzcd}
\]
We have arrows $1 \to 2$ (since $1 \mid 2$), $2 \to 6$, etc. The composite $1 \to 2 \to 6$ equals the direct arrow $1 \to 6$ (both just say $1 \mid 6$).

\subsubsection*{What Composition Means Here}

In a preorder category, composition is \emph{invisible}---there's only one arrow between any two comparable elements anyway. But it corresponds exactly to \textbf{transitivity}:
\[
\text{Arrow } a \to b \text{ and arrow } b \to c \text{ compose to give arrow } a \to c
\]
This is just: $a \leq b$ and $b \leq c$ imply $a \leq c$.

\subsubsection*{Monotone Functions = Structure-Preserving Maps}

In Weeks 1--2, we emphasized that the arrows between objects matter as much as the objects themselves. What are the ``good'' maps between preorders?

A function $f: P \to Q$ between preorders is \textbf{monotone} (or order-preserving) if:
\[
a \leq b \implies f(a) \leq f(b)
\]
In categorical language: \emph{if there's an arrow $a \to b$, then there's an arrow $f(a) \to f(b)$}. The function $f$ ``preserves arrows.''

Such a structure-preserving map between categories is called a \textbf{functor}.

\textbf{Examples of monotone functions:}
\begin{itemize}
  \item $\lfloor \cdot \rfloor: (\R, \leq) \to (\Z, \leq)$ (floor function)
  \item $|S|: (\Pow(X), \subseteq) \to (\N, \leq)$ (cardinality, when $X$ is finite)
  \item $n \mapsto 2n: (\N, \leq) \to (\N, \leq)$
\end{itemize}

\subsubsection*{Products and Coproducts in Preorders}

The universal properties from Week 2 specialize nicely to preorders:
\begin{itemize}
  \item \textbf{Product} of $a$ and $b$ = greatest lower bound (meet): the largest $c$ with $c \leq a$ and $c \leq b$
  \item \textbf{Coproduct} of $a$ and $b$ = least upper bound (join): the smallest $c$ with $a \leq c$ and $b \leq c$
\end{itemize}

\textbf{Example.} In $(\{1, 2, 3, 6\}, \mid)$:
\begin{itemize}
  \item Product of 2 and 3 is $\gcd(2, 3) = 1$
  \item Coproduct of 2 and 3 is $\text{lcm}(2, 3) = 6$
\end{itemize}

\subsubsection*{Exercises: Preorders as Categories}

\begin{enumerate}
  \item Draw the divisibility preorder on $\{1, 2, 4, 8\}$ as a diagram with arrows. Is this a total order (every two elements comparable)?

  \item List all arrows in the preorder $(\Pow(\{1, 2\}), \subseteq)$. How many are there? (Don't forget identity arrows!)

  \item Verify that transitivity is composition: In $\{1, 2, 4, 8\}$ with divisibility, we have $1 \mid 2$ and $2 \mid 4$. Draw this as composing arrows $1 \to 2 \to 4$, giving $1 \to 4$.

  \item Is the floor function $\lfloor \cdot \rfloor: \R \to \Z$ monotone? Prove or disprove.

  \item Is the function $f(n) = n^2$ monotone from $(\Z, \leq)$ to $(\Z, \leq)$? What about from $(\N, \leq)$ to $(\N, \leq)$?

  \item In $(\{1, 2, 3, 6\}, \mid)$, what is the product (greatest lower bound) of 2 and 3? What is their coproduct (least upper bound)?

  \item \textbf{Universal property check:} In divisibility, verify that $\gcd(6, 10)$ satisfies: for any $d$ with $d \mid 6$ and $d \mid 10$, we have $d \mid \gcd(6, 10)$.

  \item Draw the diagram expressing the universal property of $\gcd(a, b)$ using arrows in a divisibility preorder.

  \item How many monotone functions are there from $(\{0, 1\}, \leq)$ to $(\{a, b, c\}, \leq)$ where $a \leq b \leq c$? (Hint: Where can 0 go? Then where can 1 go?)

  \item Give an example of a function $\{1, 2, 3\} \to \{1, 2, 3\}$ that is NOT monotone for the usual $\leq$.

  \item \textbf{Functors compose:} Prove that if $f: P \to Q$ and $g: Q \to R$ are both monotone, then $g \circ f: P \to R$ is monotone.

  \item \textbf{Challenge:} In $(\Pow(\{1, 2, 3\}), \subseteq)$, what is the product (greatest lower bound) of $\{1, 2\}$ and $\{2, 3\}$? The coproduct (least upper bound)? Verify using the universal property.
\end{enumerate}
\end{goingdeeper}

\begin{commonmistake}
\textbf{Confusing symmetric and antisymmetric.} These are \emph{not} opposites!
\begin{itemize}
  \item Symmetric: $aRb \Rightarrow bRa$
  \item Antisymmetric: $aRb \land bRa \Rightarrow a = b$
\end{itemize}
A relation can be both (e.g., $=$), neither, or just one. The identity relation $\{(a,a) : a \in A\}$ is both symmetric and antisymmetric.
\end{commonmistake}

\subsection*{Practice}
\begin{enumerate}
  \item For $n = 5$, list the equivalence classes of $\Z$ modulo $n$.

  \item Find the inverse of $3$ modulo $7$ using the extended Euclidean algorithm.

  \item Decide whether the relation $xRy$ iff $x - y$ is even is an equivalence relation on $\Z$.

  \item Prove that every equivalence relation on $A$ defines a partition of $A$.

  \item Let $R = \{(a,b) \in \Z \times \Z : a \leq b\}$. Which properties does $R$ have: reflexive, symmetric, antisymmetric, transitive?

  \item Find $\gcd(1071, 462)$ and express it as a linear combination of $1071$ and $462$.

  \item Solve $17x \equiv 1 \pmod{43}$.

  \item Let $R$ be a relation on $\{1,2,3\}$ with matrix:
  \[
  M = \begin{pmatrix} 1 & 1 & 0 \\ 0 & 1 & 1 \\ 0 & 0 & 1 \end{pmatrix}
  \]
  Find the transitive closure $R^+$ and give its matrix.

  \item Prove: If $a \equiv b \pmod{n}$ and $c \mid n$, then $a \equiv b \pmod{c}$.

  \item Show that the intersection of two equivalence relations on $A$ is an equivalence relation.

  \item Compute $\phi(60)$ using the formula.

  \item Use Fermat's Little Theorem to find $5^{302} \bmod 7$.

  \item Use Euler's Theorem to find $3^{340} \bmod 11$.

  \item Solve the system: $x \equiv 1 \pmod{4}$, $x \equiv 2 \pmod{5}$, $x \equiv 3 \pmod{7}$.

  \item What are the last two digits of $7^{2024}$?

  \item Prove Fermat's Little Theorem for $a = 2$ and $p = 5$ by direct computation.
\end{enumerate}
