\section{Week 5: Counting and Probability II}

\textit{Or: ``The art of choosing without caring about order''}

\subsection*{Reading}
Epp \S 9.5--9.7; 5.6--5.7. Supplemental: generating functions.\\
\textbf{Category theory companion:} Weeks 4--5 (\texttt{category\_theory\_companion.pdf}).

\subsection*{Why Combinations?}

Last week we counted permutations---arrangements where order matters. But often we don't care about order: a committee is just a set of people, not a sequence. A poker hand is the same hand regardless of which card you drew first. A subset is a subset.

This leads to binomial coefficients, one of the most important tools in discrete mathematics. They show up everywhere: in probability, in algebra (the binomial theorem), in Pascal's triangle, in combinatorial proofs. Master these and you've mastered a significant chunk of counting.

\subsection*{Learning objectives}
\begin{itemize}
  \item Compute combinations and binomial coefficients.
  \item Count with repetition using stars and bars.
  \item Use Pascal's identity and the binomial theorem.
  \item Set up and solve basic recurrence relations.
\end{itemize}

\subsection*{Key definitions and facts}

\subsubsection*{Binomial Coefficients}

\begin{definition}[Binomial coefficient]
The \textbf{binomial coefficient} $\binom{n}{k}$ (read ``$n$ choose $k$'') is:
\[
\binom{n}{k} = \frac{n!}{k!(n-k)!}
\]
It counts the number of ways to choose $k$ items from $n$ items without regard to order. Equivalently, it counts $k$-element subsets of an $n$-element set.
\end{definition}

Why the formula? Start with the $k$-permutations: $P(n, k) = \frac{n!}{(n-k)!}$ ways to choose $k$ items in order. But each subset gets counted $k!$ times (once for each ordering of its elements). Divide out: $\binom{n}{k} = \frac{P(n,k)}{k!} = \frac{n!}{k!(n-k)!}$.

\begin{theorem}[Pascal's identity]
For $1 \leq k \leq n$:
\[
\binom{n}{k} = \binom{n-1}{k-1} + \binom{n-1}{k}
\]
\end{theorem}

\emph{Combinatorial proof:} Focus on element $n$. Every $k$-subset either contains $n$ or doesn't.
\begin{itemize}
  \item Subsets containing $n$: choose $k-1$ more from the other $n-1$ elements $\Rightarrow \binom{n-1}{k-1}$
  \item Subsets not containing $n$: choose all $k$ from the other $n-1$ elements $\Rightarrow \binom{n-1}{k}$
\end{itemize}
Add them up. \qed

This identity gives Pascal's triangle, where each entry is the sum of the two above it.

\begin{theorem}[Binomial theorem]
For any $x, y$ and non-negative integer $n$:
\[
(x+y)^n = \sum_{k=0}^n \binom{n}{k} x^{n-k} y^k
\]
\end{theorem}

Why? When you expand $(x+y)^n = (x+y)(x+y)\cdots(x+y)$, each term in the expansion comes from choosing $x$ or $y$ from each factor. The coefficient of $x^{n-k}y^k$ counts how many ways to choose $y$ from exactly $k$ of the $n$ factors, which is $\binom{n}{k}$.

\begin{theorem}[Useful identities]
\begin{itemize}
  \item $\binom{n}{k} = \binom{n}{n-k}$ \quad (symmetry: choosing what to include = choosing what to exclude)
  \item $\sum_{k=0}^{n} \binom{n}{k} = 2^n$ \quad (total number of subsets)
  \item $\sum_{k=0}^{n} (-1)^k \binom{n}{k} = 0$ \quad (alternating sum; set $x = 1, y = -1$ in binomial theorem)
  \item $\binom{n}{0} + \binom{n}{2} + \cdots = \binom{n}{1} + \binom{n}{3} + \cdots = 2^{n-1}$ \quad (even and odd sized subsets)
\end{itemize}
\end{theorem}

\subsubsection*{Stars and Bars}

\begin{theorem}[Stars and bars]
The number of ways to place $n$ identical objects into $k$ distinct bins is:
\[
\binom{n+k-1}{k-1}
\]
Equivalently, this counts non-negative integer solutions to $x_1 + x_2 + \cdots + x_k = n$.
\end{theorem}

\emph{Why it works:} Represent the objects as stars (*) and use bars (|) to separate bins. Example with $n=5$ objects in $k=3$ bins: $**|*|**$ represents $(2, 1, 2)$.

Any arrangement is a string of $n$ stars and $k-1$ bars, totaling $n + k - 1$ symbols. We choose which $k-1$ positions are bars: $\binom{n+k-1}{k-1}$.

\begin{definition}[Multinomial coefficient]
The number of ways to partition $n$ objects into groups of sizes $k_1, k_2, \ldots, k_r$ (where $k_1 + \cdots + k_r = n$) is:
\[
\binom{n}{k_1, k_2, \ldots, k_r} = \frac{n!}{k_1! k_2! \cdots k_r!}
\]
\end{definition}

This generalizes the binomial coefficient: $\binom{n}{k} = \binom{n}{k, n-k}$.

\subsection*{Recurrence relations}

\begin{definition}[Recurrence relation]
A \textbf{recurrence relation} defines a sequence $\{a_n\}$ by expressing $a_n$ in terms of earlier terms, together with initial conditions.
\end{definition}

For example, the Fibonacci sequence: $F_0 = 0, F_1 = 1$, and $F_n = F_{n-1} + F_{n-2}$ for $n \geq 2$.

\begin{theorem}[Second-order linear homogeneous recurrences]
If $a_n = c_1 a_{n-1} + c_2 a_{n-2}$ for $n \geq 2$, then the \textbf{characteristic equation} is:
\[
r^2 = c_1 r + c_2 \quad \text{or equivalently} \quad r^2 - c_1 r - c_2 = 0
\]
\begin{itemize}
  \item If roots are distinct ($r_1 \neq r_2$): $a_n = \alpha r_1^n + \beta r_2^n$
  \item If roots are repeated ($r_1 = r_2 = r$): $a_n = (\alpha + \beta n) r^n$
\end{itemize}
Constants $\alpha, \beta$ are determined from initial conditions.
\end{theorem}

\begin{example}
Solve $a_n = 2a_{n-1} + 1$ with $a_0 = 0$.

\emph{Solution.} This is inhomogeneous (the $+1$ term). One approach: unroll.
\begin{align*}
a_n &= 2a_{n-1} + 1 = 2(2a_{n-2} + 1) + 1 = 4a_{n-2} + 2 + 1 \\
&= 8a_{n-3} + 4 + 2 + 1 = \cdots = 2^n a_0 + (2^{n-1} + \cdots + 2 + 1) \\
&= 0 + (2^n - 1) = 2^n - 1
\end{align*}
Check: $a_0 = 2^0 - 1 = 0$ \checkmark, $a_1 = 2 \cdot 0 + 1 = 1 = 2^1 - 1$ \checkmark.
\end{example}

\begin{goingdeeper}[Going Deeper: Generating Functions]
The \textbf{ordinary generating function} (OGF) of a sequence $\{a_n\}$ is the formal power series:
\[
A(x) = \sum_{n=0}^{\infty} a_n x^n
\]

\textbf{Example 1 (constant sequence).} If $a_n = 1$ for all $n$:
\[
A(x) = 1 + x + x^2 + \cdots = \frac{1}{1-x}
\]

\textbf{Example 2 (Fibonacci).} Let $F(x) = \sum_{n \geq 0} F_n x^n$. The recurrence $F_n = F_{n-1} + F_{n-2}$ translates to:
\[
F(x) - x = x F(x) + x^2 F(x) \quad \Rightarrow \quad F(x) = \frac{x}{1 - x - x^2}
\]
Partial fractions give the closed form for $F_n$.

Generating functions are a powerful technique for solving recurrences and counting. The algebraic manipulation of series encodes combinatorial reasoning.
\end{goingdeeper}

\begin{goingdeeper}[Category Lens: Subsets as Maps to 2]
In $\Set$, every subset $S \subseteq A$ corresponds to a \textbf{characteristic function}:
\[
\chi_S: A \to 2, \quad \chi_S(a) = \begin{cases} 1 & \text{if } a \in S \\ 0 & \text{otherwise} \end{cases}
\]
where $2 = \{0, 1\}$.

So the set of all subsets of $A$ is the exponential object $2^A$ (the set of all functions from $A$ to $2$).

\begin{itemize}
  \item $|2^A| = 2^{|A|}$ explains why there are $2^n$ subsets of an $n$-element set.
  \item $\binom{n}{k}$ counts functions $A \to 2$ where exactly $k$ elements map to $1$.
\end{itemize}

This reframes binomial coefficients as counting \emph{maps} rather than subsets---a perspective that generalizes to other categories.
\end{goingdeeper}

\subsection*{Worked examples}

\begin{example}[Stars and bars]
How many non-negative integer solutions are there to $x_1 + x_2 + x_3 = 7$?

\emph{Solution.} We have 7 identical ``units'' to distribute among 3 distinct variables. By stars and bars:
\[
\binom{7 + 3 - 1}{3 - 1} = \binom{9}{2} = \frac{9 \cdot 8}{2} = 36
\]
\end{example}

\begin{example}[Proving a sum identity]
Prove $\sum_{k=0}^{n} \binom{n}{k} = 2^n$.

\emph{Proof 1 (Binomial theorem):} Set $x = y = 1$ in $(x+y)^n = \sum_{k=0}^n \binom{n}{k}x^{n-k}y^k$. Get $2^n = \sum_{k=0}^n \binom{n}{k}$.

\emph{Proof 2 (Combinatorial):} The LHS counts subsets by size (0-subsets + 1-subsets + ... + $n$-subsets). The RHS counts subsets directly: each element is in or out, $2^n$ choices. Both count all subsets. \qed
\end{example}

\begin{example}[Arrangements with repetition]
How many ways can the letters of MISSISSIPPI be arranged?

\emph{Solution.} Total 11 letters with repetitions: M(1), I(4), S(4), P(2).

Using the multinomial coefficient:
\[
\binom{11}{1,4,4,2} = \frac{11!}{1! \cdot 4! \cdot 4! \cdot 2!} = \frac{39916800}{1 \cdot 24 \cdot 24 \cdot 2} = 34650
\]
\end{example}

\begin{example}[Pascal's identity algebraically]
Prove $\binom{n}{k} = \binom{n-1}{k-1} + \binom{n-1}{k}$ algebraically.

\emph{Proof.}
\begin{align*}
\binom{n-1}{k-1} + \binom{n-1}{k} &= \frac{(n-1)!}{(k-1)!(n-k)!} + \frac{(n-1)!}{k!(n-k-1)!}
\end{align*}
Common denominator is $k!(n-k)!$:
\begin{align*}
&= \frac{(n-1)! \cdot k}{k!(n-k)!} + \frac{(n-1)! \cdot (n-k)}{k!(n-k)!} \\
&= \frac{(n-1)![k + (n-k)]}{k!(n-k)!} = \frac{(n-1)! \cdot n}{k!(n-k)!} = \frac{n!}{k!(n-k)!} = \binom{n}{k}
\end{align*}
\end{example}

\begin{example}[Binomial expansion]
Expand $(2x - 1)^4$.

\emph{Solution.} By the binomial theorem with $a = 2x$ and $b = -1$:
\begin{align*}
(2x - 1)^4 &= \sum_{k=0}^{4} \binom{4}{k} (2x)^{4-k} (-1)^k \\
&= (2x)^4 - 4(2x)^3 + 6(2x)^2 - 4(2x) + 1 \\
&= 16x^4 - 32x^3 + 24x^2 - 8x + 1
\end{align*}
\end{example}

\begin{example}[Positive integer solutions]
How many positive integer solutions are there to $x_1 + x_2 + x_3 = 10$?

\emph{Solution.} For positive integers ($x_i \geq 1$), substitute $y_i = x_i - 1$ so $y_i \geq 0$:
\[
(y_1 + 1) + (y_2 + 1) + (y_3 + 1) = 10 \implies y_1 + y_2 + y_3 = 7
\]
Now count non-negative solutions by stars and bars:
\[
\binom{7 + 3 - 1}{3 - 1} = \binom{9}{2} = 36
\]
\end{example}

\begin{example}[Committee with constraints]
A committee of 5 is chosen from 6 men and 4 women. How many committees have at least 2 women?

\emph{Solution.} ``At least 2 women'' means 2, 3, or 4 women:
\begin{itemize}
  \item 2 women, 3 men: $\binom{4}{2} \binom{6}{3} = 6 \cdot 20 = 120$
  \item 3 women, 2 men: $\binom{4}{3} \binom{6}{2} = 4 \cdot 15 = 60$
  \item 4 women, 1 man: $\binom{4}{4} \binom{6}{1} = 1 \cdot 6 = 6$
\end{itemize}
Total: $120 + 60 + 6 = 186$.

\emph{Alternative (complement):} Total committees: $\binom{10}{5} = 252$. Fewer than 2 women: $\binom{4}{0}\binom{6}{5} + \binom{4}{1}\binom{6}{4} = 6 + 60 = 66$. Answer: $252 - 66 = 186$. \checkmark
\end{example}

\begin{example}[Distributing identical objects]
In how many ways can 10 identical apples be distributed among 4 children?

\emph{Solution.} This is stars and bars with $n = 10$ objects, $k = 4$ bins:
\[
\binom{10 + 4 - 1}{4 - 1} = \binom{13}{3} = \frac{13 \cdot 12 \cdot 11}{6} = 286
\]
\end{example}

\begin{example}[Vandermonde's identity]
Prove $\binom{m+n}{r} = \sum_{k=0}^{r} \binom{m}{k}\binom{n}{r-k}$.

\emph{Combinatorial proof:} The LHS counts $r$-subsets of a set with $m$ red and $n$ blue elements. The RHS counts the same by cases: choose $k$ red elements and $r-k$ blue elements, for each possible $k$. \qed
\end{example}

\subsection*{Practice}
\begin{enumerate}
  \item Compute $\binom{12}{5}$.

  \item How many 8-card poker hands contain exactly 3 hearts?

  \item Prove Pascal's identity combinatorially.

  \item Use the binomial theorem to expand $(2x-1)^5$.

  \item How many positive integer solutions are there to $x_1 + x_2 + x_3 + x_4 = 15$?

  \item Prove: $\binom{n}{0}^2 + \binom{n}{1}^2 + \cdots + \binom{n}{n}^2 = \binom{2n}{n}$. (Hint: think about choosing $n$ items from $2n$.)

  \item A committee of 5 is to be chosen from 6 men and 4 women. How many committees have at least 2 women?

  \item Solve the recurrence $a_n = 3a_{n-1} - 2$ with $a_0 = 1$.

  \item Solve the recurrence $a_n = 4a_{n-1} - 4a_{n-2}$ with $a_0 = 1$, $a_1 = 2$.

  \item Let $t_n$ be the number of ways to tile a $2 \times n$ board with $1 \times 2$ dominoes. Find a recurrence for $t_n$ with initial conditions, and compute $t_5$.

  \item Find the ordinary generating function for the sequence $a_n = n$ for $n \geq 0$.
\end{enumerate}
