\section{Week 2: Functions and Cardinality}
\subsection*{Reading}
Epp \S 7.1--7.4.

\subsection*{Learning objectives}
\begin{itemize}
  \item Identify domain, codomain, and image of a function.
  \item Distinguish injective, surjective, and bijective functions.
  \item Use inverses and composition to solve functional equations.
  \item Compare sizes of sets using countability arguments.
\end{itemize}

\subsection*{Key definitions and facts}
\begin{definition}[Injective and surjective]
A function $f:A\to B$ is injective if $f(x)=f(y)$ implies $x=y$.
It is surjective if for every $b\in B$ there exists $a\in A$ with $f(a)=b$.
\end{definition}

\begin{proposition}[Composition]
If $f$ and $g$ are injective, then $g\circ f$ is injective.
If $f$ and $g$ are surjective, then $g\circ f$ is surjective.
\end{proposition}

\begin{definition}[Countable]
A set $A$ is countably infinite if there is a bijection $A\leftrightarrow\N$.
A set is countable if it is finite or countably infinite.
\end{definition}

\subsection*{Worked example}
\begin{example}
Let $f:\N\to\N$ be $f(n)=2n$. Show $f$ is injective but not surjective.
\emph{Sketch.} If $2n=2m$ then $n=m$, so $f$ is injective.
Odd numbers are not in the image, so $f$ is not surjective.
\end{example}

\subsection*{Practice}
\begin{enumerate}
  \item Give an explicit bijection between $\Z$ and $\N$.
  \item Decide whether $f(x)=x^3$ from $\R$ to $\R$ is bijective and justify.
  \item Prove that if $g\circ f$ is injective, then $f$ is injective.
  \item Show that a finite set cannot be in bijection with a proper subset of itself.
\end{enumerate}
