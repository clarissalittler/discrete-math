\section{Week 2: Functions and Cardinality}
\subsection*{Reading}
Epp \S 7.1--7.4.

\subsection*{Learning objectives}
\begin{itemize}
  \item Identify domain, codomain, and image of a function.
  \item Distinguish injective, surjective, and bijective functions.
  \item Use inverses and composition to solve functional equations.
  \item Compare sizes of sets using countability arguments.
  \item Apply the pigeonhole principle to function problems.
\end{itemize}

\subsection*{Key definitions and facts}
\begin{definition}[Function]
A function $f: A \to B$ assigns to each element $a \in A$ exactly one element $f(a) \in B$.
\begin{itemize}
  \item \textbf{Domain:} the set $A$
  \item \textbf{Codomain:} the set $B$
  \item \textbf{Image/Range:} $\{f(a) : a \in A\} \subseteq B$
\end{itemize}
\end{definition}

\begin{definition}[Image and preimage of sets]
Let $f: A \to B$ be a function.
\begin{itemize}
  \item For $S \subseteq A$, the \textbf{image} of $S$ under $f$ is $f(S) = \{f(x) : x \in S\} \subseteq B$.
  \item For $T \subseteq B$, the \textbf{preimage} (or inverse image) of $T$ under $f$ is $f^{-1}(T) = \{x \in A : f(x) \in T\} \subseteq A$.
\end{itemize}
\end{definition}

\begin{warning}
The notation $f^{-1}(T)$ for preimage does \emph{not} require $f$ to have an inverse function. The preimage $f^{-1}(T)$ is always defined as a set, even when $f$ is not bijective.
\end{warning}

\begin{definition}[Injective, surjective, bijective]
Let $f: A \to B$ be a function.
\begin{itemize}
  \item $f$ is \textbf{injective} (one-to-one) if $f(x) = f(y)$ implies $x = y$.
  \item $f$ is \textbf{surjective} (onto) if for every $b \in B$, there exists $a \in A$ with $f(a) = b$.
  \item $f$ is \textbf{bijective} if it is both injective and surjective.
\end{itemize}
\end{definition}

\begin{definition}[Inverse function]
If $f: A \to B$ is bijective, then $f^{-1}: B \to A$ exists and satisfies:
$f^{-1}(f(a)) = a$ for all $a \in A$, and $f(f^{-1}(b)) = b$ for all $b \in B$.
\end{definition}

\begin{proposition}[Composition preserves properties]
Let $f: A \to B$ and $g: B \to C$.
\begin{itemize}
  \item If $f$ and $g$ are injective, then $g \circ f$ is injective.
  \item If $f$ and $g$ are surjective, then $g \circ f$ is surjective.
  \item If $f$ and $g$ are bijective, then $g \circ f$ is bijective with $(g \circ f)^{-1} = f^{-1} \circ g^{-1}$.
\end{itemize}
\end{proposition}

\begin{proposition}[Composition tests]
\begin{itemize}
  \item If $g \circ f$ is injective, then $f$ is injective.
  \item If $g \circ f$ is surjective, then $g$ is surjective.
\end{itemize}
\end{proposition}

\begin{definition}[Countable and uncountable]
A set $A$ is \textbf{countably infinite} if there is a bijection $A \leftrightarrow \N$.
A set is \textbf{countable} if it is finite or countably infinite.
A set is \textbf{uncountable} if it is not countable.
\end{definition}

\begin{theorem}[Countability results]
\begin{itemize}
  \item $\Z$ and $\Q$ are countable.
  \item The union of countably many countable sets is countable.
  \item $\R$ is uncountable (Cantor's diagonal argument).
  \item $|\Pow(\N)| = |\R| > |\N|$.
\end{itemize}
\end{theorem}

\subsection*{Worked examples}
\begin{example}
Let $f: \Z \to \Z$ be defined by $f(n) = n^2 - 1$. Find $f(\{-2, 0, 3\})$ and $f^{-1}(\{0, 3, 8\})$.

\emph{Solution.}

\textbf{Image:} Compute $f$ on each element of $\{-2, 0, 3\}$:
\begin{itemize}
  \item $f(-2) = (-2)^2 - 1 = 4 - 1 = 3$
  \item $f(0) = 0^2 - 1 = -1$
  \item $f(3) = 3^2 - 1 = 9 - 1 = 8$
\end{itemize}
Thus $f(\{-2, 0, 3\}) = \{-1, 3, 8\}$.

\textbf{Preimage:} Find all $n \in \Z$ such that $f(n) = n^2 - 1 \in \{0, 3, 8\}$.
\begin{itemize}
  \item $n^2 - 1 = 0 \Rightarrow n^2 = 1 \Rightarrow n = \pm 1$
  \item $n^2 - 1 = 3 \Rightarrow n^2 = 4 \Rightarrow n = \pm 2$
  \item $n^2 - 1 = 8 \Rightarrow n^2 = 9 \Rightarrow n = \pm 3$
\end{itemize}
Thus $f^{-1}(\{0, 3, 8\}) = \{-3, -2, -1, 1, 2, 3\}$.
\end{example}

\begin{example}
Let $f:\N\to\N$ be $f(n)=2n$. Show $f$ is injective but not surjective.

\emph{Proof.}
\textbf{Injective:} Suppose $f(n) = f(m)$, i.e., $2n = 2m$. Dividing by 2, we get $n = m$. Thus $f$ is injective.

\textbf{Not surjective:} Consider $1 \in \N$. If $f(n) = 1$, then $2n = 1$, which has no solution in $\N$. So $1$ is not in the image of $f$. \qed
\end{example}

\begin{example}
Prove: If $g \circ f$ is injective, then $f$ is injective.

\emph{Proof.} Suppose $f(x) = f(y)$. Then $g(f(x)) = g(f(y))$, i.e., $(g \circ f)(x) = (g \circ f)(y)$.
Since $g \circ f$ is injective, $x = y$. Thus $f$ is injective. \qed
\end{example}

\begin{example}
Give a bijection between $\Z$ and $\N$.

\emph{Solution.} Define $f: \Z \to \N$ by:
\[
f(n) = \begin{cases}
2n & \text{if } n > 0 \\
-2n + 1 & \text{if } n \leq 0
\end{cases}
\]
This maps: $0 \mapsto 1$, $-1 \mapsto 3$, $1 \mapsto 2$, $-2 \mapsto 5$, $2 \mapsto 4$, etc.
\end{example}

\begin{example}
Determine whether $f: \R \to \R$ defined by $f(x) = x^3$ is bijective.

\emph{Solution.}

\textbf{Injective:} Suppose $f(a) = f(b)$, i.e., $a^3 = b^3$. Taking cube roots (which is well-defined on $\R$), we get $a = b$. So $f$ is injective.

\textbf{Surjective:} For any $y \in \R$, we need $x$ with $x^3 = y$. Take $x = \sqrt[3]{y}$ (cube roots exist for all real numbers, including negatives). Then $f(x) = (\sqrt[3]{y})^3 = y$. So $f$ is surjective.

Since $f$ is both injective and surjective, $f$ is bijective. The inverse is $f^{-1}(y) = \sqrt[3]{y}$.
\end{example}

\begin{example}
Prove that $(0, 1)$ is uncountable using Cantor's diagonal argument.

\emph{Proof.} Suppose for contradiction that $(0, 1)$ is countable, so we can list all numbers in $(0, 1)$:
\[
r_1 = 0.d_{11}d_{12}d_{13}\ldots, \quad r_2 = 0.d_{21}d_{22}d_{23}\ldots, \quad r_3 = 0.d_{31}d_{32}d_{33}\ldots, \quad \ldots
\]
where each $d_{ij}$ is a digit. Construct a new number $x = 0.e_1e_2e_3\ldots$ where:
\[
e_n = \begin{cases} 5 & \text{if } d_{nn} \neq 5 \\ 6 & \text{if } d_{nn} = 5 \end{cases}
\]
Then $x \in (0, 1)$ but $x \neq r_n$ for any $n$ (they differ in the $n$th decimal place). This contradicts the assumption that all elements of $(0, 1)$ were listed. Therefore $(0, 1)$ is uncountable. \qed
\end{example}

\begin{example}
Let $f: A \to B$ and $g: B \to C$. Prove that if $g \circ f$ is surjective, then $g$ is surjective.

\emph{Proof.} Let $c \in C$. We need to find some $b \in B$ with $g(b) = c$.

Since $g \circ f$ is surjective, there exists $a \in A$ with $(g \circ f)(a) = c$, i.e., $g(f(a)) = c$.

Let $b = f(a) \in B$. Then $g(b) = g(f(a)) = c$.

So for every $c \in C$, we found $b \in B$ with $g(b) = c$. Thus $g$ is surjective. \qed
\end{example}

\begin{goingdeeper}[Going Deeper: The Art of Diagram Chasing]
Building on Week 1's introduction to diagrams, we now develop \emph{diagram chasing} as a proof technique and encounter our first \emph{universal property}.

\subsubsection*{Diagram Chasing as Proof}

Given that some diagram commutes, we often want to prove that certain composites are equal. The method: find two paths between the same endpoints and use commutativity.

\textbf{Example.} Suppose this diagram commutes:
\[
\begin{tikzcd}
A \arrow[r, "f"] \arrow[d, "h"'] & B \arrow[d, "g"] \\
C \arrow[r, "k"'] & D
\end{tikzcd}
\]
and we also know that $m \circ g = n \circ k$ for some $m, n$. Then:
\[
m \circ g \circ f = n \circ k \circ h
\]
\emph{Proof:} $m \circ g \circ f = m \circ (g \circ f) = m \circ (k \circ h) = (m \circ k) \circ h$... wait, that's not quite right. Let's be more careful: $m \circ g \circ f = (m \circ g) \circ f = (n \circ k) \circ f$. Hmm, we need $k \circ h = g \circ f$. So: $m \circ g \circ f = m \circ (g \circ f) = m \circ (k \circ h)$. And $(m \circ g) \circ f = (n \circ k) \circ f$...

Actually, the key insight is simpler: from the square, $g \circ f = k \circ h$. So $m \circ g \circ f = m \circ k \circ h$. If additionally $m \circ k = n \circ k$... This shows why we must chase carefully!

\subsubsection*{Cancellation Properties}

Recall from the main notes that injective functions are ``left-cancellable'':
\[
f \circ g = f \circ h \implies g = h \quad \text{(when } f \text{ is injective)}
\]
And surjective functions are ``right-cancellable'':
\[
g \circ f = h \circ f \implies g = h \quad \text{(when } f \text{ is surjective)}
\]

These properties can be drawn as diagrams:
\[
\begin{tikzcd}
X \arrow[r, shift left, "g"] \arrow[r, shift right, "h"'] & A \arrow[r, "f"] & B
\end{tikzcd}
\quad\text{(left cancellation)}
\qquad
\begin{tikzcd}
A \arrow[r, "f"] & B \arrow[r, shift left, "g"] \arrow[r, shift right, "h"'] & X
\end{tikzcd}
\quad\text{(right cancellation)}
\]

\subsubsection*{The ``Unique Arrow'' Pattern}

A powerful pattern emerges: \emph{there exists a unique arrow making the diagram commute}. When we have uniqueness:
\begin{itemize}
  \item Any two arrows satisfying the condition must be equal
  \item This lets us prove equality by showing both arrows satisfy the same property
\end{itemize}

\subsubsection*{Universal Property of Products}

The Cartesian product $A \times B$ satisfies a \emph{universal property}: for any set $X$ with functions $f: X \to A$ and $g: X \to B$, there exists a \textbf{unique} function $\langle f, g \rangle: X \to A \times B$ making this diagram commute:
\[
\begin{tikzcd}
 & X \arrow[dl, "f"'] \arrow[d, dashed, "{\langle f, g \rangle}"] \arrow[dr, "g"] & \\
A & A \times B \arrow[l, "\pi_1"] \arrow[r, "\pi_2"'] & B
\end{tikzcd}
\]
The function is $\langle f, g \rangle(x) = (f(x), g(x))$. The universal property says this is the \emph{only} way to ``factor through'' $A \times B$.

\textbf{Why uniqueness matters:} If someone gives you another function $h: X \to A \times B$ with $\pi_1 \circ h = f$ and $\pi_2 \circ h = g$, you can immediately conclude $h = \langle f, g \rangle$!

\subsubsection*{Exercises: Diagram Chasing}

\begin{enumerate}
  \item \textbf{Cancellation practice:} Let $f: A \to B$ be injective, and suppose $f \circ g = f \circ h$ for $g, h: X \to A$. Prove $g = h$ by considering what happens element-by-element.

  \item Draw the diagram expressing ``$f$ is left-cancellable''---show $g$, $h$, and $f$, and indicate the conclusion $g = h$.

  \item If $g \circ f$ is injective, prove that $f$ must be injective. (Hint: Suppose $f(a) = f(b)$ and show $a = b$.)

  \item If $g \circ f$ is injective, must $g$ be injective? Prove or give a counterexample.

  \item If $g \circ f$ is surjective, prove that $g$ must be surjective.

  \item If $g \circ f$ is surjective, must $f$ be surjective? Prove or give a counterexample.

  \item \textbf{Product verification:} Let $A = \{1, 2\}$, $B = \{a, b, c\}$, $X = \{*\}$ (a one-element set). Define $f(*) = 1$ and $g(*) = b$. What is $\langle f, g \rangle(*)$?

  \item For $A = \{1, 2\}$, $B = \{a, b\}$, $X = \{x, y\}$, define $f(x) = 1$, $f(y) = 2$, $g(x) = a$, $g(y) = b$. Write out $\langle f, g \rangle$ explicitly.

  \item \textbf{Uniqueness in action:} Suppose $h, h': X \to A \times B$ both satisfy $\pi_1 \circ h = f$, $\pi_2 \circ h = g$ and the same for $h'$. Prove $h = h'$ using element-by-element reasoning.

  \item \textbf{Diagram chase:} Given this commuting diagram:
  \[
  \begin{tikzcd}
  A \arrow[r, "f"] \arrow[d, "h"'] & B \arrow[d, "g"] \\
  C \arrow[r, "k"'] & D
  \end{tikzcd}
  \]
  Prove that $g \circ f = k \circ h$ by writing out what ``the diagram commutes'' means.

  \item If $f: A \to B$ is a bijection with inverse $f^{-1}: B \to A$, draw the two triangles showing $f^{-1} \circ f = \id_A$ and $f \circ f^{-1} = \id_B$.

  \item \textbf{Challenge:} Prove that if $f: A \to B$ has both a left inverse $g$ (meaning $g \circ f = \id_A$) and a right inverse $h$ (meaning $f \circ h = \id_B$), then $g = h$. (Hint: Compute $g \circ f \circ h$ two ways.)
\end{enumerate}
\end{goingdeeper}

\subsection*{Practice}
\begin{enumerate}
  \item Give an explicit bijection between $\Z$ and $\N$.
  \item Decide whether $f(x)=x^3$ from $\R$ to $\R$ is bijective and justify.
  \item Prove that if $g\circ f$ is injective, then $f$ is injective.
  \item Show that a finite set cannot be in bijection with a proper subset of itself.
  \item Prove that $f: A \to B$ is injective iff there exists $g: B \to A$ with $g \circ f = \text{id}_A$.
  \item Prove that $f: A \to B$ is surjective iff there exists $g: B \to A$ with $f \circ g = \text{id}_B$.
  \item Show that the set of all finite subsets of $\N$ is countable.
\end{enumerate}
