\documentclass[11pt]{article}

\usepackage[margin=1in]{geometry}
\usepackage{amsmath,amssymb,amsthm}
\usepackage{booktabs,longtable}
\usepackage{tabularx}
\usepackage{enumitem}
\usepackage{hyperref}
\usepackage{textcomp}
\usepackage{microtype}
\usepackage{newunicodechar}

% Disable automatic section numbering (sections are already labeled "Week 1", etc.)
\setcounter{secnumdepth}{0}
\setcounter{tocdepth}{2}

\setlength{\parindent}{0pt}
\setlength{\parskip}{0.6em}

% Fallback if \textsubscript is not available
\providecommand{\textsubscript}[1]{\raisebox{-0.25ex}{\scriptsize #1}}

% --- Unicode character mappings ---
% The PDF-to-text extraction keeps many Unicode math symbols. These
% mappings make the .tex source compile cleanly under pdfLaTeX.

\newunicodechar{§}{\S}
\newunicodechar{±}{\ensuremath{\pm}}
\newunicodechar{·}{\ensuremath{\cdot}}
\newunicodechar{×}{\ensuremath{\times}}
\newunicodechar{Θ}{\ensuremath{\Theta}}
\newunicodechar{Σ}{\ensuremath{\Sigma}}
\newunicodechar{Ω}{\ensuremath{\Omega}}
\newunicodechar{δ}{\ensuremath{\delta}}
\newunicodechar{ε}{\ensuremath{\varepsilon}}
\newunicodechar{φ}{\ensuremath{\varphi}}

\newunicodechar{ᶜ}{\textsuperscript{c}}
\newunicodechar{–}{--}
\newunicodechar{—}{---}
\newunicodechar{’}{'}
\newunicodechar{“}{``}
\newunicodechar{”}{''}
\newunicodechar{•}{\textbullet}
\newunicodechar{…}{\ldots{}}
\newunicodechar{′}{\ensuremath{\prime}}

\newunicodechar{⁰}{\textsuperscript{0}}
\newunicodechar{¹}{\textsuperscript{1}}
\newunicodechar{²}{\textsuperscript{2}}
\newunicodechar{³}{\textsuperscript{3}}
\newunicodechar{⁴}{\textsuperscript{4}}
\newunicodechar{⁵}{\textsuperscript{5}}
\newunicodechar{⁶}{\textsuperscript{6}}
\newunicodechar{⁷}{\textsuperscript{7}}
\newunicodechar{⁹}{\textsuperscript{9}}
\newunicodechar{⁻}{\textsuperscript{-}}
\newunicodechar{ⁿ}{\textsuperscript{n}}

\newunicodechar{₀}{\textsubscript{0}}
\newunicodechar{₁}{\textsubscript{1}}
\newunicodechar{₂}{\textsubscript{2}}
\newunicodechar{₃}{\textsubscript{3}}
\newunicodechar{₄}{\textsubscript{4}}
\newunicodechar{₅}{\textsubscript{5}}
\newunicodechar{₆}{\textsubscript{6}}
\newunicodechar{₈}{\textsubscript{8}}
\newunicodechar{ₙ}{\textsubscript{n}}
\newunicodechar{ᵢ}{\textsubscript{i}}
\newunicodechar{ⱼ}{\textsubscript{j}}

\newunicodechar{ℕ}{\ensuremath{\mathbb{N}}}
\newunicodechar{ℤ}{\ensuremath{\mathbb{Z}}}
\newunicodechar{ℚ}{\ensuremath{\mathbb{Q}}}
\newunicodechar{ℝ}{\ensuremath{\mathbb{R}}}

\newunicodechar{→}{\ensuremath{\to}}
\newunicodechar{⇒}{\ensuremath{\Rightarrow}}
\newunicodechar{⇔}{\ensuremath{\Leftrightarrow}}
\newunicodechar{∀}{\ensuremath{\forall}}
\newunicodechar{∃}{\ensuremath{\exists}}
\newunicodechar{∅}{\ensuremath{\varnothing}}
\newunicodechar{∈}{\ensuremath{\in}}
\newunicodechar{∉}{\ensuremath{\notin}}
\newunicodechar{−}{-}
\newunicodechar{∘}{\ensuremath{\circ}}
\newunicodechar{∞}{\ensuremath{\infty}}
\newunicodechar{∩}{\ensuremath{\cap}}
\newunicodechar{∪}{\ensuremath{\cup}}
\newunicodechar{≈}{\ensuremath{\approx}}
\newunicodechar{≠}{\ensuremath{\ne}}
\newunicodechar{≡}{\ensuremath{\equiv}}
\newunicodechar{≤}{\ensuremath{\le}}
\newunicodechar{≥}{\ensuremath{\ge}}
\newunicodechar{⊂}{\ensuremath{\subset}}
\newunicodechar{⊆}{\ensuremath{\subseteq}}
\newunicodechar{⌈}{\ensuremath{\lceil}}
\newunicodechar{⌉}{\ensuremath{\rceil}}
\newunicodechar{⌊}{\ensuremath{\lfloor}}
\newunicodechar{⌋}{\ensuremath{\rfloor}}
\newunicodechar{■}{\ensuremath{\blacksquare}}

% --- End Unicode mappings ---

\title{Discrete Mathematics\\[0.25em]\large 10-week lecture notes, worked examples, and exercises (with solutions appendix)}
\author{}
\date{Version: December 2025}

\begin{document}
\maketitle
Aligned to sections in Susanna S. Epp, Discrete Mathematics with Applications (5th ed.).
These notes and exercises are original and are intended as a companion to the textbook, not a replacement.

\section{How to use these notes}
Each week (module) in this document corresponds to the section ranges shown in your course schedule. For each week you’ll find: (i) a brief reading guide, (ii) narrative lecture notes, (iii) fully worked examples, and (iv) a problem set.

An appendix at the end contains complete solutions to all of the problem sets. A good workflow is: read the notes → work the examples → attempt the exercises without peeking → then check the appendix.

Notation. This document uses standard symbols: ∈ (is an element of), ⊆ (subset), ∅ (empty set), ℕ (nonnegative integers), ℤ, ℚ, ℝ.

\section{Course map (10 weeks)}
\begin{longtable}{@{}r l p{0.23\linewidth} p{0.57\linewidth}@{}}
\toprule
Week & Textbook sections & Primary theme & Keywords \\
\midrule
\endhead
1 & 6.1–6.4 & Set theory & element method, set identities, power set, Boolean algebra, paradoxes \\
2 & 7.1–7.4 & Functions + cardinality & domain/codomain, injective/surjective, inverses, composition, countability \\
3 & 8.1–8.4 & Relations + modular arithmetic & properties, closures, equivalence classes, congruences, inverses mod n \\
4 & 9.1–9.4 & Counting \& probability I & sample spaces, multiplication/addition rules, inclusion–exclusion, pigeonhole \\
5 & 9.5–9.7 & Counting \& probability II & combinations, repetition, binomial coefficients, binomial theorem \\
6 & 9.8; 1.4; 4.9 & Expected value + intro graphs & probability axioms, expectation, graphs, degree, handshake theorem \\
7 & 10.1–10.3 & Graph theory I & trails/paths/circuits, Euler, adjacency matrices, isomorphism \\
8 & 10.4–10.6 & Trees \& graph algorithms & trees, rooted trees, spanning trees, shortest paths \\
9 & 12.1–12.3 & Regular expressions \& automata & regex, DFA, minimization, equivalence of states \\
10 & 11.1–11.5 & Algorithm efficiency & growth rates, O/Ω/Θ, analyzing loops/recurrences \\
\bottomrule
\end{longtable}

\tableofcontents
\clearpage
\clearpage
\section{Week 1: Set theory}
\textbf{Reading:} Epp §6.1–6.4
\subsection*{Learning objectives}
\begin{itemize}[leftmargin=*,itemsep=0.25em]
\item Use set-builder and roster notation fluently; translate between English statements and sets.
\item Prove set identities using the element method (show x ∈ LHS ⇔ x ∈ RHS).
\item Apply common set laws (commutative, associative, distributive, De Morgan) correctly.
\item Work with power sets and partitions; recognize when an example disproves a claim.
\item Interpret basic Boolean algebra as “algebra of sets” and connect it to logic.
\end{itemize}
\subsection{6.1 Sets, subsets, and the element method}
A set is a collection of distinct objects called elements. We write x ∈ A to mean “x is an
element of A,” and x ∉ A to mean “x is not an element of A.”
A set can be specified by listing its elements (roster notation) or by a defining property
(set-builder notation). For example, \{1, 2, 3\} = \{ x ∈ ℤ : 1 ≤ x ≤ 3 \}.
Key definitions
Subset: A ⊆ B means every element of A is also an element of B.
Proper subset: A ⊂ B means A ⊆ B and A ≠ B.
Set equality: A = B means A ⊆ B and B ⊆ A (equivalently, ∀x, x ∈ A ⇔ x ∈ B).
Common operations: union A ∪ B, intersection A ∩ B, difference A \(\setminus\) B, complement Aᶜ
(relative to a universe U).
The most reliable way to prove a set identity is the element method. To prove A = B, pick
an arbitrary element x and show x ∈ A iff x ∈ B. To prove A ⊆ B, pick arbitrary x ∈ A and
show x ∈ B.

\subsubsection*{Example 1.1 (Element method: distributive law)}
Prove that A ∩ (B ∪ C) = (A ∩ B) ∪ (A ∩ C).
Solution. Let x be arbitrary. Then:
x ∈ A ∩ (B ∪ C) ⇔ (x ∈ A) and (x ∈ B ∪ C).
The condition x ∈ B ∪ C means (x ∈ B) or (x ∈ C). So:
x ∈ A ∩ (B ∪ C) ⇔ (x ∈ A) and ((x ∈ B) or (x ∈ C)).
Distribute “and” over “or”: this is equivalent to ((x ∈ A and x ∈ B) or (x ∈ A and x
∈ C)).
That is, x ∈ (A ∩ B) ∪ (A ∩ C). Since x was arbitrary, the sets are equal. ■
Two more concepts appear constantly in discrete math:
Power set and partitions
Power set: P(A) is the set of all subsets of A. If A has n elements, then P(A) has 2ⁿ
elements.
Partition: A partition of A is a collection of nonempty subsets whose union is A and
which are pairwise disjoint.
\subsubsection*{Example 1.2 (Power set)}
Let A = \{a, b, c\}. List P(A) and verify that it has 2³ = 8 elements.
Solution. The subsets are: ∅, \{a\}, \{b\}, \{c\}, \{a,b\}, \{a,c\}, \{b,c\}, \{a,b,c\}. There are
8, matching 2³. ■
\subsection{6.2 Set laws and De Morgan’s laws}
Set operations satisfy algebraic laws that look like the laws of arithmetic. These laws let
you simplify expressions and prove identities. In practice, you’ll use two proof styles: (1)
the element method, and (2) an algebraic proof that rewrites one side into the other using
known laws.
De Morgan’s laws (for sets)
(A ∪ B)ᶜ = Aᶜ ∩ Bᶜ
(A ∩ B)ᶜ = Aᶜ ∪ Bᶜ

\subsubsection*{Example 1.3 (De Morgan via the element method)}
Prove (A ∪ B)ᶜ = Aᶜ ∩ Bᶜ.
Solution. Let x be arbitrary.
x ∈ (A ∪ B)ᶜ ⇔ x ∉ (A ∪ B).
x ∉ (A ∪ B) means “x is not in A and not in B,” i.e., (x ∉ A) and (x ∉ B).
That is equivalent to (x ∈ Aᶜ) and (x ∈ Bᶜ), which means x ∈ Aᶜ ∩ Bᶜ.
So x ∈ (A ∪ B)ᶜ ⇔ x ∈ Aᶜ ∩ Bᶜ for arbitrary x, hence the sets are equal. ■
\subsection{6.3 Disproofs and algebraic proofs}
To disprove a universal claim like “for all sets A, B, …, statement S holds,” you only need
one counterexample—one choice of sets that makes S false. Good counterexamples are
usually small, concrete sets (often subsets of \{1,2,3\}).
\subsubsection*{Example 1.4 (Disprove a false set identity)}
Claim: A \(\setminus\) (B ∩ C) = (A \(\setminus\) B) ∩ (A \(\setminus\) C). Is it true?
Solution. Take A = \{1,2\}, B = \{1\}, C = \{2\}.
Compute B ∩ C = ∅, so A \(\setminus\) (B ∩ C) = A \(\setminus\) ∅ = \{1,2\}.
But A \(\setminus\) B = \{2\} and A \(\setminus\) C = \{1\}, so (A \(\setminus\) B) ∩ (A \(\setminus\) C) = \{2\} ∩ \{1\} = ∅.
Left side is \{1,2\} and right side is ∅, so the claim is false. ■
\subsection{6.4 Boolean algebra, paradoxes, and why definitions matter}
The collection of sets (within a fixed universe U) behaves like a Boolean algebra: ∪ acts
like OR, ∩ acts like AND, and complement acts like NOT. This is more than a cute analogy:
it lets you translate problems between set identities and logical identities.
\subsubsection*{Example 1.5 (A Boolean-style simplification)}
Simplify the set expression (A ∩ B) ∪ (A ∩ Bᶜ).
Solution. Use distributivity:
(A ∩ B) ∪ (A ∩ Bᶜ) = A ∩ (B ∪ Bᶜ).
But B ∪ Bᶜ = U (the universe), so the expression becomes A ∩ U = A. ■

Finally, some famous paradoxes show why “set of all sets with property P” must be
handled carefully. Russell’s paradox is the classic example: if you allow the set R = \{ x : x
∉ x \}, then asking whether R ∈ R creates a contradiction. Modern set theory avoids this
by restricting how sets can be formed (axioms rather than unrestricted comprehension).
\subsection*{Week 1 problem set}
\textbf{M1-1.} Let U = \{1,2,3,4,5\}. Let A = \{1,2,4\}, B = \{2,3,5\}, C = \{1,3,4\}. Compute: (a) A ∪
B, (b) A ∩ C, (c) A \(\setminus\) B, (d) (B ∪ C)ᶜ.
\textbf{M1-2.} Prove using the element method that A ∩ B ⊆ A for all sets A and B.
\textbf{M1-3.} Prove (A \(\setminus\) B) \(\setminus\) C = A \(\setminus\) (B ∪ C).
\textbf{M1-4.} True or false? If A ⊆ B then P(A) ⊆ P(B). Prove your answer.
\textbf{M1-5.} How many subsets does a set with 9 elements have? Explain briefly.
\textbf{M1-6.} Let A = \{1,2,3,4\}. Consider the collection \{\{1,2\},\{3\},\{4\}\}. Is it a partition of A?
Justify.
\textbf{M1-7.} Simplify (A ∪ B) ∩ (A ∪ Bᶜ) as much as possible.
\textbf{M1-8.} Let U be a universe and let X, Y ⊆ U. Translate the logical statement “x ∈ X
implies x ∈ Y” into an equivalent set containment statement.

\clearpage
\section{Week 2: Functions and cardinality}
\textbf{Reading:} Epp §7.1–7.4
\subsection*{Learning objectives}
\begin{itemize}[leftmargin=*,itemsep=0.25em]
\item Use the definition of function precisely (domain, codomain, range) and work with images and preimages.
\item Decide whether a function is injective (one-to-one), surjective (onto), or bijective.
\item Compute and reason about compositions; know what can and cannot be cancelled in compositions.
\item Construct inverse functions when they exist, and prove inverse properties.
\item Understand countability: build bijections and use diagonal arguments for uncountability.
\end{itemize}
\subsection{7.1 Functions on general sets}
A function f from a set A to a set B is a rule that assigns to each a ∈ A exactly one element
f(a) ∈ B. We write f: A → B, call A the domain, B the codomain, and the set \{ f(a) : a ∈ A \}
the range (or image).
Images and preimages
For S ⊆ A, the image of S is f(S) = \{ f(x) : x ∈ S \} ⊆ B.
For T ⊆ B, the preimage of T is f⁻¹(T) = \{ x ∈ A : f(x) ∈ T \} ⊆ A.
Note: f⁻¹(T) always makes sense as a set, even when f has no inverse function.
\subsubsection*{Example 2.1 (Image and preimage)}
Let f: ℤ → ℤ be defined by f(n) = n² − 1. Find f(\{−2,0,3\}) and f⁻¹(\{0,3,8\}).
Solution. Compute the image:
f(−2) = 4−1 = 3, f(0)=−1, f(3)=9−1=8, so f(\{−2,0,3\}) = \{3, −1, 8\}.
For the preimage, solve n² − 1 ∈ \{0,3,8\}. That means n² ∈ \{1,4,9\}.
So n ∈ \{±1, ±2, ±3\}. Hence f⁻¹(\{0,3,8\}) = \{−3,−2,−1,1,2,3\}. ■

\subsection{7.2 Injective, surjective, bijective; inverse functions}
Injective / surjective / bijective
Injective (one-to-one): f(a₁)=f(a₂) ⇒ a₁=a₂.
Surjective (onto): for every b ∈ B there exists a ∈ A with f(a)=b.
Bijective: both injective and surjective (equivalently: has an inverse function).
For finite sets, injective and surjective are strongly linked: if |A|=|B| and f: A→B is injective,
then it is automatically surjective (and vice versa). For infinite sets this is not true, so
definitions matter.
\subsubsection*{Example 2.2 (Injective but not surjective)}
Define f: ℤ → ℤ by f(n)=2n. Show f is injective but not surjective.
Solution. If f(n₁)=f(n₂) then 2n₁=2n₂, so n₁=n₂; hence injective.
But f(n) is always even, so there is no n with f(n)=1. Therefore f is not onto ℤ. ■
\subsubsection*{Example 2.3 (Inverse function on ℝ)}
Let g: ℝ → ℝ be g(x)=x³. Find g⁻¹ and verify g(g⁻¹(y))=y.
Solution. Solve y=x³ for x: x=\(\sqrt[3]{y}\). So g⁻¹(y)=\(\sqrt[3]{y}\).
Then g(g⁻¹(y)) = g(\(\sqrt[3]{y}\)) = (\(\sqrt[3]{y}\))³ = y, as required. ■
\subsection{7.3 Composition of functions}
If f: A→B and g: B→C, the composition g∘f: A→C is defined by (g∘f)(a)=g(f(a)). Composition
is associative: h∘(g∘f) = (h∘g)∘f whenever types match. But composition is generally not
commutative: g∘f usually differs from f∘g.
\subsubsection*{Example 2.4 (Non-commutativity of composition)}
Let f(x)=2x+3 and g(x)=x² (both from ℝ to ℝ). Compute g∘f and f∘g.
Solution. (g∘f)(x)=g(2x+3)=(2x+3)²=4x²+12x+9.
Meanwhile (f∘g)(x)=f(x²)=2x²+3.
They are different functions, so composition is not commutative here. ■
A useful fact: if f and g are both bijections, then (g∘f) is a bijection and (g∘f)⁻¹ = f⁻¹∘g⁻¹.
The order reverses, just like for inverses of matrices.

\subsection{7.4 Cardinality and countability}
To compare the “sizes” of sets (including infinite sets), we use bijections. Two sets A and B
have the same cardinality if there exists a bijection f: A→B.
Countable vs. uncountable
A set is countably infinite if it has the same cardinality as ℕ (its elements can be
listed as a₀, a₁, a₂, …).
A set is countable if it is finite or countably infinite.
A set is uncountable if it is not countable (no listing captures all elements).
\subsubsection*{Example 2.5 (ℤ is countable)}
Show that ℤ is countably infinite.
Solution. One explicit listing is 0, 1, −1, 2, −2, 3, −3, …
Formally, define f: ℕ → ℤ by f(0)=0 and for n≥1:
if n is even, f(n)=n/2; if n is odd, f(n)=−(n+1)/2.
This hits every integer exactly once, so it is a bijection and ℤ is countable. ■
\subsubsection*{Example 2.6 (Diagonal argument idea)}
Why is the set of all infinite binary strings uncountable?
Sketch. Suppose, for contradiction, that you could list them as s₀, s₁, s₂, … where
each sᵢ is an infinite 0–1 sequence.
Build a new sequence t by flipping the diagonal bit: set t(i) = 1 if sᵢ(i)=0 and t(i)=0
if sᵢ(i)=1.
Then t differs from sᵢ at position i for every i, so t is not equal to any listed
sequence—contradiction.
Hence no complete listing exists, so the set is uncountable. ■
\subsection*{Week 2 problem set}
\textbf{M2-1.} Let A = \{1,2,3,4\} and B = \{a,b,c\}. Define f: A→B by f(1)=a, f(2)=b, f(3)=b,
f(4)=c. Find the range of f. Is f injective? Is it surjective?
\textbf{M2-2.} Let f: ℝ→ℝ be f(x)=x². Find f(\{−2,−1,0,3\}). Find f⁻¹(\{1,4\}).
\textbf{M2-3.} Prove: If f: A→B and g: B→C are injective, then g∘f is injective.

\textbf{M2-4.} Prove: If f: A→B and g: B→C are surjective, then g∘f is surjective.
\textbf{M2-5.} Let f(x)=3x−5. Find f⁻¹(x). Then compute (f⁻¹∘f)(x) and (f∘f⁻¹)(x).
\textbf{M2-6.} Give an example of two functions f and g such that f∘g is the identity on the
domain of g, but g∘f is not the identity on the domain of f.
\textbf{M2-7.} Show that the set E = \{2n : n ∈ ℕ\} of even natural numbers is countably infinite
by giving a bijection ℕ→E.
\textbf{M2-8.} Show that the interval (0,1) ⊆ ℝ is uncountable using a diagonal-style argument
with decimal expansions (be careful about 0.4999… = 0.5 type issues).

\clearpage
\section{Week 3: Relations and modular arithmetic}
\textbf{Reading:} Epp §8.1–8.4
\subsection*{Learning objectives}
\begin{itemize}[leftmargin=*,itemsep=0.25em]
\item Represent relations as sets of ordered pairs, directed graphs, and 0–1 matrices.
\item Test relations for reflexivity, symmetry, antisymmetry, and transitivity.
\item Work with equivalence relations and equivalence classes; move between partitions and equivalence relations.
\item Compute and use congruences (mod n), including modular inverses via the Euclidean algorithm.
\item Solve basic linear congruences and interpret “mod n” arithmetic in applications (coding/cryptography).
\end{itemize}
\subsection{8.1 Relations on sets}
A relation R from A to B is any subset of A×B. If (a,b) ∈ R we write aRb. When A=B we say
R is a relation on A.
Useful relation operations
Inverse relation: R⁻¹ = \{ (b,a) : (a,b) ∈ R \}.
Domain: dom(R) = \{ a ∈ A : ∃b, (a,b) ∈ R \}.
Range: ran(R) = \{ b ∈ B : ∃a, (a,b) ∈ R \}.
Relations can be represented in three common ways: (1) as a list of ordered pairs, (2) as a
directed graph (digraph) on the elements of A, and (3) as a 0–1 matrix M where M[i,j]=1 iff
(aᵢ,aⱼ)∈R.

\subsubsection*{Example 3.1 (Divisibility as a relation)}
Let A = \{1,2,3,4\}. Define R on A by xRy iff x divides y. List R.
Solution. 1 divides everything, so (1,1),(1,2),(1,3),(1,4) are in R.
2 divides 2 and 4: add (2,2),(2,4).
3 divides 3: add (3,3).
4 divides 4: add (4,4).
So R = \{(1,1),(1,2),(1,3),(1,4),(2,2),(2,4),(3,3),(4,4)\}. ■
\subsection{8.2 Reflexivity, symmetry, transitivity}
Core properties (for relations on A)
Reflexive: ∀a∈A, (a,a)∈R.
Symmetric: (a,b)∈R ⇒ (b,a)∈R.
Antisymmetric: if (a,b)∈R and (b,a)∈R then a=b.
Transitive: (a,b)∈R and (b,c)∈R ⇒ (a,c)∈R.
When a relation fails a property, it helps to exhibit a witness. For example, to show “not
symmetric,” find (a,b)∈R but (b,a)∉R.
\subsubsection*{Example 3.2 (Property check)}
For R = “divides” on A = \{1,2,3,4\}, is R reflexive? symmetric? antisymmetric?
transitive?
Solution. Reflexive: yes, because every number divides itself, so (a,a)∈R for all a.
Symmetric: no, since (1,2)∈R but (2,1)∉R.
Antisymmetric: yes. If a|b and b|a (with a,b positive integers), then a=b.
Transitive: yes. If a|b and b|c then a|c. ■
Sometimes you want the “closest” relation that does have a property. For example, the
reflexive closure of R is obtained by adding all missing (a,a) pairs; the symmetric closure
adds (b,a) whenever (a,b) is present; and the transitive closure adds all pairs forced by
repeated transitivity.

\subsection{8.3 Equivalence relations and equivalence classes}
An equivalence relation is a relation that is reflexive, symmetric, and transitive. Its
purpose is to formalize “sameness” under some rule (same remainder mod n, same
birthday month, same connected component, …).
Equivalence class
If R is an equivalence relation on A and a∈A, the equivalence class of a is [a] = \{
x∈A : xRa \}.
Equivalence classes form a partition of A: every element lies in exactly one class.
\subsubsection*{Example 3.3 (Equivalence classes mod 4)}
Define x ≡ y (mod 4) on ℤ by “4 divides x−y.” List the distinct equivalence classes.
Solution. Two integers are equivalent iff they have the same remainder when
divided by 4.
So there are 4 classes: [0]=\{…,−8,−4,0,4,8,…\}, [1]=\{…,−7,−3,1,5,9,…\},
[2]=\{…,−6,−2,2,6,10,…\}, [3]=\{…,−5,−1,3,7,11,…\}. ■
\subsection{8.4 Modular arithmetic and modular inverses}
For integers n≥2, we write a ≡ b (mod n) when n | (a−b). This is an equivalence relation
on ℤ, and arithmetic can be performed “mod n” by working with remainders.
Modular inverse
An integer a has a multiplicative inverse mod n if there exists x such that ax ≡ 1
(mod n).
Such an inverse exists iff gcd(a,n)=1. The extended Euclidean algorithm can find it.

\subsubsection*{Example 3.4 (Find an inverse mod 26)}
Find the inverse of 7 modulo 26.
Solution. We need x such that 7x ≡ 1 (mod 26). Try the Euclidean algorithm:
26 = 3·7 + 5, 7 = 1·5 + 2, 5 = 2·2 + 1.
Back-substitute: 1 = 5 − 2·2 = 5 − 2·(7 − 1·5) = 3·5 − 2·7.
And 5 = 26 − 3·7, so 1 = 3·(26 − 3·7) − 2·7 = 3·26 − 11·7.
Thus −11·7 ≡ 1 (mod 26), so x ≡ −11 ≡ 15 (mod 26).
Check: 7·15=105 ≡ 1 (mod 26). ■
\subsubsection*{Example 3.5 (Toy RSA workflow)}
This example is for arithmetic practice, not security. Let p=5 and q=11, so
n=pq=55 and φ(n)=(p−1)(q−1)=40.
Choose e=3 (coprime to 40). Find d such that ed ≡ 1 (mod 40).
Solution. We need 3d ≡ 1 (mod 40). One solution is d=27 because 3·27=81 ≡ 1
(mod 40).
To encrypt a message m (with 0 ≤ m < n), compute c ≡ me (mod n). To decrypt,
compute m ≡ cd (mod n).
In practice you would compute powers with fast modular exponentiation rather
than expanding huge numbers. ■
\subsection*{Week 3 problem set}
\textbf{M3-1.} Let A=\{1,2,3,4\}. Define R on A by (a,b)∈R iff a+b is even. List all ordered pairs in
R.
\textbf{M3-2.} For the relation R in M3-1, determine whether R is reflexive, symmetric,
antisymmetric, and transitive.
\textbf{M3-3.} Let A=\{a,b,c\}. Let R=\{(a,a),(b,b),(c,c),(a,b),(b,a)\}. Is R an equivalence relation? If
yes, list its equivalence classes.
\textbf{M3-4.} Given the partition of A=\{1,2,3,4,5,6\} into blocks \{1,4\}, \{2,5\}, \{3,6\}, write the
corresponding equivalence relation R (as a set of ordered pairs).
\textbf{M3-5.} Solve the congruence 9x ≡ 6 (mod 15). List all solutions modulo 15.
\textbf{M3-6.} Find the inverse of 17 modulo 43.

\textbf{M3-7.} Compute (12345 mod 7) and (12345 mod 11) without a calculator by reducing
step by step.
\textbf{M3-8.} A message is encrypted with the toy RSA setup p=5, q=11, e=3. If the plaintext is
m=12, compute the ciphertext c = m\textasciicircum{}e mod 55.

\clearpage
\section{Week 4: Counting and probability I}
\textbf{Reading:} Epp §9.1–9.4
\subsection*{Learning objectives}
\begin{itemize}[leftmargin=*,itemsep=0.25em]
\item Define sample spaces and events and compute probabilities in equally likely finite models.
\item Use the multiplication rule (fundamental counting principle) and possibility trees.
\item Use the addition rule and inclusion–exclusion for two sets/events.
\item Apply complementary counting to simplify problems.
\item Use the pigeonhole principle to prove existence statements.
\end{itemize}
\subsection{9.1 Sample spaces, events, and basic probability}
A probability model starts with a sample space S: the set of all possible outcomes. An
event is a subset E ⊆ S. In the simplest “equally likely” setting, each outcome has
probability 1/|S|, so P(E)=|E|/|S|.
\subsubsection*{Example 4.1 (Two heads in three coin flips)}
Flip a fair coin 3 times. What is the probability of getting exactly 2 heads?
Solution. The sample space has 2³=8 equally likely outcomes.
Exactly two heads occur in HHT, HTH, THH: 3 outcomes.
So P(exactly 2 heads)=3/8. ■
\subsection{9.2 Possibility trees and the multiplication rule}
The multiplication rule says: if a process has k stages, and stage i has nᵢ choices
regardless of earlier choices, then the total number of outcomes is n₁·n₂·…·n\_k. Possibility
trees are a diagrammatic way to track stages when the number of choices depends on
earlier choices.
\subsubsection*{Example 4.2 (Counting passwords)}
How many 6-character strings can be formed from the alphabet \{A,…,Z,0,…,9\} if
repetition is allowed?
Solution. There are 26+10=36 choices for each position. By the multiplication rule:
36⁶ strings. ■

\subsection{9.3 The addition rule and inclusion–exclusion (two sets)}
If A and B are disjoint sets (A ∩ B = ∅), then |A ∪ B| = |A| + |B|. If they overlap, you must
subtract the overlap once: |A ∪ B| = |A| + |B| − |A ∩ B|. The same formula holds for
probabilities: P(A ∪ B)=P(A)+P(B)−P(A∩B).
\subsubsection*{Example 4.3 (Counting divisible numbers 1–100)}
How many integers from 1 to 100 are divisible by 2 or 5?
Solution. Let A be multiples of 2, B be multiples of 5.
|A|=⌊100/2⌋=50, |B|=⌊100/5⌋=20, |A∩B|=multiples of 10: ⌊100/10⌋=10.
So |A∪B|=50+20−10=60. ■
\subsection{9.4 The pigeonhole principle}
The pigeonhole principle is a small idea with disproportionate power: if you put more
objects than boxes, then some box contains at least two objects. The generalized version
says: if N objects are placed into k boxes, then some box contains at least ⌈N/k⌉ objects.
\subsubsection*{Example 4.4 (Birth months)}
Show that in any group of 13 people, at least two were born in the same month.
Solution. There are 12 months (boxes) and 13 people (objects). By the pigeonhole
principle, some month contains at least two birthdays. ■
\subsection*{Week 4 problem set}
\textbf{M4-1.} A fair die is rolled once. What is the probability that the result is (a) even, (b)
greater than 4, (c) even or greater than 4?
\textbf{M4-2.} A student must choose a username consisting of 2 letters followed by 3 digits.
Letters may repeat; digits may repeat. How many usernames are possible?
\textbf{M4-3.} How many 5-letter strings over \{A,B,C\} contain at least one A? (Hint:
complement.)
\textbf{M4-4.} In a class, 18 students take math, 12 take CS, and 7 take both. How many take at
least one of the two courses?
\textbf{M4-5.} Use the pigeonhole principle to show that among any 6 integers, there are two
with the same remainder when divided by 5.
\textbf{M4-6.} Show that in any set of 10 distinct integers, there exist two whose difference is
divisible by 9.
\textbf{M4-7.} A bag contains 5 red, 4 blue, and 3 green balls. If you draw 2 balls without
replacement, what is the probability they are the same color?

\textbf{M4-8.} How many permutations of the letters in the word LEVEL are there? (Treat
identical letters as indistinguishable.)

\clearpage
\section{Week 5: Counting and probability II}
\textbf{Reading:} Epp §9.5–9.7
\subsection*{Learning objectives}
\begin{itemize}[leftmargin=*,itemsep=0.25em]
\item Use combinations (binomial coefficients) to count subsets and selections without order.
\item Solve “stars and bars” problems (combinations with repetition) correctly.
\item Apply Pascal’s identity and the binomial theorem.
\item Give combinatorial proofs of simple identities.
\item Use combinations to compute probabilities (hypergeometric-style counts).
\end{itemize}
\subsection{9.5 Combinations and binomial coefficients}
When order does not matter, we count combinations. The number of r-element subsets of
an n-element set is written “n choose r” and denoted C(n,r) or (n r).
Binomial coefficient
C(n,r) = n! / (r!(n−r)!) for integers 0 ≤ r ≤ n.
Interpretation: number of ways to choose r items from n distinct items.
\subsubsection*{Example 5.1 (Committee count)}
How many 4-person committees can be formed from 10 students?
Solution. Order does not matter, so the answer is C(10,4)=10!/(4!6!)=210. ■
A common mistake is to use permutations when combinations are needed. A quick check:
if the problem says “committee,” “subset,” “choose,” or “select” (without roles), it’s
usually combinations.
\subsection{9.6 Combinations with repetition (stars and bars)}
Sometimes you choose r items allowing repeats from n types. Equivalently, you distribute
r identical balls into n distinct boxes (allowing empty boxes).
Stars and bars
Number of solutions in nonnegative integers to x₁ + x₂ + … + x\_n = r is C(r+n−1,
n−1).
Interpretation: number of multisets of size r drawn from n types.

\subsubsection*{Example 5.2 (Distributing identical items)}
How many nonnegative integer solutions are there to x₁+x₂+x₃ = 10?
Solution. Here n=3 and r=10, so the count is C(10+3−1, 3−1)=C(12,2)=66. ■
\subsection{9.7 Pascal’s identity and the binomial theorem}
The binomial coefficients satisfy a recursion that matches Pascal’s triangle: C(n,r) =
C(n−1,r) + C(n−1,r−1). One way to understand it is combinatorial: when choosing r
elements from \{1,…,n\}, either you choose n or you don’t.
Binomial theorem
(x + y)ⁿ = Σ\_\{r=0\}\textasciicircum{}\{n\} C(n,r) x\textasciicircum{}\{n−r\} y\textasciicircum{}\{r\}.
It expands a power of a sum into a sum of terms weighted by binomial coefficients.
\subsubsection*{Example 5.3 (Coefficient extraction)}
Find the coefficient of x⁷ in (2x − 3)¹⁰.
Solution. Write (2x − 3)¹⁰ = Σ C(10,r) (2x)\textasciicircum{}\{10−r\} (−3)\textasciicircum{}\{r\}.
We want exponent 7 on x, so 10−r = 7 ⇒ r=3.
Coefficient = C(10,3)·2⁷·(−3)³ = 120·128·(−27) = −414,720. ■
\subsection*{Week 5 problem set}
\textbf{M5-1.} How many ways are there to choose 5 books from a shelf of 12 distinct books?
\textbf{M5-2.} A pizza shop offers 8 toppings. How many distinct 3-topping pizzas are possible
(assume no topping is repeated)?
\textbf{M5-3.} How many 6-card hands from a standard 52-card deck contain exactly 2 aces?
\textbf{M5-4.} How many nonnegative integer solutions are there to x₁+x₂+x₃+x₄ = 15?
\textbf{M5-5.} How many integer solutions are there to x₁+x₂+x₃ = 10 with xᵢ ≥ 2 for all i?
\textbf{M5-6.} Use Pascal’s identity to compute C(11,5) from smaller binomial coefficients.
\textbf{M5-7.} Prove the identity C(n,r) = C(n,n−r) combinatorially (in words).
\textbf{M5-8.} Use the binomial theorem to expand (x+1)⁵ and then evaluate the expansion at
x=1 to compute 2⁵.

\clearpage
\section{Week 6: Probability axioms, expected value,}
and intro graphs
\textbf{Reading:} Epp §9.8, §1.4, §4.9
\subsection*{Learning objectives}
\begin{itemize}[leftmargin=*,itemsep=0.25em]
\item Use the probability axioms to derive standard identities (complements, unions, bounds).
\item Compute expected values of discrete random variables and use linearity of expectation.
\item Model situations with graphs; compute degrees; recognize simple, complete, and bipartite graphs.
\item Apply the handshake theorem to relate degrees and edges and to prove impossibility results.
\end{itemize}
\subsection{9.8 Probability axioms and expected value}
In real applications, “equally likely outcomes” is too restrictive. Instead we define a
probability function P on a sample space S that assigns a real number P(E) to each event E
⊆ S and satisfies three axioms:
Probability axioms
1. 0 ≤ P(E) ≤ 1 for every event E.
2. P(S) = 1.
3. If E₁, E₂, … are pairwise disjoint, then P(E₁ ∪ E₂ ∪ …) = P(E₁)+P(E₂)+… (countable
additivity).
From these axioms you can derive the usual rules such as P(Eᶜ)=1−P(E) and
P(A∪B)=P(A)+P(B)−P(A∩B).
Expected value
A (discrete) random variable X assigns a number to each outcome.
If X takes values x₁,…,x\_k with probabilities p₁,…,p\_k, then E[X]=Σ xᵢ pᵢ.
Linearity: E[X+Y]=E[X]+E[Y] even when X and Y are dependent.

\subsubsection*{Example 6.1 (Expected value of a die)}
Let X be the result of rolling a fair six-sided die. Compute E[X].
Solution. X takes values 1,2,3,4,5,6 each with probability 1/6.
E[X] = (1+2+3+4+5+6)/6 = 21/6 = 3.5. ■
\subsubsection*{Example 6.2 (Linearity without independence)}
Two cards are drawn without replacement from a standard deck. Let X be the
number of aces in the two-card hand. Find E[X].
Solution. Let I₁ be the indicator that the first card is an ace, and I₂ the indicator
that the second card is an ace.
Then X = I₁ + I₂, so E[X]=E[I₁]+E[I₂].
Now E[I₁]=P(first is ace)=4/52=1/13. Also E[I₂]=P(second is ace)=4/52=1/13
(symmetry).
So E[X]=2/13. Note we did not need independence. ■
\subsection{1.4 Graph basics: the language of graphs}
A graph is a mathematical model for “objects connected by links.” A graph G consists of a
set V of vertices (nodes) and a set E of edges (connections). Edges may connect two
distinct vertices or (in some definitions) a vertex to itself (a loop).
Graph vocabulary (undirected graphs)
Vertices v and w are adjacent if \{v,w\} is an edge.
The degree deg(v) is the number of incident edges (count a loop twice).
A simple graph has no loops and no multiple edges between the same pair of
vertices.
A complete graph K\_n connects every pair of distinct vertices.
A bipartite graph has vertices split into two parts with edges only across the split.
\subsection{4.9 The handshake theorem}
The handshake theorem is one of the fastest ways to turn local degree information into
global conclusions about a graph.

Handshake theorem
Let G be any finite undirected graph with edge set E. Then Σ\_\{v∈V\} deg(v) = 2|E|.
Corollary: The sum of degrees is even.
Corollary: The number of vertices with odd degree is even.
\subsubsection*{Example 6.3 (Degree sequence sanity check)}
Can a simple graph have degree sequence (3,3,3,1)?
Solution. The sum of degrees is 3+3+3+1=10, which is even, so the handshake
theorem does not immediately rule it out.
But in a simple graph on 4 vertices, the maximum possible degree is 3. That part is
fine.
Try to realize it: if one vertex has degree 1, it is adjacent to exactly one other
vertex. The remaining three vertices must have degrees 3,3,3, which would force
each of them to connect to all other vertices—including the degree-1
vertex—giving it degree 3. Contradiction.
So no such simple graph exists. ■
\subsection*{Week 6 problem set}
\textbf{M6-1.} Use the probability axioms to prove: if A ⊆ B then P(A) ≤ P(B).
\textbf{M6-2.} Let A and B be events with P(A)=0.6, P(B)=0.5, and P(A∩B)=0.2. Compute P(A∪B)
and P(Aᶜ ∩ B).
\textbf{M6-3.} A game pays \$10 with probability 0.3 and pays \$0 otherwise. What is the
expected payout?
\textbf{M6-4.} A fair coin is flipped until the first head appears. Let X be the number of flips.
Compute E[X] (hint: use a geometric series or a known formula).
\textbf{M6-5.} Draw a simple graph with 6 vertices whose degrees are (3,3,2,2,2,0), or explain
why none exists.
\textbf{M6-6.} A graph has 13 edges and 9 vertices. The degrees of 8 vertices are:
2,2,3,3,3,4,4,4. What is the degree of the 9th vertex?
\textbf{M6-7.} Prove that in any undirected graph, the number of odd-degree vertices is even.
\textbf{M6-8.} How many edges does the complete graph K₈ have? Justify using degrees or
combinations.

\clearpage
\section{Week 7: Graph theory I: trails, matrices,}
isomorphism
\textbf{Reading:} Epp §10.1–10.3
\subsection*{Learning objectives}
\begin{itemize}[leftmargin=*,itemsep=0.25em]
\item Distinguish walks, trails, paths, circuits; recognize Euler trails/circuits and their degree conditions.
\item Build adjacency and incidence matrices and use matrix powers to count walks.
\item Use graph invariants (degree sequence, components, cycles) to test for non-isomorphism.
\item Construct an isomorphism when two graphs are isomorphic (give an explicit vertex mapping).
\end{itemize}
\subsection{10.1 Walks, trails, paths, circuits; Euler trails/circuits}
A walk is a sequence of vertices where consecutive vertices are joined by edges. A trail is
a walk with no repeated edges. A path is a walk with no repeated vertices. A circuit is a
walk that starts and ends at the same vertex.
Euler trails and Euler circuits (undirected graphs)
An Euler trail uses every edge exactly once (start and end may differ).
An Euler circuit is an Euler trail that starts and ends at the same vertex.
Criterion (connected graphs, ignoring isolated vertices):
\begin{itemize}[leftmargin=*,itemsep=0.25em]
\item Euler circuit exists iff every vertex has even degree.
\item Euler trail (but not circuit) exists iff exactly two vertices have odd degree.
\end{itemize}
\subsubsection*{Example 7.1 (Euler criterion)}
A connected graph has degrees (4,2,2,2,2). Does it have an Euler circuit?
Solution. All degrees are even, so an Euler circuit exists. ■
\subsection{10.2 Matrix representations of graphs}
Label the vertices v₁,…,v\_n. The adjacency matrix A is the n×n matrix where A[i,j] is the
number of edges between vᵢ and vⱼ (for a simple graph, 0 or 1). For undirected simple
graphs, A is symmetric and has zeros on the diagonal.

\subsubsection*{Example 7.2 (Adjacency matrix and counting walks)}
Let G be the path graph v₁—v₂—v₃. Its adjacency matrix is
A = [[0,1,0],[1,0,1],[0,1,0]].
Compute A² and interpret (A²)[1,3].
Solution. Multiply:
A² = [[1,0,1],[0,2,0],[1,0,1]].
The entry (A²)[1,3]=1 counts the number of length-2 walks from v₁ to v₃. Indeed
there is exactly one: v₁→v₂→v₃. ■
\subsection{10.3 Graph isomorphism}
Two graphs G and H are isomorphic if they are the same up to relabeling of vertices.
Formally, an isomorphism is a bijection φ: V(G)→V(H) such that \{u,v\} is an edge in G iff
\{φ(u),φ(v)\} is an edge in H.
Strategy checklist:
\begin{itemize}[leftmargin=*,itemsep=0.25em]
\item Compare easy invariants first: number of vertices, number of edges, degree multiset, number of components.
\item Compare structural features: triangles, cycles of given lengths, cut vertices, bipartiteness.
\item If invariants match, try to build an explicit vertex mapping that preserves adjacency.
\end{itemize}
\subsubsection*{Example 7.3 (Non-isomorphism via degrees)}
Graph G has degree multiset \{3,3,2,2,2\}. Graph H has degree multiset
\{4,2,2,2,1\}. Can they be isomorphic?
Solution. No. Isomorphic graphs have the same degree multiset. These differ, so G
and H are not isomorphic. ■
\subsection*{Week 7 problem set}
\textbf{M7-1.} A connected graph has exactly four vertices of odd degree. Explain why it cannot
have an Euler trail.
\textbf{M7-2.} Determine whether the complete bipartite graph K₃,₃ has an Euler circuit, an Euler
trail, or neither. Justify using degrees.
\textbf{M7-3.} For the cycle graph C₄ (a square), write the adjacency matrix (label vertices in
order around the cycle).

\textbf{M7-4.} Using your adjacency matrix from M7-3, compute A² and interpret the diagonal
entries.
\textbf{M7-5.} Give two non-isomorphic graphs on 6 vertices that have the same degree multiset
(a degree multiset alone is not a complete invariant).
\textbf{M7-6.} Let G be a simple graph with adjacency matrix A. Explain why the entry (A³)[i,i]
counts length-3 closed walks starting at vᵢ, and how triangles show up in A³.
\textbf{M7-7.} Decide whether the following degree sequence is graphical (i.e., can occur for a
simple graph): (3,3,3,1,1,1). If yes, draw such a graph; if not, justify.
\textbf{M7-8.} Two graphs each have 8 vertices and 10 edges. One has a vertex of degree 6; the
other has maximum degree 4. Are they isomorphic? Explain.

\clearpage
\section{Week 8: Trees and graph algorithms}
\textbf{Reading:} Epp §10.4–10.6
\subsection*{Learning objectives}
\begin{itemize}[leftmargin=*,itemsep=0.25em]
\item Recognize trees and use equivalent characterizations (connected + acyclic, unique simple path, edges=vertices−1).
\item Work with rooted trees (parent/child, levels, height) and m-ary trees.
\item Construct spanning trees and minimum spanning trees (MST) for weighted graphs.
\item Run a shortest-path algorithm (Dijkstra) on small weighted graphs and interpret results.
\end{itemize}
\subsection{10.4 Trees: examples and basic properties}
Informally, a tree is a connected graph with no cycles. Trees are the backbone of many
data structures and network designs because they are “just enough edges to stay
connected.”
Equivalent characterizations of a tree (finite, undirected)
For a graph with n vertices, the following are equivalent:
1) Connected and acyclic.
2) Connected and has exactly n−1 edges.
3) Acyclic and has exactly n−1 edges.
4) There is a unique simple path between any two vertices.
\subsubsection*{Example 8.1 (Edges in a tree)}
A tree has 18 vertices. How many edges does it have?
Solution. Any tree with n vertices has n−1 edges. Here n=18, so edges=17. ■
A common proof pattern: show that adding one edge to a tree creates exactly one cycle,
and removing any edge disconnects it. This is the reason spanning trees are “minimally
connected.”

\subsection{10.5 Rooted trees and m-ary trees}
A rooted tree is a tree with a distinguished vertex called the root. This induces parent/child
relationships: every vertex except the root has a unique parent (the next vertex on the
path to the root).
m-ary tree facts
A rooted tree is m-ary if each vertex has at most m children.
A rooted tree is full m-ary if every internal vertex has exactly m children.
In a full m-ary tree with i internal vertices, the number of leaves is L = (m−1)i + 1.
\subsubsection*{Example 8.2 (Leaves in a full binary tree)}
A full binary tree (m=2) has 10 internal vertices. How many leaves does it have?
Solution. Use L=(m−1)i+1 = (2−1)·10+1 = 11. ■
\subsection{10.6 Spanning trees and shortest paths}
Given a connected graph G, a spanning tree is a subgraph that is a tree and includes all
vertices of G. If edges have weights (costs), a minimum spanning tree (MST) is a spanning
tree with minimum total weight.
A standard MST method is Kruskal’s algorithm: sort edges by weight and keep adding the
next lightest edge that does not create a cycle.
For shortest paths from a source vertex in a graph with nonnegative weights, a standard
method is Dijkstra’s algorithm. It repeatedly “locks in” the next vertex with smallest
tentative distance.
Pseudocode: Dijkstra’s algorithm
Dijkstra(G, source s):
    for each vertex v:
        dist[v] = ∞
        prev[v] = None
    dist[s] = 0
    Q = set of all vertices
    while Q not empty:
        u = vertex in Q with smallest dist[u]
        remove u from Q
        for each neighbor v of u still in Q:
            alt = dist[u] + weight(u,v)
            if alt < dist[v]:
                dist[v] = alt
                prev[v] = u
    return dist, prev

\subsubsection*{Example 8.3 (Dijkstra on a tiny graph)}
Suppose edges have weights: s–a (2), s–b (5), a–b (1), a–t (6), b–t (2). Find the
shortest distance from s to t.
Solution (sketch). Start with dist[s]=0, others ∞.
Relax from s: dist[a]=2, dist[b]=5. Next pick a (2). Relax from a:
via a→b: dist[b] becomes min(5, 2+1)=3. via a→t: dist[t]=8.
Next pick b (3). Relax from b: dist[t] becomes min(8, 3+2)=5.
So the shortest distance s→t is 5, achieved by s→a→b→t. ■
\subsection*{Week 8 problem set}
\textbf{M8-1.} A connected graph has 15 vertices and 14 edges. It has no cycles. Prove it is a
tree.
\textbf{M8-2.} A tree has 9 vertices. If you add one new edge between two previously
non-adjacent vertices, how many cycles are created? Explain.
\textbf{M8-3.} In a full 3-ary tree, there are 8 internal vertices. How many leaves are there?
\textbf{M8-4.} Draw a rooted tree of height 2 where the root has 3 children, one of those
children has 2 children, and all other vertices are leaves. State the number of leaves.
\textbf{M8-5.} Given a connected graph with vertices \{1,2,3,4\} and edges \{12,13,14,23,34\},
find a spanning tree (list its edges).
\textbf{M8-6.} Run Kruskal’s algorithm on a graph with vertices \{a,b,c,d\} and weighted edges:
ab(1), ac(5), ad(4), bc(2), bd(6), cd(3). List the MST edges and total weight.
\textbf{M8-7.} Run Dijkstra’s algorithm from source s on the weighted graph: s–a(1), s–b(4),
a–b(2), a–t(6), b–t(3). Give the final dist values.
\textbf{M8-8.} Explain why Dijkstra’s algorithm requires nonnegative edge weights by giving a
small counterexample with a negative weight.

\clearpage
\section{Week 9: Regular expressions and finite-state}
automata
\textbf{Reading:} Epp §12.1–12.3
\subsection*{Learning objectives}
\begin{itemize}[leftmargin=*,itemsep=0.25em]
\item Define alphabets, strings, and languages; use Σ* and ε correctly.
\item Write and interpret regular expressions using union, concatenation, and Kleene star.
\item Design deterministic finite automata (DFA) for simple pattern constraints.
\item Simulate a DFA on an input string and decide acceptance.
\item Minimize a DFA by identifying equivalent (indistinguishable) states.
\end{itemize}
\subsection{12.1 Formal languages and regular expressions}
An alphabet Σ is a finite set of symbols. A string over Σ is a finite sequence of symbols
from Σ. The empty string is ε. The set of all strings over Σ is Σ*. A language is any subset
of Σ*.
Regular expression operators (core ones)
If R and S are regular expressions, then:
\begin{itemize}[leftmargin=*,itemsep=0.25em]
\item (R | S) denotes union (either a string from R or from S).
\item RS denotes concatenation (a string from R followed by a string from S).
\item R* denotes Kleene star (zero or more repetitions of strings from R).
\end{itemize}
\subsubsection*{Example 9.1 (A simple regex)}
Over Σ=\{a,b\}, give a regular expression for all strings that start with a and end
with b.
Solution. The middle can be any string over \{a,b\}, i.e., (a|b)*. So one answer is:
a(a|b)*b. ■
Regular expressions are good for describing patterns, but to reason algorithmically we
often convert them into automata.
\subsection{12.2 Deterministic finite-state automata (DFA)}
A deterministic finite automaton (DFA) consists of:

DFA definition (informal)
\begin{itemize}[leftmargin=*,itemsep=0.25em]
\item A finite set of states Q.
\item An input alphabet Σ.
\item A transition function δ: Q×Σ → Q (deterministic: exactly one next state).
\item A start state q₀ ∈ Q.
\item A set of accepting states F ⊆ Q. A string is accepted if, starting at q₀ and following transitions symbol by symbol, the final state is in F.
\end{itemize}
\subsubsection*{Example 9.2 (DFA for “even number of 1s” in binary strings)}
Design a DFA over Σ=\{0,1\} that accepts exactly those strings with an even
number of 1s.
Solution. Use two states: E (even so far) and O (odd so far). Start at E. On input 0,
stay in the same state; on input 1, toggle between E and O. Accepting state is E. ■
Transition table for Example 9.2
State | on 0 | on 1
------+------|------
  E   |  E   |  O
  O   |  O   |  E
\subsection{12.3 Simplifying (minimizing) finite-state automata}
Different DFAs can recognize the same language. Minimization produces an equivalent
DFA with as few states as possible. The key idea is state equivalence: two states p and q
are equivalent if no future input can distinguish them (they lead to acceptance or rejection
together for every continuation).
A practical minimization method is partition refinement: start by splitting states into
accepting vs non-accepting, then repeatedly split blocks when transitions on some symbol
go to different blocks.

\subsubsection*{Example 9.3 (Mini minimization example)}
Suppose a DFA has states \{A,B,C,D\}, accepting states \{C,D\}. If A and B both go to
C on input 0 and to D on input 1, and C and D both loop to themselves on both
symbols, then A and B are equivalent and can be merged.
Reason. Starting from A or B, after one symbol you land in C or D in the same way.
From then on, C and D behave deterministically. No string can separate A from B,
so they are indistinguishable. ■
\subsection*{Week 9 problem set}
\textbf{M9-1.} Over Σ=\{a,b\}, write a regular expression for all strings that contain the substring
“ab” at least once.
\textbf{M9-2.} Describe in words the language of the regular expression (a|b)*aa(a|b)* over
Σ=\{a,b\}.
\textbf{M9-3.} Construct a DFA over Σ=\{0,1\} that accepts strings that end with 01.
\textbf{M9-4.} For your DFA in M9-3, trace the computation on the input strings 01, 101, 1110,
and 1001. Indicate accept/reject.
\textbf{M9-5.} Design a DFA over Σ=\{0,1\} that accepts binary strings representing numbers
divisible by 3 (allow leading zeros).
\textbf{M9-6.} Give a regular expression over Σ=\{0,1\} for all strings of length at least 2 whose
first and last symbols are the same.
\textbf{M9-7.} Minimize the DFA with states \{S,A,B\}, alphabet \{0,1\}, start state S, accepting
states \{A\}, and transitions: δ(S,0)=A, δ(S,1)=B, δ(A,0)=A, δ(A,1)=B, δ(B,0)=A,
δ(B,1)=B. (Which states can be merged?)
\textbf{M9-8.} Explain why the language L = \{ a\textasciicircum{}n b\textasciicircum{}n : n ≥ 0 \} is not regular (give a
pigeonhole/pumping-lemma style argument, at a high level).

\clearpage
\section{Week 10: Analysis of algorithm efficiency}
\textbf{Reading:} Epp §11.1–11.5
\subsection*{Learning objectives}
\begin{itemize}[leftmargin=*,itemsep=0.25em]
\item Compare growth rates of common functions (logarithmic, polynomial, exponential).
\item Use big-O, big-Ω, and big-Θ definitions and prove simple bounds.
\item Analyze the worst-case step counts of basic algorithms (loops, nested loops).
\item Analyze divide-and-conquer algorithms (binary search, merge sort) at the level of growth rates.
\item Solve simple recurrences or recognize common forms (e.g., T(n)=2T(n/2)+n).
\end{itemize}
\subsection{11.1 Growth of functions: intuition and graphs}
When we analyze algorithms, we usually care about how running time grows with input
size n. At large n, exact constants matter less than growth rate. Common families (in
increasing growth order) include log n, n, n log n, n², n³, 2ⁿ, n!.
\subsubsection*{Example 10.1 (Which grows faster?)}
Which function grows faster as n→∞: n log₂ n or n\textasciicircum{}\{1.5\}?
Answer (reason). n\textasciicircum{}\{1.5\} grows faster. One way to see it: divide by n log n to
compare:
n\textasciicircum{}\{1.5\} / (n log n) = n\textasciicircum{}\{0.5\} / log n → ∞ because \(\sqrt{n}\) eventually dominates log n.
■
\subsection{11.2 Big-O, big-Ω, and big-Θ}
Asymptotic notation (core definitions)
f(n) is O(g(n)) if ∃C,n₀ such that for all n ≥ n₀, 0 ≤ f(n) ≤ C·g(n).
f(n) is Ω(g(n)) if ∃c,n₀ such that for all n ≥ n₀, 0 ≤ c·g(n) ≤ f(n).
f(n) is Θ(g(n)) if f(n) is both O(g(n)) and Ω(g(n)).
Think: O is an upper bound (“grows no faster than”), Ω is a lower bound (“grows at least
as fast as”), and Θ is a tight bound (“same order of growth”).

\subsubsection*{Example 10.2 (Prove a Θ bound)}
Show that f(n)=3n²+5n+1 is Θ(n²).
Solution. For n ≥ 1, we have 3n² ≤ 3n²+5n+1 ≤ 3n²+5n²+n² = 9n².
So with c=3, C=9, and n₀=1, we get 3n² ≤ f(n) ≤ 9n² for all n ≥ 1.
Hence f(n) is Θ(n²). ■
\subsection{11.3 Application: analyzing simple loops}
A practical approach: decide what a “basic operation” is (comparison, addition, array
access), then count how many times it executes as a function of n. You can then convert
that exact count into an asymptotic class.
Counting example: nested loop
\# Example: nested loop
count = 0
for i = 1 to n:
    for j = 1 to n:
        count = count + 1
The inner statement runs n times for each of n values of i, so total iterations = n·n = n².
This is Θ(n²).
\subsection{11.4 Exponential and logarithmic functions}
Logarithms turn multiplication into addition: log(ab)=log a + log b. This is why algorithms
that repeatedly halve a problem size often run in logarithmic time. Exponential functions
like 2ⁿ grow extremely fast and usually indicate brute-force behavior.
\subsubsection*{Example 10.3 (Binary search running time)}
Binary search halves the search interval each step. After k steps, the remaining
size is about n/2\textasciicircum{}k.
We stop when n/2\textasciicircum{}k ≈ 1, meaning 2\textasciicircum{}k ≈ n, so k ≈ log₂ n.
Thus binary search runs in Θ(log n) comparisons. ■
\subsection{11.5 Application: divide-and-conquer and recurrences}
Divide-and-conquer algorithms often lead to recurrences. A classic example is merge sort:

Merge sort recurrence (informal)
To sort n items, merge sort sorts two halves of size n/2 and then merges them in
linear time.
This gives T(n) = 2T(n/2) + cn, which solves to T(n) = Θ(n log n).
\subsubsection*{Example 10.4 (Solving T(n)=2T(n/2)+n by expansion)}
Assume n is a power of 2 and T(1)=1. Expand:
T(n)=2T(n/2)+n
=2[2T(n/4)+n/2]+n = 4T(n/4)+2n
=8T(n/8)+3n
After k steps: T(n)=2\textasciicircum{}k T(n/2\textasciicircum{}k)+k·n.
Stop when n/2\textasciicircum{}k=1 ⇒ k=log₂ n. Then:
T(n)=n·T(1)+(log₂ n)·n = n + n log₂ n = Θ(n log n). ■
\subsection*{Week 10 problem set}
\textbf{M10-1.} Order the following functions from slowest-growing to fastest-growing (as n→∞):
log n, n, n log n, n², 2ⁿ.
\textbf{M10-2.} Show that 7n+20 is Θ(n).
\textbf{M10-3.} Show that n² is O(n³) and that n³ is not O(n²).
\textbf{M10-4.} A loop runs i from 1 to n, and inside it runs j from 1 to i. How many times does
the inner statement execute? Give a Θ bound.
\textbf{M10-5.} In the worst case, sequential search of an array of length n checks every
element. Give the exact number of comparisons and its Θ class.
\textbf{M10-6.} Binary search on a sorted array of length n makes at most ⌈log₂(n+1)⌉
comparisons (up to constants). Explain why this is O(log n).
\textbf{M10-7.} Solve the recurrence T(n)=T(n/2)+n with T(1)=1 for n a power of 2. Give Θ
classification.
\textbf{M10-8.} Which grows faster: n log n or 10n? At approximately what n does n log₂ n
exceed 10n? (A rough estimate is fine.)

\appendix
\clearpage
\section{Appendix: Solutions to problem sets}
The solutions below correspond to the labeled exercises in Weeks 1–10. If you find
yourself reading a solution and thinking “I see it,” stop and re-derive the key step on your
own—that’s where most of the learning happens.
\subsection{Week 1 solutions}
\textbf{M1-1.} Compute each set step by step (universe U=\{1,2,3,4,5\}).
(a) A ∪ B = \{1,2,4\} ∪ \{2,3,5\} = \{1,2,3,4,5\}.
(b) A ∩ C = \{1,2,4\} ∩ \{1,3,4\} = \{1,4\}.
(c) A \(\setminus\) B = \{ elements of A not in B \} = \{1,4\}.
(d) First B ∪ C = \{2,3,5\} ∪ \{1,3,4\} = \{1,2,3,4,5\}=U, so (B ∪ C)ᶜ = ∅.
\textbf{M1-2.} Let x be arbitrary and assume x ∈ A ∩ B. By definition of intersection, x ∈ A and x ∈
B. In particular, x ∈ A. Therefore every element of A ∩ B is an element of A, so A ∩ B ⊆ A.
\textbf{M1-3.} We prove set equality by the element method. Let x be arbitrary.
x ∈ (A \(\setminus\) B) \(\setminus\) C ⇔ x ∈ (A \(\setminus\) B) and x ∉ C ⇔ (x ∈ A and x ∉ B) and x ∉ C.
This is equivalent to x ∈ A and (x ∉ B and x ∉ C), i.e., x ∈ A and x ∉ (B ∪ C).
Thus x ∈ (A \(\setminus\) B) \(\setminus\) C ⇔ x ∈ A \(\setminus\) (B ∪ C). Since x was arbitrary, the sets are equal.
\textbf{M1-4.} True. Suppose A ⊆ B. Let S ∈ P(A). By definition of power set, S ⊆ A. Since A ⊆ B,
transitivity of ⊆ gives S ⊆ B. Hence S ∈ P(B).
So every element of P(A) is an element of P(B), i.e., P(A) ⊆ P(B).
\textbf{M1-5.} A set with n elements has 2ⁿ subsets (each element is either “in” or “out”). With
n=9, that is 2⁹ = 512 subsets.
\textbf{M1-6.} Yes. Each block is nonempty; the blocks are pairwise disjoint; and their union is
\{1,2\}∪\{3\}∪\{4\} = \{1,2,3,4\}. Therefore it is a partition of A.
\textbf{M1-7.} Use the distributive identity (X ∪ Y) ∩ (X ∪ Z) = X ∪ (Y ∩ Z). With X=A, Y=B, Z=Bᶜ:
(A ∪ B) ∩ (A ∪ Bᶜ) = A ∪ (B ∩ Bᶜ) = A ∪ ∅ = A.
\textbf{M1-8.} The statement “for all x, if x ∈ X then x ∈ Y” means every element of X is also an
element of Y. That is exactly the subset relation X ⊆ Y.
\subsection{Week 2 solutions}
\textbf{M2-1.} The range is the set of outputs: \{f(1),f(2),f(3),f(4)\} = \{a,b,b,c\} = \{a,b,c\}.
Not injective because f(2)=b=f(3) but 2≠3.
Surjective because every element of B=\{a,b,c\} appears as an output.
\textbf{M2-2.} Compute the image: f(\{−2,−1,0,3\}) = \{ (−2)², (−1)², 0², 3² \} = \{4,1,0,9\}.

For the preimage, solve x² ∈ \{1,4\}. That happens when x=±1 or x=±2. So f⁻¹(\{1,4\}) =
\{−2,−1,1,2\}.
\textbf{M2-3.} Let f: A→B and g: B→C be injective. To show g∘f is injective, assume
(g∘f)(a₁)=(g∘f)(a₂).
Then g(f(a₁)) = g(f(a₂)). Since g is injective, f(a₁)=f(a₂). Since f is injective, a₁=a₂. Hence
g∘f is injective.
\textbf{M2-4.} Let f: A→B and g: B→C be surjective. We must show g∘f is surjective onto C.
Take any c ∈ C. Because g is onto, there exists b ∈ B with g(b)=c. Because f is onto, there
exists a ∈ A with f(a)=b.
Then (g∘f)(a) = g(f(a)) = g(b) = c. Hence g∘f is surjective.
\textbf{M2-5.} Solve y = 3x − 5 for x: x = (y+5)/3. So f⁻¹(x) = (x+5)/3.
Now (f⁻¹∘f)(x) = f⁻¹(3x−5) = ((3x−5)+5)/3 = x.
And (f∘f⁻¹)(x) = f((x+5)/3) = 3·(x+5)/3 − 5 = x.
\textbf{M2-6.} Here is one concrete example. Let A=\{1\} and B=\{1,2\}. Define g: A→B by g(1)=1.
Define f: B→A by f(1)=1 and f(2)=1.
Then (f∘g)(1)=f(g(1))=f(1)=1, so f∘g is the identity on A.
But g∘f is not the identity on B, because (g∘f)(2)=g(f(2))=g(1)=1 ≠ 2.
\textbf{M2-7.} A bijection ℕ→E is f(n)=2n (assuming ℕ=\{0,1,2,…\}).
It is one-to-one because 2n₁=2n₂ ⇒ n₁=n₂, and onto because every even number has the
form 2n.
\textbf{M2-8.} Assume, for contradiction, that (0,1) is countable. Then we can list its elements: x₁,
x₂, x₃, … with decimal expansions.
Construct a new number y in (0,1) by choosing its digits so that the n-th digit of y differs
from the n-th digit of xₙ (e.g., change 1→2 and 2→1, and avoid using 9).
Then y differs from xₙ in the n-th decimal place for every n, so y is not equal to any xₙ. This
contradicts that the list contained all numbers in (0,1).
Therefore (0,1) is uncountable.
\subsection{Week 3 solutions}
\textbf{M3-1.} The sum a+b is even exactly when a and b have the same parity (both odd or both
even).
Odd elements are 1 and 3, giving pairs (1,1),(1,3),(3,1),(3,3).
Even elements are 2 and 4, giving pairs (2,2),(2,4),(4,2),(4,4).
So R has 8 ordered pairs total.
\textbf{M3-2.} Reflexive: yes, because a+a is always even, so (a,a)∈R for every a.

Symmetric: yes, because if a+b is even then b+a is even.
Antisymmetric: no, since (1,3) and (3,1) are both in R but 1≠3.
Transitive: yes. If a+b and b+c are even, then a and b have the same parity and b and c
have the same parity, so a and c have the same parity, hence a+c is even and (a,c)∈R.
\textbf{M3-3.} R is reflexive because (a,a),(b,b),(c,c) are included. It is symmetric because (a,b)
and (b,a) are both included, and all (x,x) are symmetric.
For transitivity, the only nontrivial checks involve a and b: since aRb and bRa, we also
have aRa and bRb, which are in R. Anything involving c only relates c to itself.
So R is an equivalence relation. The equivalence classes are \{a,b\} and \{c\}.
\textbf{M3-4.} The equivalence relation consists of all pairs within each block:
\{(1,1),(1,4),(4,1),(4,4)\} ∪ \{(2,2),(2,5),(5,2),(5,5)\} ∪ \{(3,3),(3,6),(6,3),(6,6)\}.
\textbf{M3-5.} Solve 9x ≡ 6 (mod 15). Compute gcd(9,15)=3 and 3 divides 6, so solutions exist.
Divide the congruence by 3: 3x ≡ 2 (mod 5).
The inverse of 3 mod 5 is 2 (since 3·2=6≡1 mod5). Multiply both sides by 2:
x ≡ 4 (mod 5).
Thus modulo 15, the solutions are x ∈ \{4, 9, 14\}.
\textbf{M3-6.} Use the Euclidean algorithm: 43=2·17+9, 17=1·9+8, 9=1·8+1, so gcd(17,43)=1.
Back-substitute: 1=9−1·8 = 9−(17−1·9)=2·9−17 = 2·(43−2·17)−17 = 2·43−5·17.
So −5·17 ≡ 1 (mod 43), meaning 17⁻¹ ≡ −5 ≡ 38 (mod 43).
\textbf{M3-7.} Reduce by division:
12345 = 7·1763 + 4, so 12345 mod 7 = 4.
12345 = 11·1122 + 3, so 12345 mod 11 = 3.
\textbf{M3-8.} Compute c = m\textasciicircum{}e mod 55 with m=12 and e=3:
12³ = 1728. Divide by 55: 55·31=1705, remainder 23.
So the ciphertext is c = 23.
\subsection{Week 4 solutions}
\textbf{M4-1.} Sample space size is 6 (outcomes 1–6).
(a) Even outcomes: \{2,4,6\} so P=3/6=1/2.
(b) Greater than 4: \{5,6\} so P=2/6=1/3.
(c) Even or greater than 4: \{2,4,5,6\} so P=4/6=2/3.
\textbf{M4-2.} Two letters: 26·26 choices. Three digits: 10·10·10 choices. Multiply:
Total usernames = 26²·10³ = 676·1000 = 676,000.

\textbf{M4-3.} Total 5-letter strings over \{A,B,C\}: 3⁵=243.
Complement: strings with no A use only \{B,C\}, so 2⁵=32.
Therefore strings with at least one A: 243−32=211.
\textbf{M4-4.} Use inclusion–exclusion: |Math ∪ CS| = |Math|+|CS|−|Both| = 18+12−7 = 23.
\textbf{M4-5.} There are 5 possible remainders modulo 5 (0,1,2,3,4) but 6 integers. By the
pigeonhole principle, two integers share a remainder mod 5.
\textbf{M4-6.} There are 9 possible remainders modulo 9, but you have 10 distinct integers. Two
must share the same remainder mod 9, and their difference is divisible by 9.
\textbf{M4-7.} Total ways to choose 2 balls: C(12,2)=66.
Favorable ways (same color): C(5,2)+C(4,2)+C(3,2)=10+6+3=19.
So the probability is 19/66.
\textbf{M4-8.} LEVEL has 5 letters with L repeated twice and E repeated twice.
Distinct permutations = 5!/(2!·2!) = 120/4 = 30.
\subsection{Week 5 solutions}
\textbf{M5-1.} Choosing 5 distinct books from 12 without order gives C(12,5) = 792.
\textbf{M5-2.} A 3-topping pizza corresponds to choosing 3 toppings out of 8, order irrelevant:
C(8,3)=56.
\textbf{M5-3.} Choose 2 of the 4 aces and 4 of the 48 non-aces:
Number of hands = C(4,2)·C(48,4) = 6·194,580 = 1,167,480.
\textbf{M5-4.} Count nonnegative solutions to x₁+x₂+x₃+x₄=15 using stars and bars:
C(15+4−1,4−1)=C(18,3)=816.
\textbf{M5-5.} Let yᵢ = xᵢ−2, so yᵢ ≥ 0 and y₁+y₂+y₃ = 10−6 = 4.
Number of solutions = C(4+3−1,3−1)=C(6,2)=15.
\textbf{M5-6.} Pascal’s identity: C(11,5)=C(10,5)+C(10,4)=252+210=462.
\textbf{M5-7.} Combinatorial proof: C(n,r) counts r-element subsets of an n-element set.
Choosing r elements to include is equivalent to choosing which n−r elements to exclude,
giving C(n,n−r).
Therefore C(n,r)=C(n,n−r).
\textbf{M5-8.} (x+1)⁵ = Σ\_\{r=0\}\textasciicircum{}5 C(5,r) x\textasciicircum{}\{5−r\} = x⁵+5x⁴+10x³+10x²+5x+1.
Plug in x=1: 1+5+10+10+5+1 = 32 = 2⁵.

\subsection{Week 6 solutions}
\textbf{M6-1.} Assume A ⊆ B. Then B can be written as a disjoint union B = A ∪ (B\(\setminus\)A).
By additivity on disjoint events, P(B)=P(A)+P(B\(\setminus\)A) ≥ P(A). Hence P(A) ≤ P(B).
\textbf{M6-2.} Use inclusion–exclusion: P(A∪B)=P(A)+P(B)−P(A∩B)=0.6+0.5−0.2=0.9.
Also Aᶜ∩B is the part of B outside A, so P(Aᶜ∩B)=P(B)−P(A∩B)=0.5−0.2=0.3.
\textbf{M6-3.} Let X be the payout. Then E[X]=10·0.3 + 0·0.7 = 3. So the expected payout is \$3.
\textbf{M6-4.} Let X be the number of flips until the first head. Condition on the first flip:
With probability 1/2 you get a head immediately (X=1). With probability 1/2 you get a tail
and then you “start over,” so X=1+X′ where X′ has the same distribution as X.
So E[X] = (1/2)·1 + (1/2)·(1 + E[X]).
Solve: E[X] = 1/2 + 1/2 + (1/2)E[X] ⇒ (1/2)E[X]=1 ⇒ E[X]=2.
\textbf{M6-5.} One construction with vertices v₁,…,v₆ (v₆ isolated):
Edges among v₁,…,v₅: v₁v₂, v₁v₃, v₁v₄, v₂v₃, v₂v₅, v₄v₅.
Degrees are: deg(v₁)=3, deg(v₂)=3, deg(v₃)=2, deg(v₄)=2, deg(v₅)=2, deg(v₆)=0, as
required.
\textbf{M6-6.} By the handshake theorem, the sum of all degrees is 2|E| = 2·13 = 26.
The given 8 degrees sum to 2+2+3+3+3+4+4+4 = 25.
So the 9th degree must be 26−25 = 1.
\textbf{M6-7.} Let the degrees be d₁,…,d\_n. The handshake theorem gives d₁+…+d\_n = 2|E|,
which is even.
A sum of integers is even iff there are an even number of odd addends. Therefore the
number of odd-degree vertices is even.
\textbf{M6-8.} In K₈ every vertex has degree 7, so the sum of degrees is 8·7=56.
By the handshake theorem, 56 = 2|E|, so |E|=28.
(Equivalently: choose 2 endpoints out of 8: C(8,2)=28 edges.)
\subsection{Week 7 solutions}
\textbf{M7-1.} An Euler trail exists in a connected graph iff exactly 0 or 2 vertices have odd degree.
If the graph has 4 odd-degree vertices, it fails this criterion, so it cannot have an Euler trail.
\textbf{M7-2.} In K₃,₃ each vertex connects to all 3 vertices in the opposite part, so every vertex
has degree 3 (odd).
There are 6 odd-degree vertices. An Euler circuit requires all degrees even, and an Euler
trail requires exactly two odd degrees.
Therefore K₃,₃ has neither an Euler circuit nor an Euler trail.

\textbf{M7-3.} Label the cycle vertices 1–2–3–4–1. The adjacency matrix is:
A = [[0,1,0,1],[1,0,1,0],[0,1,0,1],[1,0,1,0]].
\textbf{M7-4.} For the matrix A in M7-3, multiply A² = A·A to get:
A² = [[2,0,2,0],[0,2,0,2],[2,0,2,0],[0,2,0,2]].
The diagonal entry (A²)[i,i]=2 counts length-2 walks that start and end at vertex i: go to
either neighbor and return.
\textbf{M7-5.} Example pair on 6 vertices with the same degree multiset but not isomorphic:
\begin{itemize}[leftmargin=*,itemsep=0.25em]
\item G = cycle C₆ (one 6-cycle). It is connected and every vertex has degree 2.
\item H = disjoint union of two triangles K₃ ∪ K₃. It has two components and every vertex has degree 2. Both have degree multiset \{2,2,2,2,2,2\} but they are not isomorphic because one is connected and the other is not. M7-6. The entry (A\textasciicircum{}k)[i,j] counts the number of length-k walks from vᵢ to vⱼ (this follows from how matrix multiplication sums over intermediate steps). So (A³)[i,i] counts length-3 closed walks starting at vᵢ. A triangle contributes closed walks of length 3: from each of its 3 vertices you can traverse the triangle in 2 directions, giving 6 closed walks total per triangle. Therefore the total number of triangles can be extracted from trace(A³)/6 in a simple graph. M7-7. Yes, it is graphical. One explicit construction: take vertices A,B,C as a triangle (edges AB,BC,CA) so each has degree 2, then connect A to D, B to E, and C to F. Degrees become: deg(A)=3, deg(B)=3, deg(C)=3, deg(D)=deg(E)=deg(F)=1, which matches (3,3,3,1,1,1). M7-8. No. The maximum degree is an isomorphism invariant. One graph has a vertex of degree 6 while the other has maximum degree 4, so they cannot be isomorphic.
\end{itemize}
\subsection{Week 8 solutions}
\textbf{M8-1.} A tree is defined as a connected graph with no cycles. The graph is given to be
connected and acyclic, so it is a tree.
(The extra fact “15 vertices and 14 edges” is consistent with the tree property |E|=|V|−1.)
\textbf{M8-2.} Exactly one cycle is created. In a tree there is a unique simple path between any two
vertices u and v.
Adding a new edge \{u,v\} closes that unique path into a single cycle, and no other cycle
can appear because all other edges are unchanged.
\textbf{M8-3.} For a full m-ary tree with i internal vertices, the number of leaves is L=(m−1)i+1.
Here m=3 and i=8, so L=(3−1)·8+1=16+1=17 leaves.

\textbf{M8-4.} One valid shape: root r has children A,B,C. Vertex A has children A₁ and A₂. Vertices
B,C,A₁,A₂ are leaves.
Number of leaves = 4.
\textbf{M8-5.} One spanning tree is the star at 1: edges \{12,13,14\}.
It contains all 4 vertices, has 3 edges, and has no cycle, so it is a spanning tree.
\textbf{M8-6.} Sort edges by weight: ab(1), bc(2), cd(3), ad(4), ac(5), bd(6).
Kruskal picks ab, then bc (no cycle), then cd (no cycle). Now all 4 vertices are connected
with 3 edges, so we stop.
MST edges: \{ab, bc, cd\}. Total weight = 1+2+3 = 6.
\textbf{M8-7.} Run Dijkstra from s. Initialize dist[s]=0, dist[a]=1, dist[b]=4, dist[t]=∞.
Pick a next (dist 1). Relax: b can be improved via a: 1+2=3 so dist[b]=3. t via a: dist[t]=7.
Pick b next (dist 3). Relax t via b: 3+3=6 so dist[t]=6.
Final distances: dist[s]=0, dist[a]=1, dist[b]=3, dist[t]=6.
\textbf{M8-8.} Counterexample: vertices s,a,b with edges s→a (1), s→b (2), b→a (−2).
The true shortest distance from s to a is 0 via s→b→a. But Dijkstra would finalize a at
distance 1 before considering the negative edge, so it can fail.
This is why Dijkstra requires all edge weights to be nonnegative.
\subsection{Week 9 solutions}
\textbf{M9-1.} A standard answer is (a|b)*ab(a|b)*. It says: any prefix, then “ab”, then any suffix.
\textbf{M9-2.} (a|b)*aa(a|b)* describes all strings over \{a,b\} that contain “aa” as a contiguous
substring at least once.
\textbf{M9-3.} Use states that remember whether the most recent symbol was 0 and whether the
last two symbols were 01.
States: q0 (start / last symbol not 0), q1 (last symbol is 0), q2 (last two symbols are 01).
Accepting state: q2.
Transitions: δ(q0,0)=q1, δ(q0,1)=q0; δ(q1,0)=q1, δ(q1,1)=q2; δ(q2,0)=q1, δ(q2,1)=q0.
\textbf{M9-4.} Trace using the DFA from M9-3:
01: q0→q1→q2 (accept).
101: q0→q0→q1→q2 (accept).
1110: q0→q0→q0→q0→q1 (reject).
1001: q0→q0→q1→q1→q2 (accept).
\textbf{M9-5.} Use 3 states for remainders mod 3: r0 (accept), r1, r2. Reading a bit b updates the
remainder by new = (2·old + b) mod 3.

Transitions:
r0: on 0→r0, on 1→r1;
r1: on 0→r2, on 1→r0;
r2: on 0→r1, on 1→r2.
\textbf{M9-6.} Strings of length at least 2 whose first and last symbols match are either:
\begin{itemize}[leftmargin=*,itemsep=0.25em]
\item start and end with 0: 0(0|1)*0, or
\item start and end with 1: 1(0|1)*1. So one regex is: 0(0|1)*0 | 1(0|1)*1. M9-7. Start with partition \{A\} (accepting) and \{S,B\} (non-accepting). Check S and B: on 0 both go to A; on 1 both go to B, which is in the same non-accepting block. So S and B are equivalent and can be merged. The minimized DFA has 2 states: one accepting state A and one non-accepting state [S,B]. M9-8. High-level pumping argument: suppose a DFA with k states recognized L=\{a\textasciicircum{}n b\textasciicircum{}n\}. Consider the k+1 strings a, a², …, a\textasciicircum{}\{k+1\}. As you read these prefixes, by pigeonhole two different lengths a\textasciicircum{}i and a\textasciicircum{}j (i < j) lead to the same state. From that state, the DFA cannot “remember” how many a’s were read, so it would accept both a\textasciicircum{}i b\textasciicircum{}i and a\textasciicircum{}j b\textasciicircum{}i (or reject both), contradicting that only equal counts should be accepted. Therefore L is not regular.
\end{itemize}
\subsection{Week 10 solutions}
\textbf{M10-1.} From slowest to fastest growth as n→∞: log n, n, n log n, n², 2ⁿ.
\textbf{M10-2.} For n ≥ 1, 7n ≤ 7n+20 ≤ 7n+20n = 27n.
So with c=7, C=27, and n₀=1, we have c·n ≤ 7n+20 ≤ C·n for all n ≥ n₀, hence 7n+20 is
Θ(n).
\textbf{M10-3.} For n ≥ 1, n² ≤ n³, so n² is O(n³) (take C=1, n₀=1).
To see n³ is not O(n²), suppose n³ ≤ Cn² for all n ≥ n₀. Dividing by n² gives n ≤ C for all
large n, impossible. So n³ is not O(n²).
\textbf{M10-4.} The inner statement runs 1+2+…+n times = n(n+1)/2. Therefore it is Θ(n²).
\textbf{M10-5.} Worst-case sequential search checks every element once, so it makes exactly n
comparisons.
Thus the running time is Θ(n).
\textbf{M10-6.} Ceilings do not change asymptotic class, so it suffices to bound log₂(n+1).

For n ≥ 1, n+1 ≤ 2n, so log₂(n+1) ≤ log₂(2n) = 1 + log₂ n.
Hence ⌈log₂(n+1)⌉ is O(log n).
\textbf{M10-7.} Assume n=2\textasciicircum{}k. Expand:
T(n)=T(n/2)+n = T(n/4)+n/2+n = … = T(1) + (n + n/2 + n/4 + … + 2).
The geometric sum is 2n−2, so T(n)=1+(2n−2)=2n−1. Therefore T(n)=Θ(n).
\textbf{M10-8.} n log n grows faster than 10n because log n → ∞.
Using base-2 logs: n log₂ n > 10n when log₂ n > 10, i.e., n > 2¹⁰ = 1024.
So around n≈1024, n log₂ n starts exceeding 10n (roughly).

\end{document}
